% !TEX root =  master.tex
\chapter{Anhang}\label{anhang}

%\section{Konzernstruktur großer Versicherer}
%\label{sec:KonzernStrukturen}

%\begin{figure}[h]
  %\centering
  %\includegraphics[width=1\textwidth]{img/beteiligungsstruktur-axa-konzern.pdf}
  %\caption[]{AXA SE - Konzernstruktur (AXA Deutschland)}
  %\label{fig:AXAKstr}
%\end{figure}
%\newpage

\section{Suchbegriffe der Literaturanalyse}
\begin{table}[ht]
    \begin{center}
    \begin{tabular}{|p{3cm}|p{3cm}|p{3,5cm}|p{3,5cm}|}
        \hline Sprache & Suchbegriff 1 & Suchbegriff 2 & Suchbegriff 3\\[0.5ex]
        \hline Deutsch & Kfz-Versicherer, Versicherer & Anforderungen & Digitale Plattformen, technische Plattformen, PaaS, APaaS, IT-Plattformen\\
        \hline Deutsch & Kfz-Versicherer, Versicherer & Herausforderungen, Probleme & -\\
        \hline Englisch & car insurance, motor insurance, insurance & requirements & digital platforms, technological platforms, PaaS, APaaS, IT-platforms\\
        \hline Englisch & car insurance, motor insurance, insurance & challenges, problems, issues & -\\
        \hline
    \end{tabular}
    \end{center}
    \caption{Suchbegriffe der Literaturanalyse}
    \label{tab:suchbegriffe}
\end{table}  

\newpage
\section{Kriterien der Literaturanalyse}
\begin{table}[ht]
    \begin{center}
    \begin{tabular}{|p{4,33cm}|p{4,33cm}|p{4,33cm}|}
        \hline  & Einschlusskriterien & Ausschlusskriterien \\[0.5ex]
        \hline Sprache & Deutsch, Englisch & Weitere Sprachen\\
        \hline Dokumententypen & Fachliteratur, internationale und nationale wissenschaftliche Zeitschriften, Studien, Magazine, Internetartikel & Weitere Veröffentlichungen\\
        \hline Publikationszeitraum & Veröffentlichungen von 2023 - einschließlich 2013 & Veröffentlichung vor 2013\\
        \hline Verfügbarkeit & Volltext verfügbar & Volltext nicht verfügbar\\
        \hline
    \end{tabular}
    \end{center}
    \caption{Kriterien der Literaturanalyse}
    \label{tab:kriterien}
\end{table}   
\newpage

\section{Zertifizierungen der SAP BTP Services}
\label{sec:BTPZertifikate}
\begin{figure}[h]
    \centering
    \includegraphics[width=1\textwidth]{img/BTP_Zertifikate.pdf}
    %\caption[]{Zertifizierungen der SAP BTP Services (SAP SE 2023)}
    \label{fig:BTPZertifikate}
  \end{figure}
\newpage

\section{Fragenkatalog der Leitfadeninterviews}
\label{sec:Fragenkatalog}
Im laufenden Gespräch wurde die Anzahl gestellter Fragen und deren Formulierung variiert. Aufnahme und Transkription fanden über die die MS-Teams-Funktion statt.

\begin{table}[H]
\begin{tabularx}{\linewidth}{lX}
\multicolumn{2}{l}{(A) Informationsphase} \\\hline \hline 
    \    & Ich führe das Interview im Rahmen meiner Projektarbeit 2 durch, welche den Titel: \enquote{Digitale Plattformen als Wettbewerbsfaktor für den deutschen Kfz-Versicherungsmarkt am Beispiel der SAP Business Technology Platform} trägt. Das Ziel dieses Interviews ist es, Anforderungen der Kfz-Versicherer an digitale Plattformen zu identifizieren und zu priorisieren. \\\hline
    \    & Dabei sind im Rahmen meiner Arbeit mit dem Begriff digitale Plattform, technologische Plattformen gemeint, welche eine Menge von Kernprodukten, -technologien oder -services bereitstellen, auf deren Basis weitere komplementäre Produkte und Services entwickelt werden können. Somit bezeichnet der Begriff digitale Plattform im Rahmen meiner Arbeit technische Plattformen wie die SAP BTP nicht aber transaktionsorientierte Plattformen wie bspw. Uber.\\\hline
    \\     
\multicolumn{2}{l}{(B) Einstiegsphase}  \\\hline \hline
    \  1 & Welche Erfahrungen hast Du in der Versicherungsbranche? Wie sieht Dein Werdegang aus und was ist Deine aktuelle Position im Unternehmen? \\\hline     
    \\  
\multicolumn{2}{l}{(C) Hauptphase}  \\\hline \hline
    \  2 & Welche Entwicklungen waren in der Kfz-Versicherungsbranche in den letzten Jahren erkennbar und mit welchen Herausforderungen hat die Branche derzeit zu kämpfen?  \\\hline
    \  3 & Wie wirkt sich das auf die IT-Systeme in den Versicherungsunternehmen aus, bzw. welche Anforderungen lassen sich daraus an technische Plattformen ableiten? \\\hline
    \  4 & Wofür werden Deiner Einschätzung nach technologische Plattformen bei Kfz-Versicherern in den nächsten 3-5 Jahren vor allem verwendet werden? \\\hline
    \  5 & Welche Anforderungen muss eine technische Plattform für Kfz-Versicherer daher unbedingt erfüllen, bzw. welche Funktionalitäten und Services sollte eine technische Plattform aus Deiner Sicht unbedingt mitbringen, um für die Kfz-Versicherer einen Mehrwert zu schaffen?  \\\hline
    
    \end{tabularx}
    \end{table} 
    \newpage


\begin{table}[H]
\begin{tabularx}{\linewidth}{lX}
\multicolumn{2}{l}{(C) Hauptphase} \\\hline \hline
    \  6 & Eine Herausforderung der Kfz-Versicherer, von der sowohl in der Literatur als auch in der Praxis immer wieder gesprochen wird, sind die sogenannten Legacy-Systeme. Welche Anforderungen ergeben sich dadurch aus Deiner Sicht an technische Plattformen für Kfz-Versicherer? \\\hline
    \  7 & Neben den bereits genannten Punkten habe ich im Rahmen meiner Arbeit Anforderungen der Kfz-Versicherer an technische Plattformen aus der Literatur herausgearbeitet und diese zu Clustern gruppiert. Diese werde ich Dir jetzt kurz vorstellen und Dich bitten, Bezug zu den einzelnen Anforderungen zu nehmen: Kannst Du diese bestätigen und siehst Du in dem jeweiligen Cluster noch weitere Anforderungen, die für Kfz-Versicherer bei technischen Plattformen besonders wichtig sind? \\\hline       
    \  8 & Quantitativer Frageteil: Bitte priorisiere jede Anforderung auf einer Skala von 1-3 nach ihrer Wichtigkeit (1 für niedrig, 2 für mittel, 3 für hoch) für Kfz-Versicherer in den nächsten 3-5 Jahren. \\\hline
    \\
    \multicolumn{2}{l}{(D) Schlussphase}  \\\hline \hline
    \  9 & Wir sind nun am Ende des Interviews angelangt. Gibt es noch etwas, dass Du zum Thema: \enquote{Digitale Plattformen als Wettbewerbsfaktor für den deutschen Kfz-Versicherungsmarkt} hinzufügen möchtest, über das war wir noch nicht gesprochen haben? \\\hline
\end{tabularx}
    %   \captionsetup{justification=centering}
    %   \caption{Fragenkatalog -- Semistrukturiertes Experteninterview}
    %   \label{tab:fragenkatalog}
\end{table} 

\section{Auswahl der Experten}
\label{sec:Expertenwahl}
\improvement{Interview anonymisieren oder so verwerten???}
Bei der Auswahl der Experten wurde darauf geachtet, dass die Experten sowohl Erfahrung in der Kfz-Versicherungsbranche als auch über ein breites Fachwissen zu digitalen Plattformen verfügen. Alle Befragten haben bereits in Versicherungsunternehmen gearbeitet und arbeiten entweder als Berater oder IT-Architekt weiterhin mit deutschen Kfz-Versicherungsunternehmen zusammen.

Interviewpartner 1 ist Christos Lemonidis. Er kann auf mehr als 30 Jahre Erfahrung in der Versicherungsbranche zurückblicken und hat sich im Rahmen seines Studiums auf Versicherungsbetriebs- und -mathematik spezialisiert. Zudem ist er ein zertifizierter Aktuar. Vor seiner Tätigkeit bei SAP hat er in mehreren Versicherungsunternehmen gearbeitet und sich mit Themen wie Risikomanagement, IFRS, Tarifkalkulation und Rückversicherungsoptimierung, auch im Kontext von Kfz-Versicherungen, auseinandergesetzt. Seit über 16 Jahren ist er als Branchenexperte für Versicherungen bei SAP tätig und berät in dieser Funktionen Kunden zu den Kernversicherungslösungen der SAP.  

Interviewpartner 2 ist Alexander Grebert. Er ist der Global Head des Insurance Expert Teams bei SAP Fioneer, einem Joint Venture von Dediq und SAP, welches gegründet wurde, um das SAP-Portfolio für Finanzdienstleistungen gemeinsam auszubauen und in branchenspezifische Lösungen zu investieren. Vor seinem Wechsel zu SAP Fioneer hat Alexander Grebert als ehemaliger Versicherungskaufmann und Betriebswirt für Kfz-, Lebens- und Sachversicherer in Vertriebs-, Underwriting- und internen IT-Bereichen gearbeitet. Später war er als IT-Projektmanager für Erstversicherungslösungen bei der msg verantwortlich und dann in derselben Rolle für den gesamten Finanzdienstleistungssektor bei SAP. Zuletzt war er bei der SAP als \enquote{Director Insurance Architecture and Digital Transformation} tätig, bevor er im September 2021 zur SAP Fioneer wechselte.

Interviewpartner 3 ist Eduard Schmidt. Zu Beginn seiner Karriere arbeitete er in der IT-Abteilung der HDI in Hannover, wo er an der Umstellung des Bestands-, Partner- und Zahlungssystems beteiligt und für die Entwicklung von Kunden- und Vermittlerportalen verantwortlich war. Nach sechs Jahren wechselte er zur SAP und implementierte dort zunächst als Berater die Policymanagement-Lösung bei Versicherungsunternehmen, insbesondere bei Kfz-Versicherern. Heute berät er für die SAP Versicherungsunternehmen bei der Konzeptionierung und Gestaltung der Systemarchitektur.

\section{Transkription der Experteninterviews}

Von allen Teilnehmern wurde eine ausdrückliche Erlaubnis zur Transkription der Gespräche ausgestellt. Die Interviews wurden \enquote{vereinfacht} transkribiert. Das bedeutet, dass Dialekte, grammatikalische Fehler oder entbehrliche Formulierungen verbessert und Wortgenau ins Hochdeutsche übersetzt wurden.\autocite[Vgl.][S. 292]{TAUSENDPFUND2020}

\newpage
\subsection{Experteninterview 1 - Christos Lemonidis (SAP SE)}

\textit{Durchgeführt am 14.04.2023, 09:00 Uhr -- 09:39 Uhr:}


Interviewer/ Experte 1:
Teams-Programm-Test 
\#0:0:58\#

Interviewer:
Ok – dann steigen wir in das Interview ein. Zunächst vorab schon einmal vielen Dank dafür, dass du dir heute die Zeit genommen hast, mit mir das Interview zu führen. Ist es für dich in Ordnung im Rahmen meiner Projektarbeit 2 namentlich benannt zu werden oder möchtest du lieber anonym bleiben?
\#0:1:23\#

Experte 1:
Du kannst mich in deiner Projektarbeit gerne namentlich nennen.
\#0:1:29\#

Interviewer:
Alles klar, so machen wir es. Vielleicht nur so viel noch einmal zur Einleitung: Der Titel meiner Projektarbeit lautet „Digitale Plattformen als Wettbewerbsfaktor für den deutschen Kfz-Versicherungsmarkt am Beispiel der SAP Business Technology Plattform“. Das Ziel dieses Interviews ist es eben die Anforderungen der Kfz-Versicherer an digitale Plattformen zu identifizieren und danach auch zu priorisieren. Mit dem Begriff „digitale Plattformen“ sind hierbei technologische Plattformen gemeint, welche eine Menge von Kernprodukten, -technologien oder Services bereitstellen, auf deren Basis weitere Services entwickelt werden können, wie beispielsweise die SAP BTP - aber eben keine transaktionsorientierten Plattformen, wie zum Beispiel Uber. Nur so viel zur Definition, was mit dem Begriff digitale Plattform gemeint ist.
\#0:2:56\#

Experte 1:
Gut – das habe ich, soweit erst einmal verstanden.
\#0:3:01\#

Interviewer:
Ok - dann können wir also beginnen. Ich werde Dir heute im Verlauf dieses Experteninterviews 8 Fragen stellen, zu Beginn zu Deiner Person, wo Du Dich kurz vorstellen kannst, dann im Hauptteil der Fragenblock um die Anforderungen an „Digitale Plattformen“ zu identifizieren. Danach werde ich selbst einen PowerPoint-Slide auflegen mit der ich Dir die Anforderungen zeige, die ich gefunden habe. Diese dann mit Dir kurz besprechen und danach mit Dir priorisieren. Am Ende des Interviews noch eine Frage zum Ausblick. So, wenn Du keine Fragen vorab hast, würde ich jetzt mit der ersten Frage beginnen.
\#0:4:11\#

Experte 1:
Soweit keine Fragen. 
\#0:4:17\#

Interviewer:
Dann beginnen wir mit Deiner Person - welche Erfahrungen hast Du in der Versicherungsbranche und wie sieht Dein bisheriger Werdegang aus? Was ist Deine aktuelle Position im Unternehmen? Wenn Du das bitte kurz in eins bis zwei Sätzen beschreiben könntest.
\#0:4:40\#

Experte 1:
Ok - also meiner Erfahrung im Versicherungsbereich gehen tatsächlich schon knapp 30 Jahre zurück, ich hatte schon im Studium den Schwerpunkt auf Versicherungsbetriebslehre und Versicherungs¬mathematik. Ich bin seit dem Uni-Abschluss als Versicherungsmathematiker tätig und habe durch Weiterbildungen den Titel „geprüfter Aktuar der deutschen Aktuar-Vereinigung“ erlangt. Während meiner Zeit bei der Versicherung die Themen Risiko Management, IFRS, aber auch Tarifkalkulation und Rückversicherungsoptimierung gemacht. Und in diesem Rahmen habe ich mich auch mit Kfz-Tarifen auseinandergesetzt. Ja – ich bin jetzt seit 16 Jahren bei der SAP und habe auch dort den Schwerpunkt auf unseren Versicherungslösungen, insbesondere den Kern-Versicherungs¬anwendungen, wie Bestandsführungssysteme, Schadenmanagementsysteme oder Inkasso- Systemen. Genau und jetzt? Aktuell liegt mein Schwerpunkt eher Finance-and-Risk, nichtsdestotrotz sind mir natürlich die Versicherungsthemen nach wie vor gut bekannt.
\#0:5:40\#

Interviewer:
Danke - dazu kurz eine Rückfrage von mir. Du hast eben den Begriff „Aktuar-Vereinigung“ angesprochen. Was versteht man darunter, ich habe das vorher noch nicht gehört?
\#0:5:54\#

Experte 1:
Ja, genau - ein Aktuar ist der lateinische Begriff für den Versicherungsmathematiker. Er beschäftigt sich sehr sorgfältig um die Versicherungstarife und um die Versicherungsproduktion - sozusagen eben im modelltheoretischen Sinne - kümmert. Der Abschluss zum Aktuar ist eine Zusatzausbildung, die man nach dem Studium machen kann, um sich in diesem Versicherungsbereich eben noch einmal zu spezialisieren. Also so ähnlich wie bei einem Steuerberater. Diese Prüfung legt man eben bei der deutschen Aktuar Vereinigung - DAV – ab. Um den Titel dauerhaft führen zu dürfen, muss man regelmäßig Weiterbildungsveranstaltungen besuchen. Hierzu nutze ich meine jährlichen SAP-Weiterbildungsstunden, auch wenn ich aktuell im engeren Sinne als Aktuar nicht mehr tätig bin. 
\#0:6:49\#

Interviewer:
Alles klar - danke für deine Erläuterung. Jetzt können wir zu der Hauptphase des Interviews übergehen. Welche Entwicklungen waren in der Kfz-Versicherungsbranche in den letzten Jahren erkennbar und mit welchen Herausforderungen hat die Branche aktuell zu kämpfen?
\#0:7:04\#

Experte 1:
Also einerseits einmal aus der Marktsicht ist natürlich nach wie vor so, dass der Kfz-Markt hart umkämpft ist. Wir kennen wir die Werbeschlachten der einzelnen Kfz-Versicherer, die regelmäßigen Wechselspielchen am Ende jeden Jahres, die dann die „Sparfüchse“ machen. Die Kfz-Versicherung ist als Sicht der Versicherer immer noch im gewissen Sinne ein Einstiegsprodukt, um einen Versicherungskunden – im Allgemeinen -zu gewinnen. Die Kfz-Versicherung wird als „Kuppelprodukt“ angesehen. Das bedeutet, hat man ein Kfz-Versicherungsprodukt, dann kann man damit auch die anderen Versicherungsprodukte erschließen. Also ich denke, so ist die Marktsicht nach wie vor hier. Also es ist ein sehr umkämpfter Markt, wobei jetzt gerade so in den letzten 2 bis 3 Jahren hat die HUK-Coburg, die ja - so als Preisführer agiert - hier schon ihre ihrer Marktstellung ausgebaut und auch ihren Abstand auf die Allianz deutlich ausgebaut. Das kann man - meines Erachtens -schon einmal so als eine Marktbewegung einstufen. Das andere Thema sind natürlich die Telematik-Tarife und dann muss man sehen, welche in anderen Ländern deutlich mehr Fahrt aufgenommen hat, wie in Deutschland. In Deutschland sind diese Telematik-Tarife eher eine Ausnahme. Man kann sie in Deutschland eher als „Versuchsballone“ von den großen Versicherern einordnen. Bisher haben diese Telematik Tarife nicht die riesige Marktdurchdringung. Das kann man auch damit begründen, dass die Tarifwelt in Deutschland schon ziemlich ausdifferenziert ist, so dass man eine relativ große Preisspreizung zwischen risikoarmen und risikobehafteten Fahrern schon hat. Damit sind dann die Spielräume bereits relativ gering. Demzufolge ist ja auch nicht überraschend, dass Telematik Tarife – aus Versicherer-Sicht insbesondere auch für junge Fahrer attraktiv sind – ja - junge männliche Fahrer, so wie du. Die jungen männlichen Fahrer haben aus Sicht des Aktuars natürlich den höchsten Risikobedarf. Andererseits wenn diese Versichertengruppe sorgsam fahren und die das dann auch durch die Telematik-Daten nachweisen können, dass sie vorsichtige und umsichtige Fahrer sind, dann haben sie natürlich ein Hebel, auch was die Rabatte angeht. Bei den anderen Tarifen ist das so, dass da natürlich relativ wenig Spielraum in den Tarifen ist, weil das Tarifniveau in Deutschland schon relativ ausgereizt ist. Ein weiteres Thema, das für die Kfz-Versicherer dann eine Rolle spielt, dass die Autos selbst immer digitaler werden.  Der Verbrennungs¬motor, also die Leistungsklasse – wieviel kW/ PS hat das Auto, spielt ja schon immer die große Rolle, aber die Elektronikkomponenten im Auto bekommen eine immer stärkere Bedeutung, wie das früher der Fall war und das bietet natürlich dann auch Möglichkeiten im Auto - während der Fahrt - Services anzubieten und das ist natürlich auch eine Chance für Kfz-Versicherer da in das digitale Business einzusteigen. Das bedeutet, dass man sich zum Beispiel in einem Porsche elektronische Zusatz-PS dazu buchen kann, wenn man das will. So etwas könnte dann natürlich auch die Fahreigenschaften vom Auto ändern, zum Beispiel auch in Abhängigkeit des Fahrzeugstandortes, als in welchem Land wird gerade das Auto bewegt. Im internationalen Verkehr könnte man sich ja schon vorstellen, dass man dann auch entsprechenden Versicherungsschutz bietet, der sich dann der Fahrsituation oder dem Land, in dem er gerade ist, anpasst.
\#0:11:35\#

Interviewer:
Ok – das heißt zusammengefasst, dass der Markt generell sehr umkämpft in der Branche ist und Telematik-Tarife für Kfz-Versicherer ein relevantes Thema sind. Alle versuchen neue Kunden zu werben und generell wird das Auto immer vernetzter, es gibt immer mehr Daten, die man im Auto irgendwie auswerten bzw. verwenden kann. In diesem Zusammenhang die nächste Frage: Wie wirkt sich das auf die IT-Systeme in den Versicherungsunternehmen aus, bzw. welche Anforderungen lassen sich daraus denn an technische Plattformen ableiten? Sprich, was muss die Plattformen alles können und in den Bereichen unterstützen zu können.
\#0:12:08\#

Experte 1:
Also – wenn man einmal bei den Telematik-Tarifen bleibt - dann ist es so, dass da eine ganze Menge Daten verarbeitet werden müssen, also da hat man das Datentransport-Problem, aber auch das Datenverarbeitungs-Problem. Einerseits ist es natürlich so, dass Telematik-Daten sehr umfangreich sind, das ist einmal die Datenmenge und zum anderen auch das Auswertungsthema. Es nützt ja nichts, dass man die ganzen Daten nur hat, sondern man muss dann ja auch die Daten verarbeiten – also auswerten. Im Sinne von, welches sind die entscheidenden Parameter, aus denen sich dann ableiten lässt, ob ein Fahrer jetzt ein guter Fahrer oder ein risikobehafteter Fahrer ist, ob er weniger oder viel fährt etc. Also da muss man natürlich dann auch das Auswertungs-Know-how aufbauen und natürlich auch die Auswertungsmöglichkeiten dann zur Verfügung stellen. Da ist sicherlich noch einiges, was man natürlich noch nachgelagert machen kann. Aber wenn man sich vorstellt, wie dann auch ein regelmäßiger Datenfluss aus dem Auto ist und man will dann entsprechende Services mit anbieten, dann geht es ja auch darum, diese Daten in Real Time zu verarbeiten. Das heißt, dass man hier sehr zeitnah etwas dann vielleicht anbieten kann. Also ich würde ich sagen, das sind so in etwa aktuell die großen Herausforderungen aus Sicht der Kfz-Versicherer - Datenmenge. Auswertung und Echtzeitverarbeitung.
\#0:13:33\#

Interviewer:
Ok - das habe ich verstanden. Da vielleicht noch die Frage in Bezug auf die technischen Plattformen: Wofür werden, deiner Einschätzung nach, technologische Plattformen - wie die SAP BTP - bei Kfz-Versicherern in den nächsten 3-5 Jahren vor allem verwendet werden? Was erscheint denn deiner Meinung nach da am realistischsten?
\#0:14:00\#

Experte 1:
Ich glaube, dass der Bereich Telematik aus Sicht der SAP jetzt gar nicht so das Entscheidende sein wird, weil das dann schon auch so ein Spezialthema ist, in dem wir jetzt ja gar nicht so engagiert sind. Aber der ganze Themenblock „zusätzliche Services“, der spielt natürlich eine Rolle, weil man zusätzliche Services anbietet, dann muss man die ja auch abrechnen können. Und dann ist man eher in einem Customer-Experience-Umfeld oder auch einem Shop-Environment. Das ist natürlich schon etwas, was für die SAP sehr interessant sein kann. Also: Wie bringe ich die Angebote, die Services jetzt nicht aus der technischen Sicht, aber wie bekomme ich die Services auch kommerziell im Auto unter? Wie kann ich die Services über die im Auto vorhandenen Kommunikationssysteme anbieten? Und wenn die dann in Anspruch genommen wird, muss es ja auch abrechnen können. Noch einmal zu dem Beispiel „zusätzliche Porsche-PS buchen“ - “ wenn die Bereitstellung von Turbo-PS für 2 Stunden 3,70 \euro kostet, dann muss das auch irgendwo eingebucht, abgerechnet und dann auch kassiert werden. Das würde ich sagen, ist dann schon ein Geschäftsfeld für die SAP.
\#0:15:33\#

Interviewer:
Ok – du hast gerade die Punkte Funktionalität und Services angesprochen. Frage: Welche Anforderungen muss - deiner Meinung nach - eine technische Plattform für Kfz-Versicherer daher unbedingt erfüllen bzw. welche Funktionalitäten und Services sollte eine technische Plattform aus deiner Sicht unbedingt mitbringen, um für die Kfz-Versicherer einen Mehrwert zu schaffen? 
\#0:15:53\#

Experte 1:
Ja, ich glaube schon das ist das Thema Echtzeitverarbeitung - das Services-Thema muss möglichst schnell gehen und zwar in dem Moment, wo es notwendig ist. Das gleiche gilt natürlich auch für die ganzen erforderlichen betriebswirtschaftlichen Prozesse, die da dahinterstehen. Die müssen halt auch in Echtzeit dann Verfügung stehen. Also das sind jetzt keine Prozesse, wo man sich eine Batch-Verarbeitung vorstellen oder wo dann im Anschluss - irgendwie einen Tag oder Woche später - nochmal dann Papierkram erzeugt wird. Sondern das muss dann ein hochdigitaler Prozess sein, der in Echtzeit dann eben auch die Prozesse darstellt, weil das wahrscheinlich auch das ist, was dann der Konsument im Fahrzeug - in seinem „digitalen Raumschiff“ - dann auch erwartet. 
\#0:16:45\#

Interviewer:
Genau, genau - eine Herausforderung der Kfz-Versicherer, von der sowohl in der Literatur als auch in der Praxis immer wieder gesprochen wird, sind die sogenannten Legacy-Systeme. Welche Anforderungen ergeben sich dadurch aus deiner Sicht an technische Plattformen? Gibt es da Besonderheiten in Bezug auf die Alt-Systeme, die zu berücksichtigen sind ?
\#0:17:09\#

Experte 1:
Ja gut - in Bezug auf die Alt-Systeme ist doch so, dass diese oftmals so eine Echtzeitverarbeitung gar nicht darstellen können, weil die Alt-Systeme batchorientiert sind. Und dann ist natürlich die Aufgabe von einer technischen Plattform, im Gewissen Sinne auch zu entkoppeln. Ja, das sie dem Kunden gegenüber Echtzeitverarbeitung ermöglicht und aber trotzdem dann Zugriff auf die notwendigen Daten, insbesondere Stammdaten, dann auch so ein Bestandsführungssystem hat. Ja, man benötigt dann schon so eine „Entkopplungsschicht“, die natürlich dann aber auch integriert sein muss.
\#0:17:52\#

Interviewer:
Ok – nachgefragt: „Entkoppeln“ heißt in dem Punkt dann zu sagen, dass man die Alt-Systeme irgendwie versucht, zu Kapseln und dann als einen geschlossenen Baustein an die Plattform anzubinden oder wie kann ich mir die Entkopplung, die du beschrieben hast, vorstellen?
\#0:18:07\#

Experte 1:
Hm - da gehe eher einmal davon aus, dass die Alt-Systeme nur in einem beschränkten Umfang API-fähig sind. Das man sich dann Modelle überlegen muss, wie schafft man es die statischen Informationen aus dem Bestandsführungssystem einzubinden bzw. zu verwenden, auch für die Services und die Services eben dann kundenorientiert auszuprägen und auf der anderen Seite die Agilität, die man benötigt, um an der Kundenschnittstelle schnell und qualitativ hochwertige Services zur Verfügung stellen zu können.
\#0:19:02\#

Interviewer:
Du hast gerade eben schon das Schlagwort Schnittstellen angesprochen, gibt es hier typische Schnittstellen, die bei Alt-Systemen häufig vorzufinden sind. Aktuelle ist ja häufig State of the Art, die sogenannten REST-Schnittstellen. Früher war glaube ich auch viel SOAP im Einsatz. Hast du dazu Erfahrungen gesammelt?
\#0:19:22\#

Experte 1:
Ja, es ist ja schon so, dass die Versicherungen auch heutzutage da so einiges machen müssen. Aber das sind dann oft auch so Kupplungsarchitekturen, die bereits in einem Versicherungsunternehmen jetzt schon vorhanden sind, um Kundenschnittstellensysteme mit den Backend-Systemen verbinden zu können. Ob das dann direkt immer möglich ist, es wird wahrscheinlich in dem ein oder anderen Einzelfall gehen, aber in der Regel sind da schon eben genau solche Kupplungsarchitekturen not¬wendig, um Agilität auf der einen und statische Informationen auf der anderen Seite zu verknüpfen.
\#0:20:02\#

Interviewer:
Ok - dann würde ich jetzt einmal zu meiner PowerPoint kommen: Und zwar habe ich im Rahmen meiner Arbeit aus der Literatur Anforderungen der Kfz-Versicherer an technische Plattformen herausgearbeitet, von denen einige bereits genannt wurden, und diese zur besseren Strukturierung in verschiedene Cluster eingeteilt. Diese werde ich dir jetzt kurz vorstellen und dich bitten Bezug zu den einzelnen Anforderungen zu nehmen: Sprich kannst du diese bestätigen und sieht du in dem jeweiligen Cluster noch weitere Anforderungen die für Kfz-Versicherer bei technischen Plattformen besonders wichtig sind? Einen kleinen Moment - so du müsstest jetzt meinen Bildschirm sehen.
\#0:20:57\#

Experte 1:
Ja, mit der ersten Spalte Integration.
\#0:21:00\#

Interviewer:
Ganz genau - so bei den Anforderungspunkten, die ich hier im Cluster „Integration“ gefunden habe und du hattest hiervon eben schon 1 bis2 genannt gehabt, sind zu nennen „Unterstützen einer service-orientierten Architektur“, um eben auch – wie du meintest - die Alt-Systeme koppeln zu können bzw. kapseln zu können.  Um diese dann als Services für neuere Applikationen entsprechend zur Verfügung zu stellen. Die Alt-Systeme müssen eben an die neuen Plattformen angebunden werden können, das ist eine ganz wichtige Grundvoraussetzung für die Plattformen. Diese müssen mit offenen Schnittstellen-Standards letztlich kompatibel sein. Ich weiß nicht, ob dir der Begriff „digitale Ökosysteme“ etwas sagt, was ja aktuell auch so ein Trendthema in der Versicherungs¬branche ist. Da ist es ja so, dass gerade mit Kunden, Partnern oder auch Drittanbietern von Applikationen der Datenaustausch sehr wichtig ist und dazu wird eben aktuell häufig auch die REST-Schnittstelle verwendet, die man hier als State-Of-The-Art bezeichnen kann. Und eben als vierten Anforderungspunkt dieses Clusterbereichs das „API Management Tool“, sprich es gibt immer mehr Schnittstellen, die müssen eben auch kontrolliert und verwaltet werden. Sprich - wer greift auf die ganzen API überhaupt zu? Dazu an dich die Frage: Kannst du die einzelnen Anforderungen bestätigen – siehst du vielleicht im Bereich „Integration“ darüber hinaus noch weitere wichtige Punkte?
\#0:22:14\#

Experte 1:
Ja, ich kann die von dir genannten Punkte so bestätigen, ich würde aber vielleicht noch den Punkt „Kopplungsarchitektur“ noch mit aufnehmen.
\#0:22:22\#

Interviewer:
Ok – dann notiere ich mir das direkt einmal - so. Das nächste Cluster wäre dann der Bereich „Entwicklung“. Da ist es, ich glaube, zum einen ganz wichtig, dass man auf diesen technischen Plattformen eben neue Anwendungen entwickeln und auch betreiben kann. Das gilt auch für mobile Anwendungen - das ist ja auch ein Thema, das aktuell immer wichtiger wird. Im Idealfall sollten auch bestehende Anforderungen erweitert werden können, um eben die einzelnen Bedarfe der Versicherer anpassen zu können. Es sind aktuell ja – gerade auch im Versicherungsbereich – die Entwicklungsressourcen sehr knapp und wenn da die technische Plattform auch gleichzeitig Low-Code- und No-Code-Werkzeuge zur Verfügung stellen kann, dann würde das sicher auch helfen, dass vielleicht die Leute aus den Fachabteilungen auch selbst einmal mit einfachen Drag-and-Drop-Systemen Erweiterungen bauen können. Als weitere Anforderung ist - auch im Bereich der Telematik-Tarife - der Punkt „Entwicklung von IoT-Anwendungen“ zu nennen. Den hattest du ja auch schon vorhin kurz angesprochen. Da wäre auch hier die Frage an dich: Hast du zu diesen Anforderungspunkten bereits Erfahrungen gesammelt? Oder gibt es vielleicht noch weitere Punkte im Bereich der Entwicklung von Applikationen für diese technische Plattform, die ich jetzt noch nicht genannt habe?
\#0:23:38\#

Experte 1:
Nein, das passt ganz gut. Den ersten Punkt würde ich vielleicht noch einmal deutlicher herausstellen. Der ist – meines Erachtens - im Vergleich zu den nachfolgenden Punkten eine ganz wichtige Anforderung. Die anderen 3 Punkte sind - sage ich einmal so – eher nach innen gerichtet sind. Aber, dass man hier mobile Anwendungen hat, das ist ja eigentlich - gerade jetzt bezogen auf Kfz - im doppelten Sinne wichtig, auch bezüglich der Kleinteiligkeit der Anwendungen. Dann vielleicht noch ein weiterer wichtiger Aspekt, dass die technische Plattform in andere oder größere Anwendungen eingebettet ist. Es ist ja nicht so, dass die Versicherung dann das führende System ist, daher ist es wichtig bestimmte Funktionalitäten dann direkt in einem Navigationssystem oder auch in ein Fahrzeug-System anbieten zu können. Daher vielleicht noch das „Einbetten in andere Anwendungen“ als zusätzlichen Anforderungspunkt. Und bei Punkt 1 ohne die Klammer um den Begriff „mobile“, weil das - meiner Meinung nach - die Kernanforderung hier ist. 
\#0:24:47\#

Interviewer:
Frage dazu: Das heißt, du würdest dann den Punkt „mobile Anwendungen“ auch als einzelnen Punkt gezielt hervorheben? 
\#0:25:09\#

Experte 1:
Ja, das finde ich eigentlich schon wichtig und wie schon gesagt, dass diese Anwendungen eingebettet sind - in andere Systeme, weil es nicht das führende System darstellen wird.
\#0:25:24\#

Interviewer:
Ok, das habe ich verstanden. Das nächste Cluster wäre der Bereich „Datenverarbeitung“. Da glaube ich, ist der Punkt „spartenübergreifende Sicht auf Daten“ ein ganz wichtiger Punkt für die Kfz-Versicherer, um eben ein besseres Kundenverständnis zu bekommen und auch den Kunden gezielte Angebote machen zu können, wie eben im Bereich von Telematik-Tarifen. Auch das „Empfangen von Sensordaten und von physischen Objekten“ ist als Anforderungspunkt für diese technischen Plattformen- wie beispielhaft bei Autos - glaub ich auch sehr wichtig. Aufgrund der unstrukturierten Datenbasis müssen Daten mit einer Extract-Transform-Load-Lösung aufbereitet werden. Oder, was du auch schon meintest, vor allem im Bereich der vernetzten Autos eben die Analyse der großen Datenmengen. Auch für dieses Cluster wieder die Frage: Siehst du weitere Anforderungen oder passt die Aufzählung soweit?
\#0:26:05\#

Experte 1:
Das passt meiner Meinung nach so, die Anforderungen sind vollständig.
\#0:26:08\#

Interviewer:
Ok, sehr gut. Mit Blick auf Zeit – wir haben noch maximal eine Viertelstunde - würde ich jetzt direkt weiter gehen. Das vierte Cluster wäre der Bereich „Prozessautomatisierung“, hier eben zum einen durch RPA. Im Rahmen der Literaturrecherche hat sich gezeigt, dass bei Versicherungen auch viele repetitive Prozesse bestehen, die trotzdem noch manuell von Mitarbeitern ausgeführt werden und eben noch nicht automatisiert sind. Sprich – da könnte man RPA ganz gut verwenden, unter anderem auch die Automatisierung von Kundeninteraktionen, gerade bei Kundenanfragen, wie zum Beispiel Versicherungspolicen abrechnen oder Schadensmeldungen durchführen, da könnte eben der Chatbot im Vergleich zum normalen Sachbearbeiter zu jederzeit antworten - und man könnte eben auch Wartezeiten vermeiden. Da wäre wiederum die Frage: Stimmst du den Anforderungspunkten im Bereich „Prozessautomatisierung“ zu, siehst du vielleicht noch weitere, die nicht genannt wurden?
\#0:27:01\#

Experte 1:
Hm, da würde ich - auch wieder unter dem Gesichtspunkt deiner eigentlichen Fragestellung – den Anforderungspunkt 14 als deutlich relevanter ansehen, wie den Punkt 13, weil – ich sage einmal so –„Prozessautomatisierung durch RPA“ ist ja eigentlich der Blick nach hinten. Das ist ja eigentlich nur dafür da bestehende Mängel der existierenden Software zu beheben. Die spielen zwar da sicherlich eine Rolle, aber im Hinblick auf digitale Prozesse ist es ja auf jeden Fall das Ziel, dass solche Nachbearbeitungen durch Roboter überhaupt erst gar nicht anfallen, sondern dass die direkt bearbeitet werden. Also das das würde ich dann noch unterschiedlich priorisieren.
\#0:27:58\#

Interviewer:
Ok, das können wir dann gleich noch machen. In dem letzten Cluster „Weitere Anforderungen“ kommen – meiner Meinung nach – noch 2 weitere wichtige Punkte. Zum einen der Punkt „grundlegende Datensicherheit“, da können die Versicherungsunternehmen – glaube ich - schlecht die Sicherheit von Systemen komplett überprüfen. Daher greifen sie auf Zertifizierungen zurück, welche zum einen im Bereich Informationssicherheit-Management-Systeme, als auch im Bereich DSGVO benannt wurden, das sind hier die beiden ISO-Normen und eben der Punkt „Vertikale und horizontale Skalierbarkeit“ -gerade um auch mit den hohen Anfragen bei Kfz-Versicherern, beispielsweise wie sie im Herbst aufgrund der vielen Tarif-Wechsel-Kündigungen stattfinden, klar zu kommen und entsprechend weitere Ressourcen im Rahmen der Skalierbarkeit zur Verfügung  zu stellen. Stimmst du meinen Anforderungen da auch zu? 
\#0:28:44\#

Experte 1:
Ja, das passt sehr gut. Also zusammenfassend – als allgemeine Anmerkung - würde ich jetzt noch einmal versuchen die einzelnen Anforderungspunkte nicht gleichberechtigt nebeneinander zu stellen, sondern entsprechend zu priorisieren. Das heißt, noch einmal die Anforderungspunkte in Vordergrund zu stellen, die dann besonders mit digitalen Versicherungsprozessen zu tun haben. 
\#0:29:21\#

Interviewer:
Warte – den Punkt meintest du gerade, würdest du bei der „Integration“ sehen – ok. Da war noch der Punkt „Einbindung in andere Systeme“ - den hattest du dem Bereich „Entwicklung“ zugeordnet– ist das richtig so?
\#0:29:43\#

Experte 1:
Hm - ich würde den Anforderungspunkt „Einbettung in andere Systeme“ nennen.
\#0:29:51\#

Interviewer:
Genau -optional würdest du hier noch die „mobilen Anwendungen“ extra – mit hoher Priorität aufführen?
\#0:30:13\#

Experte 1:
Ja, also mir ist wichtig festzuhalten, dass die aufgeführten Anforderungspunkte in jedem Fall nicht alle gleich wichtig sind für Kfz-Versicherer. 
\#0:30:23\#

Interviewer:
So dann würde ich jetzt noch einmal zum Ende unseres Interviews kurz mit dir die Punkte priorisieren wollen. Und zwar wäre es jetzt hier die Idee, wir haben jetzt hier die einzelnen Anforderungen, die eine Plattform erfüllen sollte und dann die Bitte an dich die Wichtigkeit der einzelnen Punkte von 1 bis 3 zu priorisieren und wo die eben den Kfz-Versicherern vor allen Dingen den nächsten 3 bis 5 Jahren helfen können und vielleicht bei jedem kurz mit 1-2 Sätzen kurz beschreiben, warum du den Punkt der jeweiligen Priorität zuordnest -von hoch bis niedrig.
\#0:31:01\#

Experte 1:
Okay, also, dann fangen wir mal mit Anforderung Nr. 3 an, das würde ich alles hoch einstufen. Dann das API Management Tool als mittel einstufen, die anderen 2 Punkte als niedrig.
\#0:31:28\#

Interviewer:
Die Punkte „Legacy-Systeme“ und „Serviceorientierte Architektur“ als niedrig einzustufen, weil das in der Regel schon vorhanden schon vorhanden ist oder weil es einfach generell nicht so wichtig ist, was wäre da die Erklärung?
\#0:31:37\#

Experte 1:
Das ist – meiner Meinung nach - ja ein grundsätzliches Problem, dass die Versicherer an der Stelle haben. Das würde ich jetzt nicht unbedingt aus Sicht von KFZ an eine technische Plattform in Richtung Kundenschnittstelle als wichtigste Priorität sehen – Ok?.
\#0:31:52\#

Interviewer:
Heißt das, das sind Punkte, die deiner Auffassung nach sowieso gemacht werden müssen oder siehst du das so ein bisschen nach dem Motto. „Das wird sowieso funktionieren“ Kann ich mir das etwa so vorstellen?
\#0:32:08\#

Experte 1:
Nein – ich sage einmal, das sind grundlegende Themen, mit denen sich die Versicherer beschäftigen müssen, unabhängig davon, was sie sich im Kfz-Versicherungsbereich für zusätzliche Services vorstellen.
\#0:32:21\#

Interviewer:
Ok – das heißt, du würdest: Es ist ein allgemein wichtiger Bereich, aber in Bezug auf die Plattform-Anforderungen nicht so wichtig, weil es dabei andere gibt, die in jedem Fall wichtiger sind, wie zum Beispiel die offenen Schnittstellen.
\#0:32:33\#

Experte 1:
Ja genau – so ist es.
\#0:32:35\#

Interviewer:
Ok - dann würde ich sagen, wir machen mit dem Bereich „Entwicklung“ in der Priorisierung weiter.
\#0:32:38\#

Experte 1:
Ja, dann würde ich die „mobilen Anwendungen“ nach oben - als hohe Priorität.
\#0:32:41\#

Interviewer:
Genau und das hatten wir bereits ja auch schon kurz besprochen.
\#0:32:45\#

Experte 1:
Ebenso der Punkt „Einbettung in andere Systeme“, insbesondere die mobilen Anwendungen.
\#0:32:49\#

Interviewer:
Genau. Du würdest diesen Punkt auch ganz oben, sehen?
\#0:32:53\#

Experte 1:
Ja, genau dann das den ersten Punkt „Entwicklung und Betrieb neuer Anwendungen“ dann auch bei Mittel. Der Punkt IoT-Anwendungen und die anderen 2 wieder in die Priorität niedrig.
\#0:32:58\#

Interviewer:
OK - also würdest du die Verwendung von Low-Code-/ No-Code-Werkzeugen nicht als so wichtig ansehen, weil der Bedarf nicht da ist oder weil diese noch gar nicht so weit sind, was ist da deine Begründung?
\#0:33:18\#

Experte 1:
Wie bereits gesagt, die Bewertung sollte immer aus Sicht der grundlegenden Fragestellung erfolgen. Also, was sind so die Anforderungen. Ich möchte ja vorne an der Kundenschnittstelle etwas Neues machen und wie ich das dann letztendlich umsetze, das sehe ich eher als nachrangig an. Das Entscheidende ist es ja nicht, ob er das in einem Low-Code oder No-Code umsetzt, sondern dass die Anwendung funktioniert. Das Werkzeug ist dann an der Stelle nicht entscheidend.
\#0:34:07\#

Interviewer:
Ok - und dann noch genau der Punkt „Erweiterungen bestehender Anwendungen“, den du ganz unten einsortierst. Würdest du sagen, dass die Anwendungen, die eine technische Plattform bereitstellen soll, gar nicht groß erweitert werden möchten, da der Bedarf nicht so groß ist oder weshalb ist hier die Einordnung ganz unten?
\#0:34:21\#

Experte 1:
Also erst einmal muss ja auch irgendetwas unten stehen, wenn man priorisieren will. Ich glaube nicht, dass die funktionale Erweiterung der Back-End-Systeme hier im Vordergrund steht.
\#0:34:35\#

Interviewer:
Alles klar, okay- Das habe ich verstanden. Das sind die Punkte „IoT-Anwendungen“ und „Einbettung in andere Systeme“ wichtiger. Könntest du für mich bitte noch einmal mal - in 3 bis 4 Sätzen - den Anforderungspunkt Einbettung in andere Systeme“ kurz erläutern, also was du dir darunter vorstellst?
\#0:34:48\#

Experte 1:
Also wie gesagt, das was an der Kundenschnittstelle im Fahrzeug passieren wird, da wird ja der Versicherer das führende System hinstellen. Sondern es geht darum, also wenn man das einmal aus einer Gesamtplattform-Diskussion sieht, die hast du ja ausgegrenzt, ist es immer die Frage: Wer ist der Plattformbetreiber? Ja, und da wird die Versicherung niemals der Plattformbetreiber hinsichtlich der Kfz-Services sein, sondern es geht darum, bestimmte „Versicherungs-Schnipsel“ in einer größeren Anwendung zur Verfügung zu stellen. Man muss in der Lage sein, sich auch in andere Systeme einzubetten zu können, weil man als Versicherer selbst nie das führende System sein wird.
Ich weiß nicht, ob der Kunde dann eine Versicherungs-App in seinem Navigationssystem erwartet oder ob er einfach im Rahmen von einem Service, den er da sprichwörtlich „unter der Motorhaube“ konsumieren will, die Versicherung mit dazu bekommt. Dem entsprechend muss dann der „Software-Schnipsel“ der Versicherung darauf ausgerichtet sein muss, in einem anderen System im Hintergrund wirken zu können.
\#0:36:07\#

Interviewer:
Ok – sprich: OEM macht eine Plattform und Versicherung ist mehr Partner als eigentlicher Plattform-Betreiber. Dann würde ich sagen - mit Blick auf die Zeit - kommen wir noch zum Bereich „Datenaufbereitung“. Auch hier wieder 4 Anforderungspunkte - wie würdest du diese einsortieren? „Spartenübergreifende Sicht auf Daten“ bzw. „Empfangen von Sensordaten“ - genau?
\#0:36:29\#

Experte 1:
Hm – ich würde sagen „Analyse großen Datenmengen“ nach oben, die Sensordaten als zweit wichtig und die anderen eher nachrangig, weil es wieder allgemein ist.
\#0:36:42\#

Interviewer:
Ok – dazu noch einmal die Nachfrage im Bereich „Datenaufbereitung“ - gerade in der Analyse müssen die Daten hier auch zunächst aus der unstrukturierten Form erstmal aufbereitet werden, um sie dann Analysieren zu können. Aber für dich eben trotzdem die Priorität Datenanalyse und nicht die Datenaufbereitung da?
\#0:36:58\#

Experte 1:
Ja, das sind ja allgemeine Anforderungen. Beim nächsten Punkt – das habe ich ja vorhin schon gesagt - da wird man sagen, den Punkt „Kundeninteraktion“ würde ich hoch einstufen und den Punkt „RPA-Prozessautomatisierung“ würde ich als niedrig einstufen.
\#0:37:14\#

Interviewer:
Ok - und die letzten beiden Punkte „Datensicherheit“ und „Skalierbarkeit“ – wo sind die deiner Meinung nach einzuordnen?
\#0:37:25\#

Experte 1:
Ja, da würde ich „Datensicherheit“ von der Priorität sogar hoch einstufen, wo wir uns jetzt dann auch gegenüber anderen uns differenzieren, den Punkt „Skalierbarkeit“ als mittlere Anforderung.
\#0:37:38\#

Interviewer:
Alles klar, wunderbar - dann haben wir jetzt die Priorisierung abgeschlossen. Bitte noch einmal kurz darauf schauen – ob die Anforderungen deiner Einschätzung so richtig priorisiert sind.
\#0:37:59\#

Experte 1:
Nein - ich denke die Priorisierung, die ist ja jetzt schon ausdifferenziert – das passt.
\#0:38:07\#

Interviewer:
Wunderbar - dann sind wir jetzt am Ende des Interviews angelangt. Die abschließende Frage wäre jetzt: Gibt es noch etwas zum Thema „digitale Plattformen als Wettbewerbsfaktor für den deutschen Kfz-Versicherungsmarkt“, was du noch hinzufügen möchtest, über das wir bisher noch nicht gesprochen haben?
\#0:38:24\#

Experte 1:
Nein – es wurde alles gesagt.
\#0:38:26\#

Interviewer:
Ok -  sprich da auch da alle Punkte genannt. Vielen Dank, dass du dir die Zeit genommen hast, mit mir ein Experteninterview zu führen und meinen Fragen Rede und Antwort zu stehen. Dann würde ich jetzt hier an dieser Stelle das Recording beenden – wie gesagt noch einmal herzlichen Dank!
\#0:38:42\#


\newpage

\subsection{Experteninterview 2 - Alexander Grebert (SAP Fioneer)}

\textit{Durchgeführt am 14.04.2023, 10:30 Uhr -- 11:09 Uhr:}
\improvement{Zeit Überprüfen}

Text kommt noch 

\newpage
\subsection{Experteninterview 3 - Eduard Schmidt (SAP SE)}

\textit{Durchgeführt am 20.04.2023, 11:02 Uhr -- 11:37 Uhr:}
\improvement{Zeit Überprüfen}

Text kommt noch

\newpage

% \caption[]{}{AXA SE - Konzernstruktur (AXA Deutschland)}
%wie bekomme ich es hin dass die nicht im Abbildungsverzeichnis auftaucht?

%\subsection{Interview Leitfaden}