% !TEX root =  master.tex
\chapter{Anhang}\label{anhang}

%\section{Konzernstruktur großer Versicherer}
%\label{sec:KonzernStrukturen}

%\begin{figure}[h]
  %\centering
  %\includegraphics[width=1\textwidth]{img/beteiligungsstruktur-axa-konzern.pdf}
  %\caption[]{AXA SE - Konzernstruktur (AXA Deutschland)}
  %\label{fig:AXAKstr}
%\end{figure}
%\newpage

\section{Suchbegriffe der Literaturanalyse}
\begin{table}[ht]
    \begin{center}
    \begin{tabular}{|p{3cm}|p{3cm}|p{3,5cm}|p{3,5cm}|}
        \hline Sprache & Suchbegriff 1 & Suchbegriff 2 & Suchbegriff 3\\[0.5ex]
        \hline Deutsch & Kfz-Versicherer, Versicherer & Anforderungen & Digitale Plattformen, technische Plattformen, PaaS, APaaS, IT-Plattformen\\
        \hline Deutsch & Kfz-Versicherer, Versicherer & Herausforderungen, Probleme & -\\
        \hline Englisch & car insurance, motor insurance, insurance & requirements & digital platforms, technological platforms, PaaS, APaaS, IT-platforms\\
        \hline Englisch & car insurance, motor insurance, insurance & challenges, problems, issues & -\\
        \hline
    \end{tabular}
    \end{center}
    \caption{Suchbegriffe der Literaturanalyse}
    \label{tab:suchbegriffe}
\end{table}  

\newpage
\section{Kriterien der Literaturanalyse}
\begin{table}[ht]
    \begin{center}
    \begin{tabular}{|p{4,33cm}|p{4,33cm}|p{4,33cm}|}
        \hline  & Einschlusskriterien & Ausschlusskriterien \\[0.5ex]
        \hline Sprache & Deutsch, Englisch & Weitere Sprachen\\
        \hline Dokumententypen & Fachliteratur, internationale und nationale wissenschaftliche Zeitschriften, Studien, Magazine, Internetartikel & Weitere Veröffentlichungen\\
        \hline Publikationszeitraum & Veröffentlichungen von 2023 - einschließlich 2013 & Veröffentlichung vor 2013\\
        \hline Verfügbarkeit & Volltext verfügbar & Volltext nicht verfügbar\\
        \hline
    \end{tabular}
    \end{center}
    \caption{Kriterien der Literaturanalyse}
    \label{tab:kriterien}
\end{table}   
\newpage

\section{Zertifizierungen der SAP BTP Services}
\label{sec:BTPZertifikate}
\begin{figure}[h]
    \centering
    \includegraphics[width=1\textwidth]{img/BTP_Zertifikate.pdf}
    %\caption[]{Zertifizierungen der SAP BTP Services (SAP SE 2023)}
    \label{fig:BTPZertifikate}
  \end{figure}
\newpage

\section{Fragenkatalog der Leitfadeninterviews}
\label{sec:Fragenkatalog}
Im laufenden Gespräch wurde die Anzahl gestellter Fragen und deren Formulierung variiert. Aufnahme und Transkription fanden über die die MS-Teams-Funktion statt.

\begin{table}[H]
\begin{tabularx}{\linewidth}{lX}
\multicolumn{2}{l}{(A) Informationsphase} \\\hline \hline 
    \    & Ich führe das Interview im Rahmen meiner Projektarbeit 2 durch, welche den Titel: \enquote{Digitale Plattformen als Wettbewerbsfaktor für den deutschen Kfz-Versicherungsmarkt am Beispiel der SAP Business Technology Platform} trägt. Das Ziel dieses Interviews ist es, Anforderungen der Kfz-Versicherer an digitale Plattformen zu identifizieren und zu priorisieren. \\\hline
    \    & Dabei sind im Rahmen meiner Arbeit mit dem Begriff digitale Plattform, technologische Plattformen gemeint, welche eine Menge von Kernprodukten, -technologien oder -services bereitstellen, auf deren Basis weitere komplementäre Produkte und Services entwickelt werden können. Somit bezeichnet der Begriff digitale Plattform im Rahmen meiner Arbeit technische Plattformen wie die SAP BTP nicht aber transaktionsorientierte Plattformen wie bspw. Uber.\\\hline
    \\     
\multicolumn{2}{l}{(B) Einstiegsphase}  \\\hline \hline
    \  1 & Welche Erfahrungen hast Du in der Versicherungsbranche? Wie sieht Dein Werdegang aus und was ist Deine aktuelle Position im Unternehmen? \\\hline     
    \\  
\multicolumn{2}{l}{(C) Hauptphase}  \\\hline \hline
    \  2 & Welche Entwicklungen waren in der Kfz-Versicherungsbranche in den letzten Jahren erkennbar und mit welchen Herausforderungen hat die Branche derzeit zu kämpfen?  \\\hline
    \  3 & Wie wirkt sich das auf die IT-Systeme in den Versicherungsunternehmen aus, bzw. welche Anforderungen lassen sich daraus an technische Plattformen ableiten? \\\hline
    \  4 & Wofür werden Deiner Einschätzung nach technologische Plattformen bei Kfz-Versicherern in den nächsten 3-5 Jahren vor allem verwendet werden? \\\hline
    \  5 & Welche Anforderungen muss eine technische Plattform für Kfz-Versicherer daher unbedingt erfüllen, bzw. welche Funktionalitäten und Services sollte eine technische Plattform aus Deiner Sicht unbedingt mitbringen, um für die Kfz-Versicherer einen Mehrwert zu schaffen?  \\\hline
    
    \end{tabularx}
    \end{table} 
    \newpage


\begin{table}[H]
\begin{tabularx}{\linewidth}{lX}
\multicolumn{2}{l}{(C) Hauptphase} \\\hline \hline
    \  6 & Eine Herausforderung der Kfz-Versicherer, von der sowohl in der Literatur als auch in der Praxis immer wieder gesprochen wird, sind die sogenannten Legacy-Systeme. Welche Anforderungen ergeben sich dadurch aus Deiner Sicht an technische Plattformen für Kfz-Versicherer? \\\hline
    \  7 & Neben den bereits genannten Punkten habe ich im Rahmen meiner Arbeit Anforderungen der Kfz-Versicherer an technische Plattformen aus der Literatur herausgearbeitet und diese zu Clustern gruppiert. Diese werde ich Dir jetzt kurz vorstellen und Dich bitten, Bezug zu den einzelnen Anforderungen zu nehmen: Kannst Du diese bestätigen und siehst Du in dem jeweiligen Cluster noch weitere Anforderungen, die für Kfz-Versicherer bei technischen Plattformen besonders wichtig sind? \\\hline       
    \  8 & Quantitativer Frageteil: Bitte priorisiere jede Anforderung auf einer Skala von 1-3 nach ihrer Wichtigkeit (1 für niedrig, 2 für mittel, 3 für hoch) für Kfz-Versicherer in den nächsten 3-5 Jahren. \\\hline
    \\
    \multicolumn{2}{l}{(D) Schlussphase}  \\\hline \hline
    \  9 & Wir sind nun am Ende des Interviews angelangt. Gibt es noch etwas, dass Du zum Thema: \enquote{Digitale Plattformen als Wettbewerbsfaktor für den deutschen Kfz-Versicherungsmarkt} hinzufügen möchtest, über das war wir noch nicht gesprochen haben? \\\hline
\end{tabularx}
    %   \captionsetup{justification=centering}
    %   \caption{Fragenkatalog -- Semistrukturiertes Experteninterview}
    %   \label{tab:fragenkatalog}
\end{table} 

\section{Auswahl der Experten}
\label{sec:Expertenwahl}
Bei der Auswahl der Experten wurde darauf geachtet, dass die Experten sowohl Erfahrung in der Kfz-Versicherungsbranche als auch über ein breites Fachwissen zu digitalen Plattformen verfügen. Alle Befragten haben bereits in Versicherungsunternehmen gearbeitet und arbeiten entweder als Berater oder IT-Architekt weiterhin mit deutschen Kfz-Versicherungsunternehmen zusammen.

Interviewpartner 1 ist Christos Lemonidis. Er kann auf mehr als 30 Jahre Erfahrung in der Versicherungsbranche zurückblicken und hat sich im Rahmen seines Studiums auf Versicherungsbetriebs- und -mathematik spezialisiert. Zudem ist er ein zertifizierter Aktuar. Vor seiner Tätigkeit bei SAP hat er in mehreren Versicherungsunternehmen gearbeitet und sich mit Themen wie Risikomanagement, IFRS, Tarifkalkulation und Rückversicherungsoptimierung, auch im Kontext von Kfz-Versicherungen, auseinandergesetzt. Seit über 16 Jahren ist er als Branchenexperte für Versicherungen bei SAP tätig und berät in dieser Funktionen Kunden zu den Kernversicherungslösungen der SAP.  

Interviewpartner 2 ist Alexander Grebert. Er ist der Global Head des Insurance Expert Teams bei SAP Fioneer, einem Joint Venture von Dediq und SAP, welches gegründet wurde, um das SAP-Portfolio für Finanzdienstleistungen gemeinsam auszubauen und in branchenspezifische Lösungen zu investieren. Vor seinem Wechsel zu SAP Fioneer hat Alexander Grebert als ehemaliger Versicherungskaufmann und Betriebswirt für Kfz-, Lebens- und Sachversicherer in Vertriebs-, Underwriting- und internen IT-Bereichen gearbeitet. Später war er als IT-Projektmanager für Erstversicherungslösungen bei der msg verantwortlich und dann in derselben Rolle für den gesamten Finanzdienstleistungssektor bei SAP. Zuletzt war er bei der SAP als \enquote{Director Insurance Architecture and Digital Transformation} tätig, bevor er im September 2021 zur SAP Fioneer wechselte.

Interviewpartner 3 ist Eduard Schmidt. Zu Beginn seiner Karriere arbeitete er in der IT-Abteilung der HDI in Hannover, wo er an der Umstellung des Bestands-, Partner- und Zahlungssystems beteiligt und für die Entwicklung von Kunden- und Vermittlerportalen verantwortlich war. Nach sechs Jahren wechselte er zur SAP und implementierte dort zunächst als Berater die Policymanagement-Lösung bei Versicherungsunternehmen, insbesondere bei Kfz-Versicherern. Heute berät er für die SAP Versicherungsunternehmen bei der Konzeptionierung und Gestaltung der Systemarchitektur.

\section{Transkription der Experteninterviews}

Von allen Teilnehmern wurde eine ausdrückliche Erlaubnis zur Transkription der Gespräche ausgestellt. Die Interviews wurden \enquote{vereinfacht} transkribiert. Das bedeutet, dass Dialekte, grammatikalische Fehler oder entbehrliche Formulierungen verbessert und Wortgenau ins Hochdeutsche übersetzt wurden.\autocite[Vgl.][S. 292]{TAUSENDPFUND2020}

\newpage
\subsection{Experteninterview 1 - Christos Lemonidis (SAP SE)}

\textit{Durchgeführt am 14.04.2023, 09:00 Uhr -- 09:39 Uhr:}


Interviewer/ Experte 1:
Teams-Programm-Test 
\#0:0:58\#

Interviewer:
Ok – dann steigen wir in das Interview ein. Zunächst vorab schon einmal vielen Dank dafür, dass du dir heute die Zeit genommen hast, mit mir das Interview zu führen. Ist es für dich in Ordnung im Rahmen meiner Projektarbeit 2 namentlich benannt zu werden oder möchtest du lieber anonym bleiben?
\#0:1:23\#

Experte 1:
Du kannst mich in deiner Projektarbeit gerne namentlich nennen.
\#0:1:29\#

Interviewer:
Alles klar, so machen wir es. Vielleicht nur so viel noch einmal zur Einleitung: Der Titel meiner Projektarbeit lautet „Digitale Plattformen als Wettbewerbsfaktor für den deutschen Kfz-Versicherungsmarkt am Beispiel der SAP Business Technology Plattform“. Das Ziel dieses Interviews ist es eben die Anforderungen der Kfz-Versicherer an digitale Plattformen zu identifizieren und danach auch zu priorisieren. Mit dem Begriff „digitale Plattformen“ sind hierbei technologische Plattformen gemeint, welche eine Menge von Kernprodukten, -technologien oder Services bereitstellen, auf deren Basis weitere Services entwickelt werden können, wie beispielsweise die SAP BTP - aber eben keine transaktionsorientierten Plattformen, wie zum Beispiel Uber. Nur so viel zur Definition, was mit dem Begriff digitale Plattform gemeint ist.
\#0:2:56\#

Experte 1:
Gut – das habe ich, soweit erst einmal verstanden.
\#0:3:01\#

Interviewer:
Ok - dann können wir also beginnen. Ich werde Dir heute im Verlauf dieses Experteninterviews 8 Fragen stellen, zu Beginn zu Deiner Person, wo Du Dich kurz vorstellen kannst, dann im Hauptteil der Fragenblock um die Anforderungen an „Digitale Plattformen“ zu identifizieren. Danach werde ich selbst einen PowerPoint-Slide auflegen mit der ich Dir die Anforderungen zeige, die ich gefunden habe. Diese dann mit Dir kurz besprechen und danach mit Dir priorisieren. Am Ende des Interviews noch eine Frage zum Ausblick. So, wenn Du keine Fragen vorab hast, würde ich jetzt mit der ersten Frage beginnen.
\#0:4:11\#

Experte 1:
Soweit keine Fragen. 
\#0:4:17\#

Interviewer:
Dann beginnen wir mit Deiner Person - welche Erfahrungen hast Du in der Versicherungsbranche und wie sieht Dein bisheriger Werdegang aus? Was ist Deine aktuelle Position im Unternehmen? Wenn Du das bitte kurz in eins bis zwei Sätzen beschreiben könntest.
\#0:4:40\#

Experte 1:
Ok - also meiner Erfahrung im Versicherungsbereich gehen tatsächlich schon knapp 30 Jahre zurück, ich hatte schon im Studium den Schwerpunkt auf Versicherungsbetriebslehre und Versicherungs¬mathematik. Ich bin seit dem Uni-Abschluss als Versicherungsmathematiker tätig und habe durch Weiterbildungen den Titel „geprüfter Aktuar der deutschen Aktuar-Vereinigung“ erlangt. Während meiner Zeit bei der Versicherung die Themen Risiko Management, IFRS, aber auch Tarifkalkulation und Rückversicherungsoptimierung gemacht. Und in diesem Rahmen habe ich mich auch mit Kfz-Tarifen auseinandergesetzt. Ja – ich bin jetzt seit 16 Jahren bei der SAP und habe auch dort den Schwerpunkt auf unseren Versicherungslösungen, insbesondere den Kern-Versicherungs¬anwendungen, wie Bestandsführungssysteme, Schadenmanagementsysteme oder Inkasso- Systemen. Genau und jetzt? Aktuell liegt mein Schwerpunkt eher Finance-and-Risk, nichtsdestotrotz sind mir natürlich die Versicherungsthemen nach wie vor gut bekannt.
\#0:5:40\#

Interviewer:
Danke - dazu kurz eine Rückfrage von mir. Du hast eben den Begriff „Aktuar-Vereinigung“ angesprochen. Was versteht man darunter, ich habe das vorher noch nicht gehört?
\#0:5:54\#

Experte 1:
Ja, genau - ein Aktuar ist der lateinische Begriff für den Versicherungsmathematiker. Er beschäftigt sich sehr sorgfältig um die Versicherungstarife und um die Versicherungsproduktion - sozusagen eben im modelltheoretischen Sinne - kümmert. Der Abschluss zum Aktuar ist eine Zusatzausbildung, die man nach dem Studium machen kann, um sich in diesem Versicherungsbereich eben noch einmal zu spezialisieren. Also so ähnlich wie bei einem Steuerberater. Diese Prüfung legt man eben bei der deutschen Aktuar Vereinigung - DAV – ab. Um den Titel dauerhaft führen zu dürfen, muss man regelmäßig Weiterbildungsveranstaltungen besuchen. Hierzu nutze ich meine jährlichen SAP-Weiterbildungsstunden, auch wenn ich aktuell im engeren Sinne als Aktuar nicht mehr tätig bin. 
\#0:6:49\#

Interviewer:
Alles klar - danke für deine Erläuterung. Jetzt können wir zu der Hauptphase des Interviews übergehen. Welche Entwicklungen waren in der Kfz-Versicherungsbranche in den letzten Jahren erkennbar und mit welchen Herausforderungen hat die Branche aktuell zu kämpfen?
\#0:7:04\#

Experte 1:
Also einerseits einmal aus der Marktsicht ist natürlich nach wie vor so, dass der Kfz-Markt hart umkämpft ist. Wir kennen wir die Werbeschlachten der einzelnen Kfz-Versicherer, die regelmäßigen Wechselspielchen am Ende jeden Jahres, die dann die „Sparfüchse“ machen. Die Kfz-Versicherung ist als Sicht der Versicherer immer noch im gewissen Sinne ein Einstiegsprodukt, um einen Versicherungskunden – im Allgemeinen -zu gewinnen. Die Kfz-Versicherung wird als „Kuppelprodukt“ angesehen. Das bedeutet, hat man ein Kfz-Versicherungsprodukt, dann kann man damit auch die anderen Versicherungsprodukte erschließen. Also ich denke, so ist die Marktsicht nach wie vor hier. Also es ist ein sehr umkämpfter Markt, wobei jetzt gerade so in den letzten 2 bis 3 Jahren hat die HUK-Coburg, die ja - so als Preisführer agiert - hier schon ihre ihrer Marktstellung ausgebaut und auch ihren Abstand auf die Allianz deutlich ausgebaut. Das kann man - meines Erachtens -schon einmal so als eine Marktbewegung einstufen. Das andere Thema sind natürlich die Telematik-Tarife und dann muss man sehen, welche in anderen Ländern deutlich mehr Fahrt aufgenommen hat, wie in Deutschland. In Deutschland sind diese Telematik-Tarife eher eine Ausnahme. Man kann sie in Deutschland eher als „Versuchsballone“ von den großen Versicherern einordnen. Bisher haben diese Telematik Tarife nicht die riesige Marktdurchdringung. Das kann man auch damit begründen, dass die Tarifwelt in Deutschland schon ziemlich ausdifferenziert ist, so dass man eine relativ große Preisspreizung zwischen risikoarmen und risikobehafteten Fahrern schon hat. Damit sind dann die Spielräume bereits relativ gering. Demzufolge ist ja auch nicht überraschend, dass Telematik Tarife – aus Versicherer-Sicht insbesondere auch für junge Fahrer attraktiv sind – ja - junge männliche Fahrer, so wie du. Die jungen männlichen Fahrer haben aus Sicht des Aktuars natürlich den höchsten Risikobedarf. Andererseits wenn diese Versichertengruppe sorgsam fahren und die das dann auch durch die Telematik-Daten nachweisen können, dass sie vorsichtige und umsichtige Fahrer sind, dann haben sie natürlich ein Hebel, auch was die Rabatte angeht. Bei den anderen Tarifen ist das so, dass da natürlich relativ wenig Spielraum in den Tarifen ist, weil das Tarifniveau in Deutschland schon relativ ausgereizt ist. Ein weiteres Thema, das für die Kfz-Versicherer dann eine Rolle spielt, dass die Autos selbst immer digitaler werden.  Der Verbrennungsmotor, also die Leistungsklasse – wieviel kW/ PS hat das Auto, spielt ja schon immer die große Rolle, aber die Elektronikkomponenten im Auto bekommen eine immer stärkere Bedeutung, wie das früher der Fall war und das bietet natürlich dann auch Möglichkeiten im Auto - während der Fahrt - Services anzubieten und das ist natürlich auch eine Chance für Kfz-Versicherer da in das digitale Business einzusteigen. Das bedeutet, dass man sich zum Beispiel in einem Porsche elektronische Zusatz-PS dazu buchen kann, wenn man das will. So etwas könnte dann natürlich auch die Fahreigenschaften vom Auto ändern, zum Beispiel auch in Abhängigkeit des Fahrzeugstandortes, als in welchem Land wird gerade das Auto bewegt. Im internationalen Verkehr könnte man sich ja schon vorstellen, dass man dann auch entsprechenden Versicherungsschutz bietet, der sich dann der Fahrsituation oder dem Land, in dem er gerade ist, anpasst.
\#0:11:35\#

Interviewer:
Ok – das heißt zusammengefasst, dass der Markt generell sehr umkämpft in der Branche ist und Telematik-Tarife für Kfz-Versicherer ein relevantes Thema sind. Alle versuchen neue Kunden zu werben und generell wird das Auto immer vernetzter, es gibt immer mehr Daten, die man im Auto irgendwie auswerten bzw. verwenden kann. In diesem Zusammenhang die nächste Frage: Wie wirkt sich das auf die IT-Systeme in den Versicherungsunternehmen aus, bzw. welche Anforderungen lassen sich daraus denn an technische Plattformen ableiten? Sprich, was muss die Plattformen alles können und in den Bereichen unterstützen zu können.
\#0:12:08\#

Experte 1:
Also – wenn man einmal bei den Telematik-Tarifen bleibt - dann ist es so, dass da eine ganze Menge Daten verarbeitet werden müssen, also da hat man das Datentransport-Problem, aber auch das Datenverarbeitungs-Problem. Einerseits ist es natürlich so, dass Telematik-Daten sehr umfangreich sind, das ist einmal die Datenmenge und zum anderen auch das Auswertungsthema. Es nützt ja nichts, dass man die ganzen Daten nur hat, sondern man muss dann ja auch die Daten verarbeiten – also auswerten. Im Sinne von, welches sind die entscheidenden Parameter, aus denen sich dann ableiten lässt, ob ein Fahrer jetzt ein guter Fahrer oder ein risikobehafteter Fahrer ist, ob er weniger oder viel fährt etc. Also da muss man natürlich dann auch das Auswertungs-Know-how aufbauen und natürlich auch die Auswertungsmöglichkeiten dann zur Verfügung stellen. Da ist sicherlich noch einiges, was man natürlich noch nachgelagert machen kann. Aber wenn man sich vorstellt, wie dann auch ein regelmäßiger Datenfluss aus dem Auto ist und man will dann entsprechende Services mit anbieten, dann geht es ja auch darum, diese Daten in Real Time zu verarbeiten. Das heißt, dass man hier sehr zeitnah etwas dann vielleicht anbieten kann. Also ich würde ich sagen, das sind so in etwa aktuell die großen Herausforderungen aus Sicht der Kfz-Versicherer - Datenmenge. Auswertung und Echtzeitverarbeitung.
\#0:13:33\#

Interviewer:
Ok - das habe ich verstanden. Da vielleicht noch die Frage in Bezug auf die technischen Plattformen: Wofür werden, deiner Einschätzung nach, technologische Plattformen - wie die SAP BTP - bei Kfz-Versicherern in den nächsten 3-5 Jahren vor allem verwendet werden? Was erscheint denn deiner Meinung nach da am realistischsten?
\#0:14:00\#

Experte 1:
Ich glaube, dass der Bereich Telematik aus Sicht der SAP jetzt gar nicht so das Entscheidende sein wird, weil das dann schon auch so ein Spezialthema ist, in dem wir jetzt ja gar nicht so engagiert sind. Aber der ganze Themenblock „zusätzliche Services“, der spielt natürlich eine Rolle, weil man zusätzliche Services anbietet, dann muss man die ja auch abrechnen können. Und dann ist man eher in einem Customer-Experience-Umfeld oder auch einem Shop-Environment. Das ist natürlich schon etwas, was für die SAP sehr interessant sein kann. Also: Wie bringe ich die Angebote, die Services jetzt nicht aus der technischen Sicht, aber wie bekomme ich die Services auch kommerziell im Auto unter? Wie kann ich die Services über die im Auto vorhandenen Kommunikationssysteme anbieten? Und wenn die dann in Anspruch genommen wird, muss es ja auch abrechnen können. Noch einmal zu dem Beispiel „zusätzliche Porsche-PS buchen“ - “ wenn die Bereitstellung von Turbo-PS für 2 Stunden 3,70 \euro kostet, dann muss das auch irgendwo eingebucht, abgerechnet und dann auch kassiert werden. Das würde ich sagen, ist dann schon ein Geschäftsfeld für die SAP.
\#0:15:33\#

Interviewer:
Ok – du hast gerade die Punkte Funktionalität und Services angesprochen. Frage: Welche Anforderungen muss - deiner Meinung nach - eine technische Plattform für Kfz-Versicherer daher unbedingt erfüllen bzw. welche Funktionalitäten und Services sollte eine technische Plattform aus deiner Sicht unbedingt mitbringen, um für die Kfz-Versicherer einen Mehrwert zu schaffen? 
\#0:15:53\#

Experte 1:
Ja, ich glaube schon das ist das Thema Echtzeitverarbeitung - das Services-Thema muss möglichst schnell gehen und zwar in dem Moment, wo es notwendig ist. Das gleiche gilt natürlich auch für die ganzen erforderlichen betriebswirtschaftlichen Prozesse, die da dahinterstehen. Die müssen halt auch in Echtzeit dann Verfügung stehen. Also das sind jetzt keine Prozesse, wo man sich eine Batch-Verarbeitung vorstellen oder wo dann im Anschluss - irgendwie einen Tag oder Woche später - nochmal dann Papierkram erzeugt wird. Sondern das muss dann ein hochdigitaler Prozess sein, der in Echtzeit dann eben auch die Prozesse darstellt, weil das wahrscheinlich auch das ist, was dann der Konsument im Fahrzeug - in seinem „digitalen Raumschiff“ - dann auch erwartet. 
\#0:16:45\#

Interviewer:
Genau, genau - eine Herausforderung der Kfz-Versicherer, von der sowohl in der Literatur als auch in der Praxis immer wieder gesprochen wird, sind die sogenannten Legacy-Systeme. Welche Anforderungen ergeben sich dadurch aus deiner Sicht an technische Plattformen? Gibt es da Besonderheiten in Bezug auf die Alt-Systeme, die zu berücksichtigen sind ?
\#0:17:09\#

Experte 1:
Ja gut - in Bezug auf die Alt-Systeme ist doch so, dass diese oftmals so eine Echtzeitverarbeitung gar nicht darstellen können, weil die Alt-Systeme batchorientiert sind. Und dann ist natürlich die Aufgabe von einer technischen Plattform, im Gewissen Sinne auch zu entkoppeln. Ja, das sie dem Kunden gegenüber Echtzeitverarbeitung ermöglicht und aber trotzdem dann Zugriff auf die notwendigen Daten, insbesondere Stammdaten, dann auch so ein Bestandsführungssystem hat. Ja, man benötigt dann schon so eine „Entkopplungsschicht“, die natürlich dann aber auch integriert sein muss.
\#0:17:52\#

Interviewer:
Ok – nachgefragt: „Entkoppeln“ heißt in dem Punkt dann zu sagen, dass man die Alt-Systeme irgendwie versucht, zu Kapseln und dann als einen geschlossenen Baustein an die Plattform anzubinden oder wie kann ich mir die Entkopplung, die du beschrieben hast, vorstellen?
\#0:18:07\#

Experte 1:
Hm - da gehe eher einmal davon aus, dass die Alt-Systeme nur in einem beschränkten Umfang API-fähig sind. Das man sich dann Modelle überlegen muss, wie schafft man es die statischen Informationen aus dem Bestandsführungssystem einzubinden bzw. zu verwenden, auch für die Services und die Services eben dann kundenorientiert auszuprägen und auf der anderen Seite die Agilität, die man benötigt, um an der Kundenschnittstelle schnell und qualitativ hochwertige Services zur Verfügung stellen zu können.
\#0:19:02\#

Interviewer:
Du hast gerade eben schon das Schlagwort Schnittstellen angesprochen, gibt es hier typische Schnittstellen, die bei Alt-Systemen häufig vorzufinden sind. Aktuelle ist ja häufig State of the Art, die sogenannten REST-Schnittstellen. Früher war glaube ich auch viel SOAP im Einsatz. Hast du dazu Erfahrungen gesammelt?
\#0:19:22\#

Experte 1:
Ja, es ist ja schon so, dass die Versicherungen auch heutzutage da so einiges machen müssen. Aber das sind dann oft auch so Kupplungsarchitekturen, die bereits in einem Versicherungsunternehmen jetzt schon vorhanden sind, um Kundenschnittstellensysteme mit den Backend-Systemen verbinden zu können. Ob das dann direkt immer möglich ist, es wird wahrscheinlich in dem ein oder anderen Einzelfall gehen, aber in der Regel sind da schon eben genau solche Kupplungsarchitekturen not¬wendig, um Agilität auf der einen und statische Informationen auf der anderen Seite zu verknüpfen.
\#0:20:02\#

Interviewer:
Ok - dann würde ich jetzt einmal zu meiner PowerPoint kommen: Und zwar habe ich im Rahmen meiner Arbeit aus der Literatur Anforderungen der Kfz-Versicherer an technische Plattformen herausgearbeitet, von denen einige bereits genannt wurden, und diese zur besseren Strukturierung in verschiedene Cluster eingeteilt. Diese werde ich dir jetzt kurz vorstellen und dich bitten Bezug zu den einzelnen Anforderungen zu nehmen: Sprich kannst du diese bestätigen und sieht du in dem jeweiligen Cluster noch weitere Anforderungen die für Kfz-Versicherer bei technischen Plattformen besonders wichtig sind? Einen kleinen Moment - so du müsstest jetzt meinen Bildschirm sehen.
\#0:20:57\#

Experte 1:
Ja, mit der ersten Spalte Integration.
\#0:21:00\#

Interviewer:
Ganz genau - so bei den Anforderungspunkten, die ich hier im Cluster „Integration“ gefunden habe und du hattest hiervon eben schon 1 bis2 genannt gehabt, sind zu nennen „Unterstützen einer service-orientierten Architektur“, um eben auch – wie du meintest - die Alt-Systeme koppeln zu können bzw. kapseln zu können.  Um diese dann als Services für neuere Applikationen entsprechend zur Verfügung zu stellen. Die Alt-Systeme müssen eben an die neuen Plattformen angebunden werden können, das ist eine ganz wichtige Grundvoraussetzung für die Plattformen. Diese müssen mit offenen Schnittstellen-Standards letztlich kompatibel sein. Ich weiß nicht, ob dir der Begriff „digitale Ökosysteme“ etwas sagt, was ja aktuell auch so ein Trendthema in der Versicherungs¬branche ist. Da ist es ja so, dass gerade mit Kunden, Partnern oder auch Drittanbietern von Applikationen der Datenaustausch sehr wichtig ist und dazu wird eben aktuell häufig auch die REST-Schnittstelle verwendet, die man hier als State-Of-The-Art bezeichnen kann. Und eben als vierten Anforderungspunkt dieses Clusterbereichs das „API Management Tool“, sprich es gibt immer mehr Schnittstellen, die müssen eben auch kontrolliert und verwaltet werden. Sprich - wer greift auf die ganzen API überhaupt zu? Dazu an dich die Frage: Kannst du die einzelnen Anforderungen bestätigen – siehst du vielleicht im Bereich „Integration“ darüber hinaus noch weitere wichtige Punkte?
\#0:22:14\#

Experte 1:
Ja, ich kann die von dir genannten Punkte so bestätigen, ich würde aber vielleicht noch den Punkt „Kopplungsarchitektur“ noch mit aufnehmen.
\#0:22:22\#

Interviewer:
Ok – dann notiere ich mir das direkt einmal - so. Das nächste Cluster wäre dann der Bereich „Entwicklung“. Da ist es, ich glaube, zum einen ganz wichtig, dass man auf diesen technischen Plattformen eben neue Anwendungen entwickeln und auch betreiben kann. Das gilt auch für mobile Anwendungen - das ist ja auch ein Thema, das aktuell immer wichtiger wird. Im Idealfall sollten auch bestehende Anforderungen erweitert werden können, um eben die einzelnen Bedarfe der Versicherer anpassen zu können. Es sind aktuell ja – gerade auch im Versicherungsbereich – die Entwicklungsressourcen sehr knapp und wenn da die technische Plattform auch gleichzeitig Low-Code- und No-Code-Werkzeuge zur Verfügung stellen kann, dann würde das sicher auch helfen, dass vielleicht die Leute aus den Fachabteilungen auch selbst einmal mit einfachen Drag-and-Drop-Systemen Erweiterungen bauen können. Als weitere Anforderung ist - auch im Bereich der Telematik-Tarife - der Punkt „Entwicklung von IoT-Anwendungen“ zu nennen. Den hattest du ja auch schon vorhin kurz angesprochen. Da wäre auch hier die Frage an dich: Hast du zu diesen Anforderungspunkten bereits Erfahrungen gesammelt? Oder gibt es vielleicht noch weitere Punkte im Bereich der Entwicklung von Applikationen für diese technische Plattform, die ich jetzt noch nicht genannt habe?
\#0:23:38\#

Experte 1:
Nein, das passt ganz gut. Den ersten Punkt würde ich vielleicht noch einmal deutlicher herausstellen. Der ist – meines Erachtens - im Vergleich zu den nachfolgenden Punkten eine ganz wichtige Anforderung. Die anderen 3 Punkte sind - sage ich einmal so – eher nach innen gerichtet sind. Aber, dass man hier mobile Anwendungen hat, das ist ja eigentlich - gerade jetzt bezogen auf Kfz - im doppelten Sinne wichtig, auch bezüglich der Kleinteiligkeit der Anwendungen. Dann vielleicht noch ein weiterer wichtiger Aspekt, dass die technische Plattform in andere oder größere Anwendungen eingebettet ist. Es ist ja nicht so, dass die Versicherung dann das führende System ist, daher ist es wichtig bestimmte Funktionalitäten dann direkt in einem Navigationssystem oder auch in ein Fahrzeug-System anbieten zu können. Daher vielleicht noch das „Einbetten in andere Anwendungen“ als zusätzlichen Anforderungspunkt. Und bei Punkt 1 ohne die Klammer um den Begriff „mobile“, weil das - meiner Meinung nach - die Kernanforderung hier ist. 
\#0:24:47\#

Interviewer:
Frage dazu: Das heißt, du würdest dann den Punkt „mobile Anwendungen“ auch als einzelnen Punkt gezielt hervorheben? 
\#0:25:09\#

Experte 1:
Ja, das finde ich eigentlich schon wichtig und wie schon gesagt, dass diese Anwendungen eingebettet sind - in andere Systeme, weil es nicht das führende System darstellen wird.
\#0:25:24\#

Interviewer:
Ok, das habe ich verstanden. Das nächste Cluster wäre der Bereich „Datenverarbeitung“. Da glaube ich, ist der Punkt „spartenübergreifende Sicht auf Daten“ ein ganz wichtiger Punkt für die Kfz-Versicherer, um eben ein besseres Kundenverständnis zu bekommen und auch den Kunden gezielte Angebote machen zu können, wie eben im Bereich von Telematik-Tarifen. Auch das „Empfangen von Sensordaten und von physischen Objekten“ ist als Anforderungspunkt für diese technischen Plattformen- wie beispielhaft bei Autos - glaub ich auch sehr wichtig. Aufgrund der unstrukturierten Datenbasis müssen Daten mit einer Extract-Transform-Load-Lösung aufbereitet werden. Oder, was du auch schon meintest, vor allem im Bereich der vernetzten Autos eben die Analyse der großen Datenmengen. Auch für dieses Cluster wieder die Frage: Siehst du weitere Anforderungen oder passt die Aufzählung soweit?
\#0:26:05\#

Experte 1:
Das passt meiner Meinung nach so, die Anforderungen sind vollständig.
\#0:26:08\#

Interviewer:
Ok, sehr gut. Mit Blick auf Zeit – wir haben noch maximal eine Viertelstunde - würde ich jetzt direkt weiter gehen. Das vierte Cluster wäre der Bereich „Prozessautomatisierung“, hier eben zum einen durch RPA. Im Rahmen der Literaturrecherche hat sich gezeigt, dass bei Versicherungen auch viele repetitive Prozesse bestehen, die trotzdem noch manuell von Mitarbeitern ausgeführt werden und eben noch nicht automatisiert sind. Sprich – da könnte man RPA ganz gut verwenden, unter anderem auch die Automatisierung von Kundeninteraktionen, gerade bei Kundenanfragen, wie zum Beispiel Versicherungspolicen abrechnen oder Schadensmeldungen durchführen, da könnte eben der Chatbot im Vergleich zum normalen Sachbearbeiter zu jederzeit antworten - und man könnte eben auch Wartezeiten vermeiden. Da wäre wiederum die Frage: Stimmst du den Anforderungspunkten im Bereich „Prozessautomatisierung“ zu, siehst du vielleicht noch weitere, die nicht genannt wurden?
\#0:27:01\#

Experte 1:
Hm, da würde ich - auch wieder unter dem Gesichtspunkt deiner eigentlichen Fragestellung – den Anforderungspunkt 14 als deutlich relevanter ansehen, wie den Punkt 13, weil – ich sage einmal so –„Prozessautomatisierung durch RPA“ ist ja eigentlich der Blick nach hinten. Das ist ja eigentlich nur dafür da bestehende Mängel der existierenden Software zu beheben. Die spielen zwar da sicherlich eine Rolle, aber im Hinblick auf digitale Prozesse ist es ja auf jeden Fall das Ziel, dass solche Nachbearbeitungen durch Roboter überhaupt erst gar nicht anfallen, sondern dass die direkt bearbeitet werden. Also das das würde ich dann noch unterschiedlich priorisieren.
\#0:27:58\#

Interviewer:
Ok, das können wir dann gleich noch machen. In dem letzten Cluster „Weitere Anforderungen“ kommen – meiner Meinung nach – noch 2 weitere wichtige Punkte. Zum einen der Punkt „grundlegende Datensicherheit“, da können die Versicherungsunternehmen – glaube ich - schlecht die Sicherheit von Systemen komplett überprüfen. Daher greifen sie auf Zertifizierungen zurück, welche zum einen im Bereich Informationssicherheit-Management-Systeme, als auch im Bereich DSGVO benannt wurden, das sind hier die beiden ISO-Normen und eben der Punkt „Vertikale und horizontale Skalierbarkeit“ -gerade um auch mit den hohen Anfragen bei Kfz-Versicherern, beispielsweise wie sie im Herbst aufgrund der vielen Tarif-Wechsel-Kündigungen stattfinden, klar zu kommen und entsprechend weitere Ressourcen im Rahmen der Skalierbarkeit zur Verfügung  zu stellen. Stimmst du meinen Anforderungen da auch zu? 
\#0:28:44\#

Experte 1:
Ja, das passt sehr gut. Also zusammenfassend – als allgemeine Anmerkung - würde ich jetzt noch einmal versuchen die einzelnen Anforderungspunkte nicht gleichberechtigt nebeneinander zu stellen, sondern entsprechend zu priorisieren. Das heißt, noch einmal die Anforderungspunkte in Vordergrund zu stellen, die dann besonders mit digitalen Versicherungsprozessen zu tun haben. 
\#0:29:21\#

Interviewer:
Warte – den Punkt meintest du gerade, würdest du bei der „Integration“ sehen – ok. Da war noch der Punkt „Einbindung in andere Systeme“ - den hattest du dem Bereich „Entwicklung“ zugeordnet– ist das richtig so?
\#0:29:43\#

Experte 1:
Hm - ich würde den Anforderungspunkt „Einbettung in andere Systeme“ nennen.
\#0:29:51\#

Interviewer:
Genau -optional würdest du hier noch die „mobilen Anwendungen“ extra – mit hoher Priorität aufführen?
\#0:30:13\#

Experte 1:
Ja, also mir ist wichtig festzuhalten, dass die aufgeführten Anforderungspunkte in jedem Fall nicht alle gleich wichtig sind für Kfz-Versicherer. 
\#0:30:23\#

Interviewer:
So dann würde ich jetzt noch einmal zum Ende unseres Interviews kurz mit dir die Punkte priorisieren wollen. Und zwar wäre es jetzt hier die Idee, wir haben jetzt hier die einzelnen Anforderungen, die eine Plattform erfüllen sollte und dann die Bitte an dich die Wichtigkeit der einzelnen Punkte von 1 bis 3 zu priorisieren und wo die eben den Kfz-Versicherern vor allen Dingen den nächsten 3 bis 5 Jahren helfen können und vielleicht bei jedem kurz mit 1-2 Sätzen kurz beschreiben, warum du den Punkt der jeweiligen Priorität zuordnest -von hoch bis niedrig.
\#0:31:01\#

Experte 1:
Okay, also, dann fangen wir mal mit Anforderung Nr. 3 an, das würde ich alles hoch einstufen. Dann das API Management Tool als mittel einstufen, die anderen 2 Punkte als niedrig.
\#0:31:28\#

Interviewer:
Die Punkte „Legacy-Systeme“ und „Serviceorientierte Architektur“ als niedrig einzustufen, weil das in der Regel schon vorhanden schon vorhanden ist oder weil es einfach generell nicht so wichtig ist, was wäre da die Erklärung?
\#0:31:37\#

Experte 1:
Das ist – meiner Meinung nach - ja ein grundsätzliches Problem, dass die Versicherer an der Stelle haben. Das würde ich jetzt nicht unbedingt aus Sicht von KFZ an eine technische Plattform in Richtung Kundenschnittstelle als wichtigste Priorität sehen – Ok?.
\#0:31:52\#

Interviewer:
Heißt das, das sind Punkte, die deiner Auffassung nach sowieso gemacht werden müssen oder siehst du das so ein bisschen nach dem Motto. „Das wird sowieso funktionieren“ Kann ich mir das etwa so vorstellen?
\#0:32:08\#

Experte 1:
Nein – ich sage einmal, das sind grundlegende Themen, mit denen sich die Versicherer beschäftigen müssen, unabhängig davon, was sie sich im Kfz-Versicherungsbereich für zusätzliche Services vorstellen.
\#0:32:21\#

Interviewer:
Ok – das heißt, du würdest: Es ist ein allgemein wichtiger Bereich, aber in Bezug auf die Plattform-Anforderungen nicht so wichtig, weil es dabei andere gibt, die in jedem Fall wichtiger sind, wie zum Beispiel die offenen Schnittstellen.
\#0:32:33\#

Experte 1:
Ja genau – so ist es.
\#0:32:35\#

Interviewer:
Ok - dann würde ich sagen, wir machen mit dem Bereich „Entwicklung“ in der Priorisierung weiter.
\#0:32:38\#

Experte 1:
Ja, dann würde ich die „mobilen Anwendungen“ nach oben - als hohe Priorität.
\#0:32:41\#

Interviewer:
Genau und das hatten wir bereits ja auch schon kurz besprochen.
\#0:32:45\#

Experte 1:
Ebenso der Punkt „Einbettung in andere Systeme“, insbesondere die mobilen Anwendungen.
\#0:32:49\#

Interviewer:
Genau. Du würdest diesen Punkt auch ganz oben, sehen?
\#0:32:53\#

Experte 1:
Ja, genau dann das den ersten Punkt „Entwicklung und Betrieb neuer Anwendungen“ dann auch bei Mittel. Der Punkt IoT-Anwendungen und die anderen 2 wieder in die Priorität niedrig.
\#0:32:58\#

Interviewer:
OK - also würdest du die Verwendung von Low-Code-/ No-Code-Werkzeugen nicht als so wichtig ansehen, weil der Bedarf nicht da ist oder weil diese noch gar nicht so weit sind, was ist da deine Begründung?
\#0:33:18\#

Experte 1:
Wie bereits gesagt, die Bewertung sollte immer aus Sicht der grundlegenden Fragestellung erfolgen. Also, was sind so die Anforderungen. Ich möchte ja vorne an der Kundenschnittstelle etwas Neues machen und wie ich das dann letztendlich umsetze, das sehe ich eher als nachrangig an. Das Entscheidende ist es ja nicht, ob er das in einem Low-Code oder No-Code umsetzt, sondern dass die Anwendung funktioniert. Das Werkzeug ist dann an der Stelle nicht entscheidend.
\#0:34:07\#

Interviewer:
Ok - und dann noch genau der Punkt „Erweiterungen bestehender Anwendungen“, den du ganz unten einsortierst. Würdest du sagen, dass die Anwendungen, die eine technische Plattform bereitstellen soll, gar nicht groß erweitert werden möchten, da der Bedarf nicht so groß ist oder weshalb ist hier die Einordnung ganz unten?
\#0:34:21\#

Experte 1:
Also erst einmal muss ja auch irgendetwas unten stehen, wenn man priorisieren will. Ich glaube nicht, dass die funktionale Erweiterung der Back-End-Systeme hier im Vordergrund steht.
\#0:34:35\#

Interviewer:
Alles klar, okay- Das habe ich verstanden. Das sind die Punkte „IoT-Anwendungen“ und „Einbettung in andere Systeme“ wichtiger. Könntest du für mich bitte noch einmal mal - in 3 bis 4 Sätzen - den Anforderungspunkt Einbettung in andere Systeme“ kurz erläutern, also was du dir darunter vorstellst?
\#0:34:48\#

Experte 1:
Also wie gesagt, das was an der Kundenschnittstelle im Fahrzeug passieren wird, da wird ja der Versicherer das führende System hinstellen. Sondern es geht darum, also wenn man das einmal aus einer Gesamtplattform-Diskussion sieht, die hast du ja ausgegrenzt, ist es immer die Frage: Wer ist der Plattformbetreiber? Ja, und da wird die Versicherung niemals der Plattformbetreiber hinsichtlich der Kfz-Services sein, sondern es geht darum, bestimmte „Versicherungs-Schnipsel“ in einer größeren Anwendung zur Verfügung zu stellen. Man muss in der Lage sein, sich auch in andere Systeme einzubetten zu können, weil man als Versicherer selbst nie das führende System sein wird.
Ich weiß nicht, ob der Kunde dann eine Versicherungs-App in seinem Navigationssystem erwartet oder ob er einfach im Rahmen von einem Service, den er da sprichwörtlich „unter der Motorhaube“ konsumieren will, die Versicherung mit dazu bekommt. Dem entsprechend muss dann der „Software-Schnipsel“ der Versicherung darauf ausgerichtet sein muss, in einem anderen System im Hintergrund wirken zu können.
\#0:36:07\#

Interviewer:
Ok – sprich: OEM macht eine Plattform und Versicherung ist mehr Partner als eigentlicher Plattform-Betreiber. Dann würde ich sagen - mit Blick auf die Zeit - kommen wir noch zum Bereich „Datenaufbereitung“. Auch hier wieder 4 Anforderungspunkte - wie würdest du diese einsortieren? „Spartenübergreifende Sicht auf Daten“ bzw. „Empfangen von Sensordaten“ - genau?
\#0:36:29\#

Experte 1:
Hm – ich würde sagen „Analyse großen Datenmengen“ nach oben, die Sensordaten als zweit wichtig und die anderen eher nachrangig, weil es wieder allgemein ist.
\#0:36:42\#

Interviewer:
Ok – dazu noch einmal die Nachfrage im Bereich „Datenaufbereitung“ - gerade in der Analyse müssen die Daten hier auch zunächst aus der unstrukturierten Form erstmal aufbereitet werden, um sie dann Analysieren zu können. Aber für dich eben trotzdem die Priorität Datenanalyse und nicht die Datenaufbereitung da?
\#0:36:58\#

Experte 1:
Ja, das sind ja allgemeine Anforderungen. Beim nächsten Punkt – das habe ich ja vorhin schon gesagt - da wird man sagen, den Punkt „Kundeninteraktion“ würde ich hoch einstufen und den Punkt „RPA-Prozessautomatisierung“ würde ich als niedrig einstufen.
\#0:37:14\#

Interviewer:
Ok - und die letzten beiden Punkte „Datensicherheit“ und „Skalierbarkeit“ – wo sind die deiner Meinung nach einzuordnen?
\#0:37:25\#

Experte 1:
Ja, da würde ich „Datensicherheit“ von der Priorität sogar hoch einstufen, wo wir uns jetzt dann auch gegenüber anderen uns differenzieren, den Punkt „Skalierbarkeit“ als mittlere Anforderung.
\#0:37:38\#

Interviewer:
Alles klar, wunderbar - dann haben wir jetzt die Priorisierung abgeschlossen. Bitte noch einmal kurz darauf schauen – ob die Anforderungen deiner Einschätzung so richtig priorisiert sind.
\#0:37:59\#

Experte 1:
Nein - ich denke die Priorisierung, die ist ja jetzt schon ausdifferenziert – das passt.
\#0:38:07\#

Interviewer:
Wunderbar - dann sind wir jetzt am Ende des Interviews angelangt. Die abschließende Frage wäre jetzt: Gibt es noch etwas zum Thema „digitale Plattformen als Wettbewerbsfaktor für den deutschen Kfz-Versicherungsmarkt“, was du noch hinzufügen möchtest, über das wir bisher noch nicht gesprochen haben?
\#0:38:24\#

Experte 1:
Nein – es wurde alles gesagt.
\#0:38:26\#

Interviewer:
Ok -  sprich da auch da alle Punkte genannt. Vielen Dank, dass du dir die Zeit genommen hast, mit mir ein Experteninterview zu führen und meinen Fragen Rede und Antwort zu stehen. Dann würde ich jetzt hier an dieser Stelle das Recording beenden – wie gesagt noch einmal herzlichen Dank!
\#0:38:42\#


\newpage

\subsection{Experteninterview 2 - Alexander Grebert (SAP Fioneer)}

\textit{Durchgeführt am 14.04.2023, 10:30 Uhr -- 11:09 Uhr:}

Interviewer/ Experte 2:
Teams-Programm-Test 
\#0:0:03\#

Interviewer:
Vielen Dank, dass du dir die Zeit nimmst mit mir dieses Experteninterview im Rahmen meiner Projektarbeit 2 durchzuführen. Ist es für dich in Ordnung hierbei namentlich benannt zu werden oder möchtest du lieber anonym bleiben?
\#0:0:10\#

Experte 2:
Nein, das ist kein Problem – die namentliche Nennung ist ok - alles im grünen Bereich.
\#0:0:12\#

Interviewer:
Ok, dann fangen wir jetzt an. Ich hatte dir ja schon einmal meinen Leitfaden für das Interview gezeigt. Vielleicht nur so viel noch einmal zur Einleitung: Der Titel meiner Projektarbeit lautet „Digitale Plattformen als Wettbewerbsfaktor für den deutschen Kfz-Versicherungsmarkt am Beispiel der SAP Business Technology Plattform“. Das Ziel dieses Interviews ist es eben die Anforderungen der Kfz-Versicherer an digitale Plattformen zu identifizieren und danach auch zu priorisieren. Mit dem Begriff „digitale Plattformen“ sind hierbei technologische Plattformen gemeint, welche eine Menge von Kernprodukten, -technologien und entsprechende Services bereitstellen, auf deren Basis weitere Services entwickelt werden können, wie beispielsweise die SAP BTP - aber eben keine transaktionsorientierten Plattformen, wie zum Beispiel Uber. Nur so viel zur Definition, was mit dem Begriff digitale Plattform gemeint ist.
\#0:0:42\#

Experte 2:
Gut, das habe ich verstanden. Okay, du wolltest wissen, was mein Versicherungsbackground ist beziehungsweise was ich bisher in der Vergangenheit angestellt habe. Ursprünglich komme ich aus der Versicherungs¬branche, ich habe zunächst Versicherungskaufmann gelernt, danach habe ich dann Versicherungs¬betriebswirtschaft und Banking studiert – mit dem Abschluss Master. Ich habe ca. 10 bis 11 Jahre bei Versicherungsunternehmen in unterschiedlichen Funktionen gearbeitet. Die Stationen waren bei Lebensversicherern, auch bei Sachversicherern - eben auch bei Kfz-Versicherern. Und war - bevor ich zur „SAP Fioneer“ gekommen bin - 16 Jahre lang bei der SAP und habe mich dann im Wesentlichen eben auch um Kernversicherungslösungen gekümmert. Einerseits in der Entwicklung, als auch eben in der Projektdurchführung, beziehungsweise im „Project delivery“. Ich habe meinen Schwerpunkt im Bereich Bestand und Vertriebssysteme. Ich war schon innenhalb der SAP der Spezialist für Versicherungen gewesen und bin es jetzt auch noch innerhalb der SAP Fioneer.
\#0:1:42\#

Interviewer:
Wunderbar – dann ist bei dir ein sehr großer Versicherungs- als auch IT-Background gegeben. Dann würde ich jetzt einmal zur Hauptphase meines Interviews übergehen. Und zwar - welche Entwicklungen waren in der Kfz- Versicherungsbranche in den letzten Jahren erkennbar und mit welchen Herausforderungen hat die Branche aktuell zu kämpfen?
\#0:1:58\#

Experte 2:
Ich geh einmal davon aus, dass sich deine Frage primär auf den deutschen Markt bezieht, weil sich der deutsche Markt im Vergleich zu den internationalen Märkten schon ein bisschen anders verhält. Ich möchte aber trotzdem kurz noch einmal auf die Internationalität an dieser Stelle eingeben. Wodurch der deutsche Markt eben geprägt ist, dass er nach wie vor eben als Eintrittskarte für eine größere Kundenbeziehung zu verstehen ist. Das heißt also, hier ist der Preiswettbewerb und der Preiskampf immer noch sehr entscheidend. Du weißt, dass es jedes Jahr dieses „Rennen“ am Ende des Jahres gibt, wo man als Kunde seine Kfz-Versicherung von A nach B übertragen kann. Jeder will irgendwie den günstigsten Preis aushandeln und das ist natürlich etwas, wodurch der Kfz-Markt in Deutschland sehr stark geprägt ist. Das heißt also auch bei vielen Versicherungsunternehmen ist das nicht notwendigerweise der Gewinnbringer, es ist oftmals die Sparte, die auch sehr defizitär ist. Ok, das ist in einem anderen Land, in dem ich jetzt beispielsweise lebe, überhaupt nicht der Fall. Ich komme aus der Schweiz - ich bin zwar Deutscher, wohne aber in der Schweiz und da gibt es diese Form von Versicherungswettbewerb nicht. Hier wollen sich die Versicherer vor allem über Dienstleistungen und komplementäre Services zu differenzieren. Da gibt es eben auch sehr viele Sachen, die kennst du so in Deutschland bei den Versicherern nicht. Also, wenn mir jetzt beispielsweise irgendwie ein Unfall passiert und ich den verschuldet habe, dann habe ich so etwas wie eine „Wildcard“ bei meiner Versicherung. Das heißt also, ich werde nicht notwendigerweise dann im Folgejahr entsprechend zurückgestuft, sondern ich habe einen Schaden frei - so ungefähr. Das ist natürlich etwas, was sich dann auch irgendwo im Preis niederschlägt. Überhaupt keine Frage, die machen das ja nicht „for free“. Du zahlst natürlich dafür, aber - wie gesagt - über solche Sachen differenzieren sich die Versicherer. Auch ist beispielsweise die Preisgestaltung in der Schweiz komplett anders. Also eine Sache, die in Deutschland aus gutem Grund undenkbar ist, dass du die Nationalität als Tarifierungsmerkmal heranziehst. Das bedeutet, wenn jemand beispielsweise vom Balkan kommt, der typische „3er-BMW-Fahrer“, der hat in der Schweiz ein entsprechendes Manko. Das heißt, der Kunde zahlt unter Umständen eben - bei sonstigen gleichen Parametern - aufgrund seiner Herkunft, seiner Nationalität das Doppelte an Prämie. Weil statistisch halt erwiesen ist, dass diese „Kundengruppe“ eben vermehrt für bestimmte Unfallereignisse verantwortlich, also schuld ist. Wenn du den Blick dann weiter in Richtung Osten, beispielsweise in Richtung Asien wirfst, dann spielt das Thema Mobilität eine ganz andere Rolle. Dann geht es natürlich nicht ausschließlich um Kfz-Versicherungen, sondern da kommen auch ganz andere motorisierte Fahrzeuge dazu – wie zum Beispiel die Scooters, die ganzen 2-Wheelers. Da brauchst du nur einen Blick nach Südostasien zu werfen.
\#0:5:26\#

Interviewer:
Ok - dort ist die Lage ein bisschen anders. Meine Arbeit bezieht sich ja auf den deutschen Markt. Da hast du gerade auch die Herausforderung genannt, dass das hier ein sehr umkämpfter Markt ist, wo der Preis eine sehr große Rolle spielt. Wie wirkt sich das denn auf die IT-Systeme in den Versicherungsunternehmen aus bzw. lassen sich daraus auch vielleicht Anforderungen an technischen Plattformen ableiten - also was die Plattform unbedingt können muss?
\#0:5:43\#

Experte 2:
Also in dem Sinne, dass die technischen Plattformen - ich sag einmal - relativ einfach zu integrieren ist, sich auch in Ökosysteme eben einbringen kann. Da spielen natürlich solche Geschichten, wie Vergleichsportale oder Aggregatoren, als Vertriebskanäle eine ganz große Rolle. Also das Stichwort an der Stelle ist dann „check 24“ oder vergleichbares. Damit ergeben sich dann natürlich auch entsprechende Anforderungen an so eine Technologieplattform, was eben Laufzeitverhalten, Performance oder ähnliches angeht. Dinge, wie „Easyness“ oder „Time-to-Market“ spielen an der Stelle eine Rolle. Also ich möchte ja nicht, nur weil ich jetzt irgendeine Tarifierung anpassen oder ändern möchte, immer in ein Entwicklungsprojekt mehr oder minder abdriften und es geht auch beispielsweise darum, solche Sachen wie Realtime-Pricing zu unterstützen. Das heißt, also auf aktuelle Marktgegebenheiten eben relativ schnell zu reagieren - also eine Outside-Perspektive zu haben als Trigger. Aber auch beispielsweise Informationen, die ich in meinem eigenen Bestand habe, eben mit zu berücksichtigen, also ein adäquates Pricing aus aktueller Sicht mit Bezug auf meinem Bestand, weil ich ja nicht notwendigerweise defizitär sein möchte, aber mich natürlich auch den Wettbewerb auf der anderen Seite eben stelle. Das ist diese Outside-Perspektive, als auch die Inside-Perspektive. Ok? Das spielt natürlich eine große Rolle, wenn du das in so einem integrativen Konstrukt im Rahmen von bestehenden Ökosystemen, Vertriebskanälen, Aggregatoren, Check 24 und was man sich da immer so vorstellen kann, eben anwendest.
\#0:7:35\#

Interviewer:
OK, das habe ich verstanden - sprich Integration, offene Schnittstellen unglaublich wichtig und eben auch die Echtzeit-Datenverarbeitung, um so etwas überhaupt anbieten zu können.
\#0:7:44\#

Experte 2:
Und Performance - das ist, glaube ich, auch einer der wesentlichen Issues, einer der großen deutschen Versicherer, die die SAP bzw. die „SAP Fioneer“ als Kunde hatte, macht es jetzt, glaube ich, nicht mehr, aber die haben sehr intensiv mit Check 24 zusammengearbeitet und haben sich eben aus Performance¬gründen eine eigenständige Antragsstrecke komplett aufgebaut. Weil sie gemerkt haben, wenn sie jetzt einfach ihre bestehenden Backend-Systeme nach außen kehren würden, dass das, den Anforderungen der zusätzlichen Performance-Last, die sich daraus ergeben würde, nicht Rechnung trägt.
\#0:8:25\#

Interviewer:
Was wären denn da im Bereich Performance so klassische Anforderungen hinsichtlich Integration, Entwicklung, Datenverarbeitung, Automatisierung? Woran denkst du dabei, wenn du sagst Performance ist unglaublich wichtig?
\#0:8:42\#

Experte 2:
Also - im wesentlichen Berechnung und Tarifierung.
\#0:8:48\#

Interviewer:
OK, verstanden - sprich optimale Preis- und Risikokalkulation ist für Versicherer sehr wichtig.
\#0:9:04\#

Experte 2:
Ja, genau das generelle Laufzeitverhalten. Also wenn ich jetzt beispielsweise den Rechenkern oder ähnliches aufrufe, dass ich eben relativ schnell zum Ergebnis komme und nicht erst 30 Sekunden warten muss. Weil dann schlicht und ergreifend beispielsweise jemand abwandert. Du musst ja auch das Geschäftsmodell dir betrachten, was dann beispielsweise Check 24 oder vergleichbare Portale eben betreiben. Sie bieten dir nicht zielgerichtet nur ein Angebot, sondern unter Umständen 5 Angebote zum Vergleich an. Wenn diese 5 Angebote natürlich alle irgendwie eine „lahme Krücke“ haben, was die Tarifierung und die API-Aufrufe angeht, wirkt sich das ja auch entsprechend negativ auf Check 24 oder andere vergleichbare Portale aus.
\#0:9:42\#

Interviewer:
Ok, das habe ich verstanden - dann noch eine weitere Frage: Wofür werden - deiner Einschätzung nach - technologische Plattformen bei Kfz-Versicherern in den nächsten 3 bis 5 Jahren vor allem verwendet werden? Was könnte das sein?
\#0:9:53\#

Experte 2:
Na ja, also ich will jetzt nicht bestimmte Sachen oder Passwords überstrapazieren, aber IoT ist natürlich ein Thema - überhaupt gar keine Frage. Der Klassiker, der sich da auch mittlerweile im Markt durchgesetzt hat, ist das Thema „Pay-As-You-Drive“, also Tarife bezogen auf Kilometer oder ähnliches. Also so zu verhalten, dann auch ein adäquates Pricing anzubieten. Ich glaub das Ganze geht aber auch unter Umständen einfach noch einen Schritt weiter, als dass wir eben konsequenter die Informationen, die in einem Fahrzeug existent sind, eben dann auch für solche weitergehenden Finanzdienstleistungen-Prozesse nutzen. Also du siehst ja auch, dass beispielsweise solche Unternehmen wie Tesla, dass die den Kunden eigene Finanzdienstleistungen und Versicherungen zusammen mit ihren Fahrzeugen dann verkaufen - Ok.
\#0:10:51\#

Interviewer:
Ja – das bedeutet die Erhebung von Daten im Auto ist sehr wichtig, um die dann eben auch in den Lösungen mit verarbeiten zu können und entsprechend integrieren zu können.
\#0:11:01\#

Experte 2:
Richtig - zum Beispiel. Daraus ergeben sich auch unter Umständen ganz andere Bedürfnisse, beziehungsweise auch andere Versicherungsformen - das heißt also, brauche ich eigentlich wirklich immer den Rundumschutz am Tag, bspw. wenn das Auto davon 12 Stunden am Tag in der Garage steht.
\#0:11:21\#

Interviewer:
Ja, genau - das ist auch eine ganz entscheidende Frage. Wir hatten es gerade schon angesprochen, ganz kurz der Punkt, dass zum Beispiel auch Sensordaten verarbeitet werden müssen. Gibt es denn noch weitere Anforderungen oder Funktionalitäten und Services, die eine Plattform unbedingt mitbringen sollte, um eben bei Kfz-Versicherern in den nächsten 3 bis 5 Jahren einen Mehrwert schaffen zu können?
\#0:11:43\#

Experte 2:
Also, ich habe das ja schon einmal mit dem ersten Teil angedeutet - also für mich ist eine wichtige Anforderung, dass eben Versicherer auch in der Lage sind zu erkennen, dass sie nicht notwendiger¬weise das führende Produkt mehr anbieten, sondern dass sich das eigentlich mehr oder minder anderen Transaktionen oder anderen Gegebenheiten eben unterordnet. Ja, sprich, wenn ich Mobilität kaufe, sei es jetzt durch den Kauf eines Fahrzeuges und ich bin der Besitzer eines Fahr¬zeuges oder sei es bspw. durch die Miete eines Fahrzeuges für einen bestimmten Zeitraum, dann spielt das Thema Versicherung da natürlich auch mit rein. Das ist bis zu einem gewissen Grad natürlich auch wieder diese ganze „Pay-As-You-Go“- oder die „Pay-As-You-Drive“ Geschichte. Eigentlich läuft es darauf hinaus, dass ich Versicherungen im Prinzip konsumierbar mache und zwar dahingehend konsumierbar mache, dass ich sie eben auch mit einer Transaktion oder mit einem Gut verknüpfe. Das, glaube ich, stell eine deutlich höhere Attraktivität für den Kunden dar als das eigentliche Kernversicherungsprodukt selbst.
\#0:13:02\#

Interviewer:
Eine Herausforderung der Kfz-Versicherer, von der sowohl in der Literatur als auch in der Praxis immer wieder gesprochen wird, sind die sogenannten Legacy Systeme. Lassen sich daraus weitere Anforderungen ableiten oder Faktoren, die da für technische Plattformen besonders wichtig sind, eben in Bezug auf diese Legacy-Systeme, sprich die Alt-Systeme?
\#0:13:21\#

Experte 2:
Ja also, da war ich mir jetzt nicht ganz sicher. Ich habe ja vorhin deinen Leitfaden durchgelesen, ob wir das gleiche Verständnis haben. Also wenn es um Legacy-Systeme geht, dann bezieht sich das aus meiner Sicht im Wesentlichen eben auf die Backend-Systeme – ok. Und wenn wir jetzt über eine Technologie, wie die BTP beispielsweise, sprechen, dann ist ja die Ambition der BTP nicht notwendigerweise ein Backend-System zu ersetzen und dafür dann eine Vertragsverwaltung anzubieten. Du hast ja gesagt, du redest jetzt hier nicht von transaktionellen Systemen, sondern das ist ja eigentlich eher etwas, was komplementär zu verstehen ist – also ergänzend. So um dann beispielsweise zusätzliche Services und Integrationsmöglichkeiten eben anzubieten. Das heißt aber für mich in letzter Konsequenz, dass ich an ein Legacy-System, wenn man es denn so nach wie vor bezeichnet- ich würde es nicht Legacy System bezeichnen, sondern als existierende Systemland¬schaft oder bestehendes Backendsystem - dann ergeben sich natürlich an diese Systeme auch neue Anforderungen, die vielleicht bisher so gar nicht „auf dem Radar“ waren. Das heißt also, beispielsweise transaktionelle Funktion, eben im Sinne von Funktionsbausteinen, Services, API´s - wie auch immer - eben nach außen kehren zu können, also auch das Legacy-Systeme eben integrativer werden, in dem Sinne. Und Technologieplattformen in dem Sinne dann eben komplementär eingesetzt werden, um API-Management beispielsweise zu ermöglichen, um sich einfacher in Ökosysteme eben zu integrieren und auch besser die Time-to-Market zu unterstützen. Ein anderer Aspekt, den ich im Prinzip durch eine technische Plattform erfüllt sehe, ist, dass sie mir Zugang zu neuen Technologien bietet. Das heißt, also meine Prozesse erweitert um Bestandteile oder Inhalte, die ich so beispielsweise gar nicht notwendigerweise in meinem System abbilden kann oder will. Als Versicherer muss man ja auch „Underwriting“, also Risiko- und Antragsprüfung, betreiben – ok. Das wird auch im Kfz-Bereich bis zu einem gewissen Grad gemacht. So jetzt gibt es ja solche tollen Sachen wie AI oder künstliche Intelligenz, also irgendwelche lernenden Modelle, die mithilfe einer technischen Plattform implementiert werden und kann die als Service auch entsprechend mir zu Nutze machen. Das hilft mir dann ja auch meine vertikale Integration zu optimieren und zu verbessern. Das heißt, also auch meine Wertschöpfungskette zu optimieren. Genau und wenn ich mir jetzt das Ganze noch einmal aus dem Blickwinkel des Software-Vendors betrachte, wie beispielsweise SAP – der einen globalen Anspruch hat, da ist vielleicht noch eine andere Dimension, die du nicht notwendigerweise auf dem Radar hast. Die technische Plattform eröffnet mir natürlich auch die Möglichkeit, meinen „Core clean zu halten“, das heißt also eine Codeline mehr oder weniger für allem Kunden zur Verfügung zu stellen. Alles, was irgendwie länderspezifisches, regionsspezifisches, spartenspezifisch - was also so ein bisschen normalerweise mehr Komplexität und Variationen im Kern darstellen würde - eben auszulagern. Das ist natürlich dann auch für einen Software-Vendor sehr interessant, was das Thema Skalierbarkeit angeht, was auch die Möglichkeit angeht, eben die Lösung, beispielsweise in der Cloud konsumierbar oder ähnliches zu machen.
\#0:18:06\#

Interviewer:
Ok - mit Blick auf die Zeit möchte ich jetzt gerne die Anforderungen der Kfz-Versicherer an technische Plattformen vorstellen, die ich in der Literarturrecherche gefunden habe. Dabei habe ich diese in verschiedene Cluster eingeteilt und ich würde dich dann bitten zu den einzelnen Anforderungen kurz Bezug zu nehmen – ob du das bestätigen kannst und ob du in dem jeweiligen Cluster noch weitere Anforderungen siehst. Um danach in einem zweiten Step diese Anforderungen mit dir aus der Sicht der Kfz-Versicherer zu priorisieren. Dann teile ich jetzt einmal meinen Bildschirm – einen kleinen Moment. So - jetzt müsstest du etwas sehen können.
\#0:18:50\#

Experte 2:
Ja – das Bild ist da.
\#0:18:52\#

Interviewer:
Genau - der erste Cluster-Bereich hier ist eben der Bereich „Integration“, bei dem als Anforderungspunkte – gemäß Recherche - zu nennen sind: „Unterstützen einer serviceorientierten Architektur“, die „Anbindung von Legacy-System“ zum Unterstützen von offenen Schnittstellenstandards – du hast auch gerade schon das Schlagwort „digitale Ökosysteme“ angesprochen, was auch so ein Trendthema ist. Es gibt es ja auch schon erste Initiativen, die - ähnlich wie im Banking mit PSD2 - versuchen, auch so Schnittstellenstandards zu schaffen für einen offenen Datenaustausch. Im Kfz-Bereich gibt es zum Beispiel die Car-Claims-API, die auch auf dem REST-Standard basiert und eben das verfügen über API-Managements-Tools um eben auch die API-Aufruf konfigurieren und verwalten zu können. Daher die Frage - würdest du bei diesen Anforderungen so mitgehen können? Kannst du die bestätigen? Siehst du vielleicht in diesem Bereich noch weitere Anforderungen, die von mir nicht genannt wurden?
\#0:20:02\#

Experte 2:
Ja würde ich so mitgehen, ich würde dir vielleicht empfehlen noch ein paar „Business-getriebene Buzzwords“ mitaufzunehmen. Ein Stichwort wäre da beispielsweise das Thema „Embedded Insurance“, weil das ist dann beispielsweise eine mögliche Repräsentanz des Themas „Integration in bestehende Ökosysteme“ - ja. Also beispielsweise ein Klassiker im Thema „Embedded Insurance“, auch wenn das relativ trivial erscheint. Aber es erzählt letztendlich die Story „Wenn du auf booking.com bis und dort eine Reise buchst, dann kannst natürlich auch noch eine entsprechende Reiseversicherung oder Ausfallversicherung oder Ähnliches mit abschließen - ok.
\#0:20:31\#

Interviewer:
Ok - das wäre dann auch so ein bisschen der Punkt „Offene Schnittstellen-Standards“ oder wo ich dann sagen würde, die Plattform muss mit anderen Partnern kompatibel sein, um eben da auch integriert werden zu können.
\#0:20:48\#

Experte 2:
Ja - also wenn du da irgendetwas schreibst, nur ein Buzzword, was dann in dem Zusammenhang auch ganz wichtig ist, ist das Embedded Insurance – ok. Das andere Thema, was du vielleicht auch in dem Zusammenhang noch einmal auf die Agenda nehmen solltest, ist das Stichwort „Open Insurance“. Die SAP, als auch die jetzt in der Folge die SAP Fioneer – wir sind in einem Verein aktiv, der nennt sich „FRIDA - das ist die „Free Insurance Data Association“.
\#0:21:12\#

Interviewer:
Ja, genau – das ist mir bekannt.
\#0:21:15\#

Experte 2:
Das steht für „Free Insurance Data Association“ und ich bin da, mehr oder minder, in dem Verein tätig, beziehungsweise bin einer der Mitgliedsvertreter aus der SAP Fioneer-Sicht heraus und das ist natürlich dann auch ein Thema, was in dem Zusammenhang sehr gut passen würde ja.
\#0:21:32\#

Interviewer:
Genau - ich habe auch bei meiner Arbeit überlegt, wie kann ich denn aus diesem Ökosystem-Gedanken gerade auch mit FRIDA eine Anforderung an eine Plattform formulieren? Und die Überlegung wäre da zu sagen: FRIDA versucht, die einzelnen Bereiche - auch im Mobility-Bereich mit der Car-Claims-API eben offene Schnittstellen-Standards zu schaffen und der basiert eben auf REST. Das bedeutet, dass dann die Plattform dahingehend auch offene Schnittstellen-Standards über REST unterstützen können muss. Das wäre dann da aus meiner Sicht die entsprechende Herangehensweise.
\#0:22:04\#

Experte 2:
Ja, genau und FREIDA ist im Prinzip die deutsche Variante von Open-Insurance, aber generell ist das bald ein Thema, was auch auf europäischer Ebene betrieben wird, also nicht ausschließlich als deutsches Thema betrieben, wird. 
\#0:22:23\#

Interviewer:
Der nächste Clusterbereich, den ich identifiziert habe, ist die „Entwicklung“ mit folgenden verschiedenen Anforderungspunkten: Das sind „Entwicklung und Betrieb neuer Anwendungen“ - da auch der Support von mobilen Anwendungen, die „Erweiterung von bestehenden Anwendungen“ Wenn wir das jetzt gerade bei Kfz-Versicherern anschauen, ist das da so, dass ja die auch gar nicht so viele Entwicklungsressourcen haben. Deswegen auch der nächste Punkt „Low-Code- und No-Code-Werkzeuge“, die für die Kfz-Versicherer relevant sein könnten und eben der Anforderungspunkt „Entwicklung von IoT-Anwendungen“ - gerade auch hier im Kontext von Telematik-Tarifen zum Beispiel, dass eben da auch die Sensordaten in die Geschäftsprozesse integriert werden können.
\#0:23:09\#

Experte 2:
Ja, ich weiß jetzt nicht, ob das in der Folge noch bei deinen Clustern kommt, aber ich glaube das passt irgendwie auch zu diesem Thema „Entwicklung - Stichwort Low-Code/ No-Code“. Das geht natürlich auch einher mit dem „Enablement“ von Betriebspartnern oder Repräsentanten aus einem Ökosystem heraus. Das heißt also, wenn ich das Thema Versicherung „konsumierbar“ mache, im Sinne einer Konfiguration zur Laufzeit – ja? Dann kann beispielsweise auch ein Produktmanager von booking.com oder wer auch immer, das „enablen“, dass er eben auch Versicherungs¬pakete, beispielsweise „zusammenstrickt“ oder zusammenbaut. Am Ende ist das dann eigentlich nichts anderes, als wenn er - ich sag einmal - in einem Webshop irgendetwas zusammen¬bauen würde. Du verstehst, worauf ich hinauswill? 
\#0:24:16\#

Interviewer:
Da bin ich mir jetzt nicht sicher - Das wären dann Anforderungen auch im Bereich der Integration? Wo würde ich die dann zuordnen?
\#0:24:21\#

Experte 2:
Nein, das ist keine Anforderungen im Bereich der Integration. Das ist glaube ich etwas, das es auch darum geht eben mein Ökosystem zu „enablen“. Also nochmal: „Ich bin Produktmanager - beispielsweise bei booking.com. Ich habe von Versicherungen so wenig Ahnung „wie eine Kuh vom Tag“. Um dann möglichst verschiede Versicherungsangebote auf booking.com zu platzieren, wäre es sehr hilfreich, wenn mir der Versicherer vorgefertigte Bausteine zur Verfügung stellt, mit die ich in meine Proposition einbauen kann. 
\#0:25:05\#

Interviewer:
Okay, das bedeutet anforderungsseitig einfach einzelne Insurance-Pakete bereitzustellen, dann kann der Produktmanager auswählen - „Was will ich eigentlich gerade einbinden?“ – Habe ich das jetzt richtig verstanden?
\#0:25:13\#

Experte 2:
Ganz genau richtig - und da kann eine technologische Plattform ein Werkzeug oder ein Tool sein, dass eben dieses „Enablement“ ermöglicht – im Sinne von Low-Code/ No-Code.
\#0:25:25\#

Interviewer:
Ok - dann zum nächsten Cluster-Bereich „Datenverarbeitung“ mit den Anforderungspunkten „Spartenübergreifende Sicht auf Daten“- gerade auch um ein besseres Kundenverhältnis zu bekommen, das „Empfangen von Sensordaten von physischen Objekten“ - wenn wir jetzt das ganze hier wieder Kontext von IoT-Anwendungen betrachten und dann eben zum einen die „Analyse großer Datenmengen“ und zum anderen eben die ETL-Lösungen zur Aufbereitung von unstrukturierten Daten. Daher an dich die Frage: Stimmst du den Anforderungen zu? Siehst du in dem Bereich noch weitere Anforderungspunkte, die noch nicht benannt wurden?
\#0:25:57\#

Experte 2:
Nein - ich denke vieles kann man wahrscheinlich an diesem bestehenden Cluster subsumieren. Das passt schon - das ist ok.
\#0:26:06\#

Interviewer:
Ok - dann hätten wir noch 2 weitere Cluster. Einer wäre der Bereich „Prozessautomatisierung“, zum einen mittels RPA - da ist es ja auch so, dass bei Kfz-Versicherern viele repetitive Prozesse bestehen, die aber dennoch von Mitarbeitern aktuellen noch manuell bearbeitet werden. Da könnte eben RPA helfen. Und im Anforderungspunkt „Automatisierung von Kundeninteraktionen“ ist es eben auch mit Chatbots dann möglich, dass man Kundenanfragen, wie Versicherungspolicen oder Schadens¬meldungen, jederzeit beantworten kann und eben keine Wartezeiten hat, da man auf den Chatbot eben nicht warten muss.
\#0:26:45\#

Experte 2:
Hm - da vermisse ich jetzt beispielsweise das Thema Entscheidungsfindung.
\#0:26:50\#

Interviewer:
Das Thema „Entscheidungsfindung“? - OK.
\#0:26:52\#

Experte 2:
Ja - Ich meine früher sind die Sachen im wesentlichen Sinne regelbasiert abgelaufen. Das heißt, ich habe irgendwelche Entscheidungsbäume oder ähnliches implementiert und wenn bestimmte Bedingungen erfüllt worden sind, dann hab ich als Kunde gesagt „Okay, das akzeptiere ich“ oder „Ich akzeptiere das nicht“ oder „Ich möchte beispielsweise zusätzliche Informationen haben“. Bei dem Thema Entscheidungsfindung hilft mir jetzt beispielsweise eine technologische Plattform Zugang zu AI zu bekommen. Und AI ist nicht notwendigerweise gleich RPA.
\#0:27:29\#

Interviewer:
Das heißt, dann wäre da die weitere Anforderung, dass die Plattformen auch sicherer bei der Entscheidungsfindung unterstützen können?
\#0:27:37\#

Experte 2:
Ganz genau – so ist es.
\#0:27:39\#

Interviewer:
Das gehört ja ein Stück weit dann auch wieder in den Bereich der Datenanalyse, wenn ich sage, das sind die Daten, die ich habe und entsprechend aufbereitet habe, um dann eben darauf basierend mithilfe von Graphen - als Beispiel - eine bessere Entscheidung treffen zu können.
\#0:27:54\#

Experte 2:
Hm - da reden wir dann aber - glaube ich - von zwei unterschiedlichen Dimensionen. Das was du gerade beschreibst hat eher so für mich den Reporting-Charakter - ich nenne es jetzt einfach mal so. Ich rede aber beispielsweise von Entscheidungsfindungen, die sofort stattfinden. Das heißt also, ich bin beispielsweise in einem Prozess, sei es jetzt in einen Akquise-Prozess oder in einem Schaden¬prozess und möchte bestimmte Sachverhalte prüfen lassen. Das kann ich regelbasiert machen oder beispielsweise mithilfe von KI-Modellen.
\#0:28:28\#

Interviewer:
Ok – das habe ich verstanden. Dann hätten wir noch , bevor wir zu der Priorisierung der Anforderungen gehen, noch den letzten Cluster-Bereich „Weitere Anforderungen“ mit dem Anforderungspunkt „Datensicherheit“, das glaube ich ist so eine ganz wichtige grundlegende Anforderung mit den beiden ISO-Normen, die sich auf die Informationssicherheits-Management-systeme, als auch auf die DSVGO beziehen und eben der Punkt „vertikal und horizontale Skalier-barkeit“, um eben auch mit den hohen Anfragen bei Kfz-Versicherern, beispielsweise wie sie im Herbst aufgrund der vielen Tarif-Wechsel-Kündigungen stattfinden, klar zu kommen. Stimmst du meinen Anforderungen hier zu?
\#0:29:29\#

Experte 2:
Ja, der Bereich Datensicherheit ist für Versicherungen allgemein besonders wichtig.
\#0:29:41\#

Interviewer:
Du hattest in weiteren Punkten dann auch noch das Thema Open-Insurance genannt gehabt - mit FRIDA. Passt das für dich, wenn ich das jetzt in den Punkt „offene Schnittstellenstandards“ mitlaufen lasse, oder würdest du das als extra Punkt sehen, weil die aktuellen FRIDA-Schnittstellen ja auch REST-Schnittstellen sind? 
\#0:30:05\#

Experte 2:
Ja deswegen, wenn du den Punkt etwas breiter definierst und das Stichwort Open-Insurance nimmst und sich nicht nur auf FRIDA reduzierst, dann würde ich es als eigenen Topic aufnehmen.
\#0:30:17\#

Interviewer:
Ok - also Open-Insurance generell als Anforderung.
\#0:30:21\#

Experte 2:
Ja, genau – so ist das gemeint.
\#0:30:23\#

Interviewer:
Du hattest du noch genannt „Enablement“ von Betriebspartnern. In welchen Bereichen würdest du das sehen?
\#0:30:36\#

Experte 2:
Genau, ich hatte noch den Punkt „Enablement von Partnern genannt“, zähle den aber zur Anforderung „Low-Code und No-Code Tools“ mit dazu.
\#0:30:49\#

Interviewer:
Alles klar. Das waren die Punkte, die du genannt hattest und hier vielleicht noch der Anforderungspunkt „Entscheidungsfindung mit KI“ – der ist auch ein ganz wichtiger gewesen. So dann würde ich jetzt noch einmal zum Ende unseres Interviews kurz mit dir die Punkte priorisieren wollen. Und zwar wäre es jetzt hier die Idee, wir haben jetzt hier die einzelnen Anforderungen, die eine Plattform erfüllen sollte und dann die Bitte an dich die Wichtigkeit der einzelnen Punkte von 1 bis 3 zu priorisieren und wo die eben den Kfz-Versicherern vor allen Dingen den nächsten 3 bis 5 Jahren helfen können und vielleicht bei jedem kurz mit 1-2 Sätzen kurz beschreiben, warum du den Punkt der jeweiligen Priorität zuordnest -von hoch bis niedrig.
\#0:31:54\#

Experte 2:
Also ich würde jetzt nicht die einzelnen Anforderungspunkte, sondern ich würde die Cluster-Bereiche, die du gewählt hast, zuordnen. Für mich hat das Thema „Integration“, also Integrationsfähigkeit im Moment die höchste Bedeutung.
\#0:32:12\#

Interviewer:
Ok - das heißt also, du würdest alle Anforderungspunkte, die wir hier im Bereich Integration gesehen haben, alle mit hoher Priorität einordnen. Gibt es da einen Punkt, der besonders wichtig ist oder einen, der vielleicht ein bisschen weniger wichtig ist? Oder sind die für dich alle gleich wichtig?
\#0:32:29\#

Experte 2:
Die Anforderungspunkte sind für mich eigentlich weitestgehend ebenwürdig - ich sagen es einmal so - ja.
\#0:32:37\#

Interviewer:
Dann haben wir noch den Cluster-Bereich der Entwicklungen. Wie würdest du die einzelnen Anforderungen in dem Bereich einsortieren?
\#0:32:54\#

Experte 2:
Hm - ich glaube, das Thema IoT ist schon hoch zu priorisieren - dann breche ich jetzt doch mein Mantra, dass ich da gesagt habe „Wir machen es anhand kompletter Cluster“. Aber das würde ich beispielsweise mit Hoch versehen und mit Niedrig eher das Thema „Low-Code/ No-Code“.
\#0:33:20\#

Interviewer:
Ok - das heißt also, du würdest sagen, dass das Thema „Low-Code/ No-Code“ aktuell Versicherungen nicht so wichtig ist, einfach weil dahingehend der Bedarf nicht da ist, oder?
\#0:33:32\#

Experte 2:
Nein - weil das auch mit einer entsprechenden Komplexitätsreduktion einhergeht, weil wenn „Low-Code/ No-Code“ angedacht ist, muss das ja schon jemand mehr oder minder vorgedacht haben sagen wir es mal so. Und ich glaube, soweit sind wir hier im Moment eigentlich nicht - also im Moment sehe ich das Thema Integration sowie die Einbindung und Ausnutzung von bestehenden Daten im Sinne von IoT als wichtiger an, als das Thema „Low-Code/ No-Code“.
\#0:34:05\#

Interviewer:
Ok - danach haben wir noch den Punkt „Erweiterung bestehender Anwendungen“ sowie „Entwicklung und Betrieb von neuen Anwendungen“. Wie würdest du die beiden denn noch priorisieren im Bereich der Entwicklung?
\#0:34:16\#

Experte 2:
Da würde ich sagen, setze die beiden Punkte in den mittleren Bereich.
\#0:34:20\#

Interviewer:
Ok – dann würde ich die beiden Anforderungen einmal dahin schieben. Jetzt hätten wir den Cluster-Bereich Datenverarbeitung - dort mit vier ganz spannenden Anforderungen, unter anderem auch wieder IoT mit den Sensordaten. Wie würdest du hier die einzelnen Bereiche zuordnen?
\#0:34:37\#

Experte 2:
Hm – auch hier eher Mittel. Da kannst du das komplette Cluster eigentlich in Mittel einstufen - Ja.
\#0:34:44\#

Interviewer:
Weshalb? Weil du sagst, der Bereich Datenanalyse ist hier doch gar nicht so extrem ausschlaggebend oder wie kommt es hier zu dem Entschluss?
\#0:34:56\#

Experte 2:
Weil ich immer davon ausgehe, dass man erst einmal die „Basisarbeit“ leisten muss. Also für mich ist da schon logische Reihenfolge drin - ja. Und das ist für mich der nächste logische Schritt und nicht notwendigerweise einer der ersten Schritte, die ich gehen würde - ok.
\#0:35:15\#

Interviewer:
Sprich erster Bereich Integration und wenn die Daten integriert sind, danach kann ich die Daten aufbereiten, wenn sie aufbereitet sind, dann kann ich die Daten noch analysieren.
\#0:35:25\#

Experte 2:
Ganz genau – so ist es zu verstehen. Ja.
\#0:35:28\#

Interviewer:
Dann hätten wir jetzt noch den Bereich „Prozessautomatisierung“ und den Bereich „RPA/ Chatbots und Entscheidungsfindung“, die du eben noch angesprochen hast - wie würdest du die zwei hier einsortieren?
\#0:35:41\#

Experte 2:
Ja, ich denke auch im unteren Bereich – und zwar beide. Genauso.
\#0:35:55\#

Interviewer:
Ok – das heißt auch da wieder mit dem Blick, dass zuerst einmal die Basics gemacht werden sollten, wie eben die Integration und danach erst - sag ich mal auch so – die Fancy-Themen, wie Chatbots und RPA anbieten können.
\#0:36:01\#

Experte 2:
Genau – so in diesem Sinne.
\#0:36:03\#

Interviewer:
Ok – dann noch die letzten beiden Anforderungen - Genau.
\#0:36:05\#

Experte 2:
Das letzte Cluster ist, ja so ein bisschen Compliance Thema und das kriegen die Versicherer ja eigentlich von außen aufgezwungen und das hat damit auch eine hohe Priorität. Wir haben ja in Deutschland diese Datensicherheit und Privatheit der Daten – ich sage einmal – Daten-Fetischismus bis zu einem gewissen Grad. Das ist in anderen Ländern vielleicht nicht so ausgeprägt. Die sind da etwas offener und etwas flexibler. Aber das ist sicherlich dann mit hoher Priorität zu versehen, weil wenn du nicht „compliant“ bist, dann kannst du „den Laden zu machen“.
\#0:36:48\#

Interviewer:
Ok - und dann noch den Punkt „Vertikal und horizontal skalierbar“ - wie würdest du das hier entsprechend einsortieren?
\#0:36:57\#

Experte 2:
Skalierbarkeit spielt natürlich eine Rolle, wenn wir über das Thema Cloud sprechen. Insofern auch eines der Top-Themen im Moment aus meiner Sicht.
\#0:37:04\#

Interviewer:
Ok - dann haben wir hier jetzt noch einmal die finale Priorisierung, da kannst du jetzt bitte noch einmal darauf schauen, ob das für dich so passend eingeordnet ist.
\#0:37:23\#

Experte 2:
Ja, das passt. Wichtig ist aber vor allem das „Rational“ dabei, dass wir eben erstmal bestimmte Kernaktivitäten durchführen müssen, unsere Systemlandschaft aufräumen und aufbauen müssen und es gibt bestimmte Muss-Themen, wie die regulatorischen Themen, um die wir nicht herum-kommen und erst danach können wir uns im nächsten Schritt der „Kür“ widmen. 
\#0:37:47\#

Interviewer:
Genau - das würde ja auch ganz gut zu der Frage passen „Wo kann die Plattform – vor allen Dingen in den nächsten 3 bis 5 Jahren verwendet werden?“ und das wäre genau diese Themen – erst einmal die Integration auf die Reihe bekommen, um danach eben auch die anderen Themen angehen zu können.
\#0:38:05\#

Experte 2:
Genau - alles klar.
\#0:38:07\#

Interviewer:
Wunderbar - dann sind wir jetzt am Ende des Interviews angekommen. Ich habe noch eine abschließende Frage: Gibt es noch etwas, dass du zum Thema „Digitale Plattformen als Wettbewerbsfaktor für den deutschen Kfz- Versicherungsmarkt“ hinzufügen möchtest, über das wir noch nicht gesprochen haben?
\#0:38:25\#

Experte 2:
Alles gut, alles passend - völlig ok.
\#0:38:29\#

Interviewer:
Wunderbar - Vielen Dank, dass du dir die Zeit genommen hast, mit mir ein Experteninterview zu führen und meinen Fragen Rede und Antwort zu stehen. Dann würde ich jetzt hier an dieser Stelle das Recording beenden und mich bei dir für deine Zeit und mühen bedanken.
\#0:38:45\#



\newpage
\subsection{Experteninterview 3 - Eduard Schmidt (SAP SE)}

\textit{Durchgeführt am 20.04.2023, 11:02 Uhr -- 11:37 Uhr:}

Interviewer/ Experte 3:
Teams-Programm-Test (Transkription)
\#0:2:01\#

Interviewer:
Ich möchte das Experten-Interview aufzeichnen und für meine Projektarbeit transkribieren. Daher zu Beginn zunächst die Frage – Ist es für Dich in Ordnung in der Projektarbeit namentlich benannt zu werden oder willst Du anonym bleiben?
\#0:2:11\#

Experte 3:
Nein, das ist für mich kein Problem, Du kannst mich gerne namentlich nennen.
\#0:2:17\#

Interviewer:
Alles klar - Ich führe das Interview im Rahmen meiner Projektarbeit 2 durch, welche den Titel „Digitale Plattformen als Wettbewerbsfaktor für den deutschen Kfz-Versicherungsmarkt am Beispiel der SAP Business 90 Plattform“ trägt und Ziel dieses Interviews ist es eben zuerst die Anforderungen der Kfz-Versicherer an digitale Plattformen zu identifizieren und danach zu priorisieren. Zunächst noch ein kurzer, einleitender Hinweis zur Definition des Begriffs „Digitale Plattform“. Im Rahmen meiner Arbeit sind mit dem Begriff „digitale Plattformen“ technische Plattformen gemeint, die Funktionalitäten, Technologien und Services bereitstellen, wie beispielsweise die SAP BTP - nicht aber eben transaktionsorientierte Plattformen, wie beispielsweise Uber. Soviel zuerst einmal von meiner Seite zum Begriff „Digitale Plattform“.
\#0:3:07\#

Experte 3:
Ok – das habe ich soweit verstanden.
\#0:3:09\#

Interviewer:
Gut, dann können wir also beginnen. Ich werde Dir heute im Verlauf dieses Experteninterviews circa 8 Fragen stellen, zu Beginn zu Deiner Person, wo Du Dich kurz vorstellen kannst, dann im Hauptteil der Fragenblock um die Anforderungen an „Digitale Plattformen“ zu identifizieren. Danach werde ich selbst einen PowerPoint-Slide auflegen mit der ich Dir die Anforderungen zeige, die ich gefunden habe. Diese mit Dir kurz besprechen und danach mit Dir priorisieren. Am Ende des Interviews noch eine Frage zum Ausblick zum Ausblick. So - in Anbetracht der Zeit, wenn Du keine Fragen vorab hast, würde ich jetzt mit der ersten Frage beginnen.
\#0:4:11\#

Experte 3:
Soweit keine Fragen. 
\#0:4:14\#

Interviewer:
Wunderbar, nun kurz zu Deiner Person - welche Erfahrungen hast Du in der Versicherungsbranche und wie sieht Dein bisheriger Werdegang aus? Was ist Deine aktuelle Position im Unternehmen? Beschreibe das Bitte in Wenn Du das bitte in eins bis zwei Sätzen beschreiben könntest.
\#0:4:38\#

Experte 3:
Ich komme ursprünglich von einem Versicherungsunternehmen, der auch Kraftfahrtversicherung macht, also ich komme ursprünglich von der HDI, habe dort 6 Jahre in verschiedenen Funktionen gearbeitet. Ich habe dort Portale entwickelt für Kunden, für Vermittler, auch an Bestandssystemen selbst Migrationen durchgeführt, Prozesse entwickelt und integriert. Danach bin ich dann zur SAP gekommen. Bei der SAP selbst war ich am Anfang für das Policy-Management-System eingeteilt, habe dort dann sehr viele Implementierungsprojekte gemacht - auch sehr stark im Ausland, sehr viele Kfz-Projekte, bin dann aber sehr schnell in Richtung Systemarchitektur gegangen. Letztlich habe ich dann im Endeffekt das Thema für Versicherungen gesamtheitlich abgebildet.
\#0:5:36\#

Interviewer:
Ok – damit hast Du sowohl auf der Versicherungsseite, als auch auf der IT-Seite sehr viel Erfahrung. Dann noch eine weitere Frage: Welche Entwicklungen bzw. Herausforderungen sind in der Kfz-Versicherungsbranche aktuell erkennbar und wie wirkt sich das auf die IT-Systeme in den Unternehmen aus?
\#0:5:52\#

Experte 3:
Also ich glaub der größte Punkt für alle Kfz-Versicherer ist das Thema TCO-Reduktion oder Kosten generell, also wie kann man die Kosten reduzieren, weil der Kfz-Versicherungsmarkt war ja schon immer ein nicht sehr rentables Geschäft. Man hat das als Versicherungsunternehmen immer mit angeboten, weil die Kunden das erwartet haben und man hat damit auch immer versucht, im Endeffekt Kunden bei anderen Versicherungen abzuwerben, die sich dann im Endeffekt auch für andere Versicherungen aus dem Unternehmen entscheiden. Bei den Kfz-Versicherern selbst, wie schon gesagt, sind diese Kosten-Sparmaßnahmen im Endeffekt das höchste Gut dort. Und was die Unternehmen versuchen zu machen, ist tatsächlich die komplette Lieferkette zu digitalisieren, also schon bei der Angebotserstellung, wie die Kunden sich das Angebot einholen können beispielsweise auf Check 24 oder sonstigen Plattformen - das muss alles automatisiert gehen. Der Vertrags¬abschluss muss automatisiert sein und insbesondere gerade auch die Schaden-Verarbeitung. Also der Kunde selbst legt den Schaden digitalisiert an, er bekommt automatisch einen Werkstatttermin oder kann sich einen Werkstatttermin digital aussuchen, die ganzen Information zur Schadens¬regulierung werden weitergeleitet an eine dementsprechende Werkstatt - also das muss alles komplett digitalisiert ablaufen. Da sollte im Endeffekt kein Mensch mehr darauf arbeiten und das ist eine wesentliche Kostenmaßnahme und der zweite Schritt ist natürlich, dass man versucht, so stark wie möglich zu standardisieren und dann auch die Thematik Auslagerung in die Cloud. Also gerade für Kfz-Versicherer ist eigentlich diese ganze Thematik „Betrieb – standardisierte Prozesse - Auslagerung in die Cloud“ die große Herausforderung.
\#0:7:27\#

Interviewer:
Okay – das heißt zusammengefasst die zentralen versicherungsseitigen Anforderungen sind, möglichst viel zu digitalisieren und möglichst viele Prozesse auch zu automatisieren, um eben die Kosten bestmöglich reduzieren zu können.
\#0:7:37\#

Experte 3:
Der Automatisierungsgrad sollte bei den Versicherungsunternehmen im Idealfall - also optimal - bei 100\% liegen.
\#0:7:43\#

Interviewer:
Genau wo wird denn, Deiner Einschätzung nach, die technologische Plattform bei Kfz-Versicherern in den nächsten 3 bis 5 Jahren vor allem verwendet werden?
\#0:7:53\#

Experte 3:
Ok, also im übergeordneten Sinne ist der Kunde bzw. dessen Erwartungshaltung der Schlüssel für die technologische Plattform bzw. deren Einsatzbereich. Der Faktor Kunde wird dies maßgeblich bestimmen, er entscheidet sich bewusst für eine Kfz-Versicherung, schließt das Ganze vertraglich ab. In der Regel hat der jeweilige Versicherer eine nicht so starke Kundenbindung – ohne persönlichen Kundenkontakt - das heißt das ganze Geschäft läuft stark über die digitalen Kanäle. Also Stand heute ist es ja so, ein Interessent kann sich einloggen auf so einer Vergleichsplattform und kann sich informieren, wie sehen die Preise für eine Kfz-Versicherung aus? Und meistens entscheidet man sich im Endeffekt für das günstigere Angebot. Oder Du bist schon bei einer Versicherung und schließt die Kfz-Versicherung dort ab - aber meistens läuft dieser Weg über einen aktuellen, plattformgestützten Versicherungsvergleich. Stand heute ist es ja so, auch von der Jugend her oder von den Generationen, hier geht die Allgemeinheit immer mehr in die digitalen Medien. Das heißt, auch die Versicherer müssen sich in diesen digitalen Medien immer stärker positionieren, um den poten¬tiellen Kunden zu erreichen. Also das bedeutet, dass sie dort nicht nur ihre Versicherungsprodukte, sondern auch ihre Online-Dienste für den Kunden anbieten. Sei das jetzt über einen Chatbot, sei das irgendwie über eine WhatsApp-Gruppe oder sonst was, wo der Kunde Fragen kann und sofort Online-Hilfe/-Beratung bekommen kann. Die technologische Plattform müsste es ermöglichen, dass der Kunde mit dem Versicherer über die verschiedenen Kanäle der digitalen Medien direkt kommunizieren kann, sei es Schadensmeldung, Schadenshandling bzw. weiterer Ablauf zur Schadensregulierung. Die IT-Plattform müsste – aufgrund der Erwartungshaltung des Kunden bzw. Anwenders – diese Kommunikation programmtechnisch unterstützen, weil es die heutige Generation einfach erwartet, dass es da sein muss. Ich glaube, dass eine Kfz-Versicherung, Stand heute, die digital gar nichts anbietet, also beim Kunden auch gar nicht akzeptiert wird und letztlich nicht dauerhaft am Markt bestehen kann.
\#0:9:03\#

Interviewer:
Ok – demnach wird die digitale Präsentation des jeweiligen Kfz-Versicherers im Netz, beispielsweise mittels Chatbots, unglaublich entscheidend sein.
\#0:9:10\#

Experte 3:
Aus Kundensicht muss bei dem Kfz-Versicherer eine App verfügbar sein, wo der Kunde im Endeffekt seinen Vertrag ganz bequem abschließen kann, seine Versicherung komplett selbst verwalten kann und bei Bedarf - im Schadensfall - über die App ganz bequem komplett abwickeln kann. Alle Informationen müssen dort zentral verfügbar sein, sei es wenn mein Wagen in einer Werkstatt zur Reparatur abgegeben wird oder sei es, um sich zu informieren, wie weit ist mein Schadensfall beim Versicherer schon bearbeitet – wie ist da der Sachstand. Im Endeffekt müssen da alle Informationen sofort einsehbar sein, also im Sinne der Transparenz für den Kunden.
\#0:9:40\#

Interviewer:
Ok - sprich die Technologie-Plattform muss dann auch mobile Anwendungen unterstützen können, da das aktuell in der Kundenkommunikation ein sehr wichtiger Bereich ist.
\#0:9:46\#

Experte 3:
Ganz genau – ein weiterer Grund für den Einsatzbedarf dieser Technologie-Plattform bei den Versicherern ist der Umstand, dass diese immer mehr Geschäft mit anderen Unternehmen machen - also mit anderen Versicherungsunternehmen oder anderen während der Schadensregulierung beteiligten Dritten – Werkstätten/ Schadensgutachter - im Sinne einer Kooperation. Entscheidend ist hierbei, dass da kein Bruch in der Digitalisierung besteht. Noch einmal das Beispiel Schadensfall - das heißt der Kunde bekommt über die App freie Werkstatttermine bei verschiedenen Werkstätten angezeigt, kann in der App einen freien Termin bei einer bevorzugten Werkstatt auswählen und digital bestätigen. Wenn diese Kommunikationsform nicht verfügbar wäre, dann wäre ein Bruch in dieser Digitalisierung. Das heißt, man müsste dann wieder zum Telefon greifen, die Werkstatt anrufen oder den zuständigen Schadenssachbearbeiter anrufen. Das möchte man aus Kundensicht gar nicht, also im Endeffekt muss das so smooth wie möglich durchlaufen und genau da spielen halt diese Technologieplattformen eine ganz entscheidende Rolle.
\#0:10:40\#

Interviewer:
Sprich, entscheidend ist die Integration mit Partnern ebenso wie die Unterstützung von offenen Schnittstellenstands, um entsprechend leicht mit Partnern und Drittenkommunizieren zu können. Dann noch eine letzte Frage bevor wir zu meinen Anforderungen kommen, die ich in der Literatur gefunden habe. Welche Anforderungen muss eine technische Plattform für Kfz-Versicherer unbedingt erfüllen bzw. welche Funktionalitäten und Services muss neben dem bisher bereits genannten denn noch so mitbringen?
\#0:11:09\#

Experte 3:
Ja also - ich glaube die Plattform als solches muss erst einmal sehr, sehr offen sein, also offen technologisch gesehen - also auch was Open-Source-Themen und sonstige Sachen angeht. API-getrieben, weil gerade diese Kooperation oder diese Integration von Drittanbietern, von anderen Unternehmen, jeder verwendet da irgendwie eine andere Art von Technologie. Das muss so technologie-offen wie möglich sein, dass du alles wirklich problemlos miteinander integrieren kannst. Also ich glaube das ist der größte Faktor.
\#0:11:43\#

Interviewer:
OK - sprich der Bereich Integration und Offenheit der Plattform, um eben die Partner und Kooperationen zu ermöglichen und entsprechend zu enablen? 
\#0:11:47\#

Experte 3:
Ja - genau.
\#0:11:48\#

Interviewer:
Wunderbar - dann würde ich jetzt einmal zu den Anforderungen kommen, die ich in der Literatur identifiziert habe, die habe ich dabei in 5 verschiedene Cluster eingeteilt, würde die jetzt mit Dir gleich kurz besprechen und Dich dann bitten, da kurz Bezug zu nehmen zu den Anforderungen, kannst Du die bestätigen oder siehst Du vielleicht in dem jeweiligen Cluster noch weitere Anforderungen, die so nicht genannt worden sind oder in dem Bereich für dich noch fehlen. Dann würde ich jetzt einmal gleich meinen Bildschirm teilen - einen kleinen Moment bitte. Genau ich habe jetzt hier fünf Cluster herausgearbeitet - das erste wäre der Bereich „Integration“ mit den folgenden vier Anforderungen, was die Plattform dahingehend unbedingt können sollte - Sind die Anforderungen soweit klar? Ich kann die ja mal kurz erläutern: Unterstützung einer service-orientierten Architektur, um eben schnell Applikationen hinzufügen und wieder entfernen zu können. Anbindung von Legacy System generell wichtig, da dort sehr viele wichtige Daten liegen, welche Versicherer für die Tarifierung benötigen und teilweise auch aus regulatorischen Gründen noch aufbewahrt werden müssen. Dann die Unterstützung von offenen Schnittstellenstandards, dass hattest du ja auch schon angesprochen gehabt – gerade im Bereich der Partnerkommunikation und der Integration in Digitale Ökosysteme unglaublich wichtig, um mit den Partnern kommunizieren zu können. Des Weiteren ein API-Management Tool, damit Kfz-Versicherer bei der Vielzahl an Schnittstellen nicht die Übersicht verlieren und API-Aufrufe sowohl im als auch außerhalb des Unternehmens entsprechend kontrollieren und verwalten können. Stimmst du diesen Anforderungen zu? Siehst Du noch weitere Anforderungen im Bereich der Integration?
\#0:13:12\#

Experte 3:
Also da stimme ich zu – absolut. Ich glaube ein Punkt, den man nicht vernachlässigen darf, bei den Kunden, die das dann machen, ist auch der Vorteil, wenn man so eine Plattform hat, die schon Content mitliefert, gerade was das Thema Integration angeht. Im Idealfall vordefinierte Integrations-Content-Pakete oder auch Adapter, die du im Endeffekt nutzen kannst und wirklich jegliche Art von Technologie anbinden zu können, also wenn es da schon etwas gibt auf dieser Plattform, dann ist es Gold wert, weil das reduziert natürlich den Implementierungsaufwand. Und auch die Wartung ist glaube ich, ein immenser Faktor bei solchen Plattformen. Deswegen meinte ich ja auch Cloud-basiert. Das ist für den Kunden sehr, sehr interessant, da das aus seiner Sicht extrem wartungsarm sein muss. Wenn der Kunde so eine Plattform in der Cloud hat, ist es so die wird für ihn im Endeffekt auch geupdatet bzw. geupgradet - also der Kunde hat immer neue Releases, wo er sich nicht mehr drum kümmern muss und er hat auch dann automatisch immer die neuesten Technologien zur Hand direkt, die er nutzen kann und das ist halt auch für die Kunden Gold wert.
\#0:14:22\#

Interviewer:
Ok -  das heißt dann bei den Anforderungen noch zusätzlich, das Kriterium „Wartbarkeit“ und eben „vordefinierte Adaptoren“, um eben andere Bereiche möglichst schnell anschließend zu können. Noch einmal nachgefragt - würdest Du die vordefinierten Adaptoren dann auch zum Bereich offene Schnittstellen-Standards mit dazu packen oder wäre das für Dich nochmal eine separate Anforderung. Wenn ich sage „es muss offene Schnittstellen-Standards unterstützen“ das geht ja wahrscheinlich in eine ähnliche Richtung, oder?
\#0:14:51\#

Experte 3:
Ja also - ich würde es ein bisschen breiter sehen, ich würde es auch nicht Adaptoren nennen, weil dann bist Du bei den offenen Schnittstellen – da gebe ich Dir Recht. Aber ich würde sagen eher generell Content, also Content-Pakete auch für die Integration. Weil ein Integrationspaket als Content zwar ein Adaptor sein kann, aber eben auch eine komplette, ausgeprägte End-to-End-Schnittstelle.
\#0:15:12\#

Interviewer:
Ok - also auch schon Content-Pakete mitbringen und dann haben wir der nächste Bereich. Der Bereich der Entwicklung - hier auch wieder vier Anforderungen identifiziert. Zum einen die Entwicklung und Betrieb neuer Anwendungen, da auch der Support eben von mobilen Anwendungen. Das hattest Du ja eben auch schon angesprochen gehabt. Dann die Erweiterung bestehender Anwendungen, falls man Applikationen für den eigenen Bedarf erweitern bzw. anpassen möchte und eben Low-Code- und No-Code-Werkzeuge, um eben auch Fachbereiche enablen zu können sowie eben die Entwicklung von IoT-Anwendungen, was auch gerade im Kontext von Telematik-Tarifen für die Versicherer da interessant sein könnte.
\#0:15:48\#

Experte 3:
Ja – da stimme ich auch allem zu, aber auch die Punkte, die ich bei der Integration genannt habe, würde ich hier genauso sehen, also das muss eine „Umgebung“ sein, die muss wartbar sein, leicht wartbar sein und auch hier an der Stelle -wieder - Content. Jeder Content, der mitkommt, ist wieder ein absoluter Gewinn für das Unternehmen. Also das kann jetzt auch bei der Entwicklung sein - Content jeglicher Art. User Interfaces, die schon vordefiniert sind, die du dann nur noch leichter einpassen kannst - irgendwelche Templates. Das ist halt auch Gold wert - alles das den Prozess irgendwie beschleunigt.
\#0:16:21\#

Interviewer:
Ok – das Schlüsselwort ist „Content jeglicher Art“ zur bestmöglichen Unterstützung der Entwicklung. Das nächste Cluster ist der Bereich der Datenverarbeitung. Die erste Anforderung hier ist eine spartenübergreifende Sicht auf Daten, um den Kunden noch besser kennenlernen zu können. Im Kontext von IoT auch das Empfangen von Sensordaten, von physischen Objekten - wie beispielsweise eben Daten von Autos. Des Weiteren benötigen Kfz-Versicherer eine ETL-Lösung mit der sich unstrukturierte Daten aufbereiten lassen. Und wenn die Daten dann einmal aufbereitet sind, benötigen Kfz-Versicherer eben auch entsprechende Lösungen zur Analyse großer Datenmengen, um beispielsweise die Prämien- und Risikokalkulation zu optimieren. Stimmst du den Anforderungen soweit zu oder fehlt dir hier im Bereich der Datenverarbeitung noch eine Anforderung?
\#0:16:57\#

Experte 3:
Ja – da stimme ich auf jeden Fall zu – das passt so weit.
\#0:17:02\#

Interviewer:
Wunderbar – dann noch ein wichtiger Bereich, den ich eben schon angesprochen habe, den Bereich der Prozessautomatisierung mittels RPA, um eben repetitive Tätigkeiten, die aktuellen nur von Mitarbeitern manuell ausgeführt werden, eben zu automatisieren. Eine weitere Anforderung im Bereich der Prozessautomatisierung die Automatisierung der Kundeninteraktionen mithilfe von Chatbots, um Kundenanfragen möglichst schnell und zeitunabhängig beantworten zu können.
\#0:17:31\#

Experte 3:
Ja - also wichtig - vielleicht an der Stelle - Chatbots und RPA definitiv, aber was auch noch dazu kommt, sind auch so Sachen wie zum Beispiel Workflow oder generell Prozessorchestrierung. Weil du hast ja oft Prozesse, die laufen nicht nur auf einer Plattform, sondern auf verschiedenen Systemen, da muss etwas durchgeführt werden und dann helfen dir natürlich immer solche Tools, wie ein Prozessorchestrierungs-Tool, womit du im Endeffekt den Prozess end-to-end designen kannst und dann werden immer im Endeffekt spezielle Teile auf der jeweiligen Lösung einfach angesteuert. Aber du hast einen Prozess, der übergreifend end-to-end im Endeffekt überprüft, dass es abgearbeitet wird. 
\#0:18:08\#

Interviewer:
Für den Bereich der Prozessorchestrierung wäre dann die Anforderung, dass ich den Prozess entsprechend in der offenen Plattform auch modellieren kann oder wie habe ich das zu verstehen?
\#0:18:16\#

Experte 3:
Korrekt - modellieren und auch monitoren. Also das ich wirklich ein Ablaufplan erstellen kann und sagen kann „Ruf mal bitte Schnittstelle A bei System A“. Danach, wenn das Ergebnis erfolgreich zurückkommt, „Rufst du Schnittstelle B bei System B, dann mache ich etwas mit den Daten -mache Mapping, mache sonst etwas und wenn das erfolgreich war, dann führe ich folgendes durch“. Und im Fehlerfall mach ich auch etwas, also dass man das wirklich komplett das end-to-end designen kann, wie sieht dieser Prozess aus und dann später auch Monitoren kann.
\#0:18:43\#

Interviewer:
Ok – da hätten wir jetzt auch den vierten Anforderungsbereich besprochen, kommen wir nun zum fünften Bereich – den grundlegenden Anforderungen. Hier könnte man vielleicht auch noch den Punkt vordefinierter Content von Dir mitauffassen. Bisher aufgeführt sind hier zum einen die Datensicherheit, damit die Daten entsprechend geschützt werden, gerade auch im Kontext der Datenschutzgrundverordnung. Hier habe ich die beiden ISO-Normen aufgeführt mit denen sich die Anforderung überprüfen lässt. Zum anderen die vertikale und horizontale Skalierbarkeit der technischen Plattform, um beispielsweise mit den höheren Nutzerzahlen wie sie im Herbst aufgrund der Tarifwechsel auftreten – dann auch entsprechend umgehen zu können.
\#0:19:29\#

Experte 3:
Ja, das sehe ich auch genauso - das ist dann halt auch der Punkt, den ich im Endeffekt angesprochen habe mit dem Thema Cloud, Datenschutz - die ganzen Themen mit dabei und deswegen ist es halt auch so ein Riesenfaktor bei den Kfz-Versicherungen. Also all diese Sachen zu erfüllen, darum muss man sich kümmern, das heißt, das muss man selbst warten und man hat wieder IT-Entwicklung, die das ganze macht und deswegen ist natürlich dieser Schritt in Richtung Cloud für die so attraktiv, weil viele dieser Sachen werden ihnen dort abgenommen.
\#0:19:45\#

Interviewer:
Ok – da hätten wir in dem Bereich Prozessautomatisierung als Ergänzungen noch die Prozessorchestrierung, inklusive der Prozessmodellierung. Jetzt würde ich noch einmal die Punkte auf den nächsten Slide kopieren und Dich dann bitten, die Punkte entsprechend zu priorisieren. Und zwar haben wir hier 3 verschiedene Kategorien Hoch, Mittel und Niedrig und die Idee wäre es eben hier, dass Du die Anforderungen dahingehend definierst, wie wichtig die für Kfz-Versicherer - vor allem in den nächsten 3 Jahren - sind. Sprich - vielleicht ist in den nächsten Jahren der Bereich Integration aktuell noch wichtiger als zum Beispiel Datenanalyse, weil das zuerst gemacht werden muss oder aktuell da relevanter ist. Das nur einmal als Beispiel, dann würde ich Dich bitten die Sachen dahingehend zu priorisieren und im Idealfall kurz noch mit einem Satz zu begründen, warum die Anforderungen entsprechend da eingeordnet hast.
\#0:21:20\#

Experte 3:
Sehr, sehr schwer - also generell würde ich sagen, das sind die brennenden Themen für die nächsten 3 bis 5 Jahre und die sind alle sehr wichtig. Also ich hätte vielleicht - wenn wir einmal anfangen - Integration, Unterstützung einer serviceorientierten orientierten Architektur – die Anforderung ist definitiv Anforderungen mit hoher Priorität einzusortieren. Ebenso die Anbindung von Legacy Systemen. Offene Schnittstellenstandards (REST) würde ich eher mittel priorisieren.
\#0:21:52\#

Interviewer:
Ok – nachgefragt: Die offenen Schnittstellenstandards zielen auf die Integration mit Partnern ab, wäre Dich dann trotzdem weniger wichtig als jetzt die ersten beiden Punkte?
\#0:22:03\#

Experte 3:
Na ja - wenn du eine serviceorientierte Architektur hast. REST definiert ja nur erst einmal die Architektur dahinter. Was man da ja haben will, ist so eine Art „Wie sieht das Protokoll aus, wie man kommuniziert?“ und ich glaube, das eigentliche Protokoll ist an der Stelle erst einmal zweitrangig.
\#0:22:23\#

Interviewer:
Also - ich habe REST hier einfach nur als ein Beispiel in dem Kontext genannt.
\#0:22:26\#

Experte 3:
Alles gut - ich verstehe das schon, deswegen meinte ich aber auch Content ist so wichtig. Weil du wirst Systeme haben, zum Beispiel Legacy-Systeme, die werden höchstwahrscheinlich keine REST-Schnittstelle haben, die werden auch höchstwahrscheinlich noch nicht mal irgendwie eine HTTP-Schnittstelle haben, sondern das wird ein völlig anderes Format sein.
\#0:22:45\#

Interviewer:
OK, nachgefragt - da aus Deinen Erfahrungen: Weißt du, was für Schnittstellen da so bei Legacy-Systemen vorliegen, was da verwendet werden könnte?
\#0:22:52\#

Experte 3:
„File Transfer“ oder „Replication auf Datenbankebene“, weil du wirst noch auf ganz viele Versicherer treffen, die laufen mit ihrer Kfz-Lösung noch auf einem uralten Host-System. Da benötigt es erstmal nur eine Möglichkeit die Systeme anzubinden. Wie das Protokoll aussieht, ist erstmal zweitrangig. Das kann man dann im nächsten Schritt irgendwann machen. Aber wichtig ist, dass man das anbieten kann. So bei API-Management würde ich auch sagen, das ist niedrig, weil hier reden wir wirklich über interne Prozesse. Wenn man diese Services hat, dann wird man schon einen Weg finden, das anzubinden, aber man muss erstmal alle diese Schritte erfüllen und wenn man das alles hat, eine serviceorientierte Architektur und offene Schnittstellen, dann kann man wirklich zentral so ein API-Management einführen. Also so in der Priorität.
\#0:24:07\#

Interviewer:
Ah, ok – das habe ich verstanden. Dann hätten wir jetzt den nächsten Bereich Entwicklung.
\#0:24:15\#

Experte 3:
Den ersten Punkt „Entwicklung und Betrieb neuer Anwendungen“ ist natürlich hoch einzustufen. „Low-Code-/ No-Code-Werkzeuge“ würde ich auch sehr hoch ansehen. „Erweiterung bestehender Anwendungen“ braucht man so oder so, also würde ich jetzt aber in die mittlere Kategorie einstufen – das ist eigentliche Grundvoraussetzung. „Entwicklung von IoT-Anwendungen“ ist niedrig zu priorisieren.
\#0:24:45\#

Interviewer:
Ah, ok – dann sind Telematik-Tarife aus deiner Sicht aktuell gar nicht so das brennende Thema.
\#0:24:48\#

Experte 3:
Also das ist - wie soll ich es sagen - es gibt immer Trends irgendwo in der IT und das schlägt dann natürlich auch im Kfz-Bereich auf und so war das auch bei IoT-Anwendungen mit Telematik. Aber es hat sich nie wirklich durchgesetzt.
\#0:25:01\#

Interviewer:
Ok. Als nächsten Bereich hätten wir die Datenverarbeitung.
\#0:25:08\#

Experte 3:
Meine Priorisierung für „Spartenübergreifende Sicht auf Daten“ – wäre niedrig. “Empfang von Sensordaten“ würde ich auch als niedrig ansehen. Den nächsten Punkt „Aufbereitung der unstrukturierten Daten“ definitiv hoch. Auch die „Lösungen zur Analyse großer Datenmengen“ ist ein Thema mit hoher Priorität.
\#0:25:37\#

Interviewer:
Ok - dann haben wir noch den Bereich der „Prozessautomatisierung und der Prozessorchestrierung“ von vorhin.
\#0:25:42\#

Experte 3:
Genau - die Punkte 13 „Prozessautomatisierung durch RPA“ und 19 „Prozessorchestrierung“ sind hoch und der Punkt 14 „Automatisierung der Kundeninteraktionen mithilfe von Chatbots“ ist mittel einzustufen.
\#0:25:52\#

Interviewer:
Ok - dann haben wir jetzt noch den Bereich „Grundlegende Anforderungen“ mit den allgemeinen Punkten. Wie würdest du die dahingehend einordnen?
\#0:26:03\#

Experte 3:
Die würde ich eigentlich alle hoch einstufen. Ich muss ganz ehrlich sagen, also die Punkte, die wir hier gelistet haben, das ist ja Datensicherheit. Das kann man nicht niedrig einordnen - das geht gar nicht. Das ist etwas, was man erfüllen muss. Auch diese vertikale und horizontale Skalierbarkeit ist auch etwas, was man machen muss. Und wie bereits gesagt die Themen - gerade bei der Kostenreduktion – der Punkt „Wartbarkeit“, dass das halt niedrig ist, das spart Kosten. Und der Punkt „Content“ den so eine Plattform mitliefert, der reduziert im Endeffekt diesen Implementierungsaufwand - also auch beides sehr, sehr wichtig.
\#0:26:40\#

Interviewer:
Ok - kannst Du bitte noch kurz erläutern, was für dich unter den Punkt „Content“ zu verstehen ist, was für Dich da alles so dazu gehört, wenn man das Schlagwort „Content“ jetzt so hört.
\#0:26:51\#

Experte 3:
„Content“ kann je nach Spalte, die wir gerade hatten, können komplette Integrationspakete sein, Flows - also wie kann ich System A mit Systemen B kommunizieren lassen? Vielleicht gibt es dort schon einen Kommunikationsweg, der komplett implementiert ist, den man in dieser Plattform abgreifen kann. Adaptoren zu verschiedenen Anwendungen - das könnte ein Content-Paket sein. Vordefinierte Adaptoren, vordefinierte Schnittstellen zu gewissen Systemen auch das könnten Content-Pakete sein. Beim Thema Entwicklung - wie schon gesagt - Templates für UI´s, Templates für mobile Anwendung - wenn man da wirklich Templates hat, dass man nicht vom Scratch beginnen muss zu implementieren, sondern schon so eine Art Schablone hat, die man wieder verwenden kann und dann einfach nur zu seinen Bedürfnissen anpassen kann, erleichtert ein dort natürlich auch die Arbeit. Auch bei den Sensordaten, als Beispiel jetzt mal, könntest du genauso gut im Endeffekt Content haben. Man kennt ja diese Telematik-Daten. Wie sieht so ein Datenmodell dahinter aus, also wie schnell fahre ich? Wann habe ich gebremst? Im Endeffekt diese ganzen KPIs, um jetzt eine Bewertung vornehmen zu können, das kann man ja schon als Datenmodell - als Content-Paket wieder bereitstellen. Also ich muss mir das ja nicht selbst einfallen lassen und das entwickeln, sondern auch da gibt es wieder Content-Pakete. Auch generell bei den Punkten Extract - Transform – Load - auch da ein Datenmodell bereitstellen zu können. Bei analytischen Auswertungen - Reports auch da kann man natürlich Datenpakete bereitstellen. Ich meine, da sind die Kfz-Versicherer ja ziemlich gleich, also - Was habe ich für ein Bestand? Welche verschiedenen Objekte sind das also, welche Autos sind das? Wie viele Schäden habe ich also? Da gibt es ja auch gewisse KPIs - wie hoch ist meine Schadenquote? Das ist ja etwas, was jeder Versicherer macht und wenn man da in dieser Plattform im Endeffekt diese Content-Pakete dafür hätte – das wäre optimal. So auch beim Thema Automatisierung, wenn wir jetzt über Bots sprechen oder so, es gibt Standardprozesse, was so ein Bot machen kann. Auch da wieder, wenn wir schon vordefinierte Bots hätten, auch Modelle wieder bei so einer Prozessorchestrierung. Obwohl es da schwer ist, weil die Kunden immer ihre eigene Systemlandschaft haben. 
\#0:29:16\#

Interviewer:
Ok, das habe ich verstanden. Danach - mit Blick auf die Zeit - gerne zwei letzte Fragen: Noch einmal ein Blick zur „dunkelgrünen Spalte“ – Weshalb haben für Dich der Punkt „Low-Code-/ No-Code-Werkzeuge“ sowie der Punkt „Entwicklung und Betrieb neuer mobiler Anwendungen“ eine sehr hohe Priorität? Das wäre mir noch einmal wichtig, wenn Du das begründen könntest.
\#0:29:33\#

Experte 3:
Ja, gerne - wir haben ja, es ist in aller Munde, überall diesen Fachkräftemangel. Ich glaube in der IT-Branche ist es ganz hart. Für die Versicherungen ist es nochmals härter, weil die Versicherungs-branche für eine IT-Fachkraft kein lukrativer Beschäftigungsbereich ist. Ich sag mal, das ist jetzt nichts, was eine IT-Fachkraft magisch anzieht und die Leute sagen, da möchte ich gerne arbeiten. Corona hat das natürlich noch einmal verschärft. Das heißt einfach - Stand heute - Experten am Markt zu finden, also Entwickler, das ist extrem schwer. Das gilt insbesondere auch für die Versicherungsunternehmen. Da helfen - jetzt gerade - natürlich diese Low-Code-/ No-Code-Plattformen immens. Weil du, als Versicherer, einfach mehr Optionen am Markt hast, dir noch Entwickler zu holen, die dich dann bei dieser Digitalisierung im Endeffekt unterstützen können.
\#0:30:24\#

Interviewer:
Ok, das habe ich verstanden – und bitte noch einmal den anderen Punkt „Entwicklung und Betrieb neuer mobiler Anwendungen“. Das wäre mir auch noch einmal wichtig.
\#0:30:30\#

Experte 3:
Ja, und zwar, was ich eingehend, schon einmal gesagt habe. Also Stand heute wird sich meines Erachtens dieser Generationswandel hinsichtlich der Kommunikation immer mehr verstärken und die Gesellschaft kommuniziert einfach nur noch über diese mobilen Anwendungen und die Apps. Ja es ist ganz einfach auch so, dass stand heute in der Gesellschaft die Erwartung da ist, dass diese Anwendungen ein Versicherer haben muss. Er muss so eine App haben, wo ich als Kunde mich anmelden kann, wo ich die ganze Verarbeitung und Meldung vornehmen kann und wo ich auch alle meine Informationen zentral habe. Das ist eine Erwartung, die an den Versicherer gestellt ist, die er erfüllen muss und das wird sich sukzessive ausbauen.
\#0:31:10\#

Interviewer:
Ok - wenn man jetzt einmal versuchen würde, bei einem Versicherer so eine technische Plattform anzugehen - gibt es da eine Reihenfolge, in der man das logisch machen würde? Sprich das man vielleicht mit der Integration beginnt, der Datenanalyse, der Datenaufbereitung - wo würde man da aus Deiner Sicht zuerst ansetzen?
\#0:31:29\#

Experte 3:
Ich glaube, an aller erster Stelle ist diese Digitalisierungsmöglichkeit überhaupt erst einmal darzustellen. Also, dass man -als Versicherer - diese Apps hat, dass man diese Plattformen hat, wo der Kunde im Endeffekt sich anmelden kann, wo er schon mal sieht- ok, das ist überhaupt vorhanden - ich kann es machen. Das würde ich sagen, das ist der erste Punkt - das ist das Wichtigste. Danach folgt natürlich Integration. Das muss ja alles miteinander auch wirklich funktionieren. Ich glaube die Analyse der Daten, das ist so ein begleitender Prozess, weil als Versicherer brauche ich diese Informationen intern, um immer wieder Auswertungen machen zu können. Wo kann ich mich optimieren? Aber in der Reihenfolge hätte ich das jetzt gesehen.
\#0:32:11\#

Interviewer:
Das heißt, für Dich sind die Apps an erster Stelle.
\#0:32:15\#

Experte 3:
Genau - das digitale Bild nach außen muss da sein Punkt. Wenn das nicht da ist, läuft gar nichts.
\#0:32:20\#

Interviewer:
Ok, verstanden - dann wäre an zweiter Stelle, das Ganze entsprechend zu integrieren und auch nutzen zu können und dann – an dritter Stelle - anschließend so Sachen wie Datenanalyse oder auch Prozessautomatisierung entsprechend noch verstärkt einzusetzen.
\#0:32:40\#

Experte 3:
Ja, so sollte man bei einem Versicherer die Umsetzung einer solchen technologischen Plattform angehen.
\#0:32:45\#

Interviewer:
Da wir jetzt am Ende meines Interviews angelangt sind, hätte ich noch eine Abschlussfrage: Gibt es noch etwas zu dem Thema „digitale Plattformen als Wettbewerbsfaktor für den deutschen Kfz-Versicherungsmarkt“ hinzufügen möchtest, über das wir bisher noch nicht gesprochen haben?
\#0:32:58\#

Experte 3:
Nein - also ich kann im Endeffekt sagen, dass der Punkt „Content“ besonders wichtig ist, da alles, mit dem Versicherer die Implementierung vorantreiben können, ein Entscheidungsfaktor ist, wo die sagen ok, dann ist diese Plattform so attraktiv für mich, dass ich damit dann das halt im Endeffekt lösen möchte. 
\#0:33:25\#

Interviewer:
Und dieser Content zieht dann in der Abbildung durch die von mir dargestellten 4 Cluster durch, da vorgefertigte Lösungen oder Tools für Kfz-Versicherer einfach unglaublich wertvoll sind.
\#0:33:35\#

Experte 3:
Genau.
\#0:33:37\#

Interviewer:
Wunderbar, dann vielen Dank, dass Du Dir die Zeit genommen hast mit mir ein Experteninterview zu führen und meinen Fragen Rede und Antwort zu stehen. Dann würde ich jetzt das Recording stoppen und das Interview hiermit beenden. Vielen Dank noch einmal – mach es gut und auf ein Wiedersehen!
\#0:34:03\#

Experte 3:
Auf Wiedersehen.
\#0:34:06\#













\newpage

% \caption[]{}{AXA SE - Konzernstruktur (AXA Deutschland)}
%wie bekomme ich es hin dass die nicht im Abbildungsverzeichnis auftaucht?

%\subsection{Interview Leitfaden}