% !TEX root =  master.tex
\chapter{Anhang}\label{anhang}

\section{Konzernstruktur großer Versicherer}
\label{sec:KonzernStrukturen}

\begin{figure}[h]
  \centering
  \includegraphics[width=1\textwidth]{img/beteiligungsstruktur-axa-konzern.pdf}
  \caption[]{AXA SE - Konzernstruktur (AXA Deutschland)}
  \label{fig:AXAKstr}
\end{figure}
\newpage

\section{Zertifizierungen der SAP BTP Services}
\begin{figure}[h]
    \centering
    \includegraphics[width=1\textwidth]{img/BTP_Zertifikate.pdf}
    %\caption[]{Zertifizierungen der SAP BTP Services (SAP SE 2023)}
    \label{fig:BTPZertifikate}
  \end{figure}
\newpage


\section{Suchbegriffe der Literaturanalyse}
\begin{table}[ht]
    \begin{center}
    \begin{tabular}{|p{3cm}|p{3cm}|p{3,5cm}|p{3,5cm}|}
        \hline Sprache & Suchbegriff 1 & Suchbegriff 2 & Suchbegriff 3\\[0.5ex]
        \hline Deutsch & Kfz-Versicherer, Versicherer & Anforderungen & Digitale Plattformen, technische Plattformen, PaaS, APaaS, IT-Plattformen\\
        \hline Deutsch & Kfz-Versicherer, Versicherer & Herausforderungen, Probleme & -\\
        \hline Englisch & car insurance, motor insurance, insurance & requirements & digital platforms, technological platforms, PaaS, APaaS, IT-platforms\\
        \hline Englisch & car insurance, motor insurance, insurance & challenges, problems, issues & -\\
        \hline
    \end{tabular}
    \end{center}
    \caption{Suchbegriffe der Literaturanalyse}
    \label{tab:suchbegriffe}
\end{table}  

\newpage
\section{Kriterien der Literaturanalyse}
\begin{table}[ht]
    \begin{center}
    \begin{tabular}{|p{4,33cm}|p{4,33cm}|p{4,33cm}|}
        \hline  & Einschlusskriterien & Ausschlusskriterien \\[0.5ex]
        \hline Sprache & Deutsch, Englisch & Weitere Sprachen\\
        \hline Dokumententypen & Fachliteratur, internationale und nationale wissenschaftliche Zeitschriften, Studien, Magazine, Internetartikel & Weitere Veröffentlichungen\\
        \hline Publikationszeitraum & Individuelle Bewertung der Aktualität und heutigen Relevanz & Nicht mehr aktuell oder relevant\\
        \hline Verfügbarkeit & Volltext verfügbar & Volltext nicht verfügbar\\
        \hline
    \end{tabular}
    \end{center}
    \caption{Kriterien der Literaturanalyse}
    \label{tab:kriterien}
\end{table}   
\newpage

\section{Fragenkatalog der Leitfadeninterviews}
Im laufenden Gespräch wurde die Anzahl gestellter Fragen und deren Formulierung variiert. Aufnahme und Transkription fanden über die die MS-Teams-Funktion statt.

\begin{table}[H]
\begin{tabularx}{\linewidth}{lX}
\multicolumn{2}{l}{(A) Informationsphase} \\\hline \hline 
    \    & Ich führe das Interview im Rahmen meiner Projektarbeit 2 durch, welche den Titel: \enquote{Digitale Plattformen als Wettbewerbsfaktor für den deutschen Kfz-Versicherungsmarkt am Beispiel der SAP Business Technology Platform} trägt. Das Ziel dieses Interviews ist es, Anforderungen der Kfz-Versicherer an digitale Plattformen zu identifizieren und zu priorisieren. \\\hline
    \    & Dabei sind im Rahmen meiner Arbeit mit dem Begriff digitale Plattform, technologische Plattformen gemeint, welche eine Menge von Kernprodukten, -technologien oder -services bereitstellen, auf deren Basis weitere komplementäre Produkte und Services entwickelt werden können. Somit bezeichnet der Begriff digitale Plattform im Rahmen meiner Arbeit technische Plattformen wie die SAP BTP nicht aber transaktionsorientierte Plattformen wie bspw. Uber.\\\hline
    \\     
\multicolumn{2}{l}{(B) Einstiegsphase}  \\\hline \hline
    \  1 & Welche Erfahrungen hast Du in der Versicherungsbranche? Wie sieht Dein Werdegang und was ist Deine aktuelle Position im Unternehmen? \\\hline     
    \\  
\multicolumn{2}{l}{(C) Hauptphase}  \\\hline \hline
    \  2 & Welche Entwicklungen waren in der Kfz-Versicherungsbranche in den letzten Jahren erkennbar und mit welchen Herausforderungen hat die Branche derzeit zu kämpfen?  \\\hline
    \  3 & Wie wirkt sich das auf die IT-Systeme in den Versicherungsunternehmen aus, bzw. welche Anforderungen lassen sich daraus an technische Plattformen ableiten? \\\hline
    \  4 & Wofür werden Deiner Einschätzung nach technologische Plattformen bei Kfz-Versicherern in den nächsten 3-5 Jahren vor allem verwendet werden? \\\hline
    \  5 & Welche Anforderungen muss eine technische Plattform für Kfz-Versicherer daher unbedingt erfüllen bzw. welche Funktionalitäten und Services sollte eine technische Plattform aus Deiner Sicht unbedingt mitbringen, um für die Kfz-Versicherer einen Mehrwert zu schaffen?  \\\hline
    
    \end{tabularx}
    \end{table} 
    \newpage


\begin{table}[H]
\begin{tabularx}{\linewidth}{lX}
\multicolumn{2}{l}{(C) Hauptphase} \\\hline \hline
    \  6 & Eine Herausforderung der Kfz-Versicherer, von der sowohl in der Literatur als auch in der Praxis immer wieder gesprochen wird, sind die sogenannten Legacy Systeme. Welche Anforderungen ergeben sich dadurch aus Deiner Sicht an technische Plattformen für Kfz-Versicherer? \\\hline
    \  7 & Neben den bereits genannten Punkten habe ich im Rahmen meiner Arbeit Anforderungen der Kfz-Versicherer an technische Plattformen aus der Literatur herausgearbeitet und diese zu Clustern gruppiert. Diese werde ich Dir jetzt kurz vorstellen und Dich bitten Bezug zu den einzelnen Anforderungen zu nehmen: Kannst Du diese bestätigen und siehst Du in dem jeweiligen Cluster noch weitere Anforderungen die für Kfz-Versicherer bei technischen Plattformen besonders wichtig sind? \\\hline       
    \  8 & Quantitativer Frageteil: Bitte priorisiere jede Anforderung auf einer Skala von 1-3 nach ihrer Wichtigkeit (1 für niedrig, 2 für mittel, 3 für hoch) für Kfz-Versicherer in den nächsten 3-5 Jahren. \\\hline
    \\
    \multicolumn{2}{l}{(D) Schlussphase}  \\\hline \hline
    \  9 & Wir sind nun am Ende des Interviews angelangt. Gibt es noch etwas, dass Du zum Thema: \enquote{Digitale Plattformen als Wettbewerbsfaktor für den deutschen Kfz-Versicherungsmarkt} hinzufügen möchtest, über das war wir noch nicht gesprochen haben? \\\hline
\end{tabularx}
    %   \captionsetup{justification=centering}
    %   \caption{Fragenkatalog -- Semistrukturiertes Experteninterview}
    %   \label{tab:fragenkatalog}
\end{table} 

\section{Auswahl der Experten}
\improvement{Interview anonymisieren oder so verwerten???}
Bei der Auswahl der Experten wurde darauf geachtet, dass die Experten sowohl Erfahrung in der Kfz-Versicherungsbranche als auch über ein breites Fachwissen zu digitalen Plattformen verfügen. Alle Befragten haben bereits in Versicherungsunternehmen gearbeitet und arbeiten entweder als Berater oder IT-Architekt weiterhin mit deutschen Kfz-Versicherungsunternehmen zusammen.

Interviewpartner 1 ist Christos Lemonidis. Er kann auf mehr als 30 Jahre Erfahrung in der Versicherungsbranche zurückblicken und hat sich im Rahmen seines Studiums auf Versicherungsbetriebs- und -mathematik spezialisiert. Zudem ist er ein zertifizierter Aktuar. Vor seiner Tätigkeit bei SAP hat er in mehreren Versicherungsunternehmen gearbeitet und sich mit Themen wie Risikomanagement, IFRS, Tarifkalkulation und Rückversicherungsoptimierung, auch im Kontext von Kfz-Versicherungen, auseinandergesetzt. Seit über 16 Jahren ist er als Branchenexperte für Versicherungen bei SAP tätig und berät in dieser Funktionen Kunden zu den Kernversicherungslösungen der SAP.  

Interviewpartner 2 ist Alexander Grebert. Er ist der Global Head des Insurance Expert Teams bei SAP Fioneer, einem Joint Venture von Dediq und SAP, welches gegründet wurde, um das SAP-Portfolio für Finanzdienstleistungen gemeinsam auszubauen und in branchenspezifische Lösungen zu investieren. Vor seinem Wechsel zu SAP Fioneer hat Grebert als ehemaliger Versicherungskaufmann und Betriebswirt für Kfz-, Lebens- und Sachversicherer in Vertriebs-, Underwriting- und internen IT-Bereichen gearbeitet. Später war er als IT-Projektmanager für Erstversicherungslösungen bei der msg verantwortlich und dann in derselben Rolle für den gesamten Finanzdienstleistungssektor bei SAP. Zuletzt war er bei der SAP als \enquote{Director Insurance Architecture and Digital Transformation} tätig, bevor er im September 2021 zur SAP Fioneer wechselte.

Interviewpartner 3 ist Eduard Schmidt. Zu Beginn seiner Karriere arbeitete er in der IT-Abteilung der HDI in Hannover, wo er an der Umstellung des Bestands-, Partner- und Zahlungssystems beteiligt und für die Entwicklung von Kunden- und Vermittlerportalen verantwortlich war. Nach sechs Jahren wechselte er zur SAP und implementierte dort zunächst als Berater die Policymanagement-Lösung bei Versicherungsunternehmen, insbesondere bei Kfz-Versicherern. Heute berät er für die SAP Versicherungsunternehmen bei der Konzeptionierung und Gestaltung der Systemarchitektur.

\section{Transkription der Experteninterviews}

Von allen Teilnehmern wurde eine ausdrückliche Erlaubnis zur Transkription der Gespräche ausgestellt. Die Interviews wurden \enquote{vereinfacht} transkribiert. Das bedeutet, dass Dialekte, grammatikalische Fehler oder entbehrliche Formulierungen verbessert und Wortgenau ins Hochdeutsche übersetzt wurden.

\subsection{Interviewpartner 1 - Christos Lemonidis (SAP SE)}

\textit{Durchgeführt am 14.04.2023, 09:00 Uhr -- 09:40 Uhr:}
\improvement{Zeit Überprüfen}

Text kommt noch

\newpage

\subsection{Interviewpartner 2 - Alexander Grebert (SAP Fioneer)}

\textit{Durchgeführt am 14.04.2023, 10:30 Uhr -- 11:09 Uhr:}
\improvement{Zeit Überprüfen}

Text kommt noch 

\newpage{Interviewpartner 3 - Eduard Schmidt (SAP SE)}

\textit{Durchgeführt am 20.04.2023, 11:02 Uhr -- 11:37 Uhr:}
\improvement{Zeit Überprüfen}

Text kommt noch

\newpage

% \caption[]{}{AXA SE - Konzernstruktur (AXA Deutschland)}
%wie bekomme ich es hin dass die nicht im Abbildungsverzeichnis auftaucht?

%\subsection{Interview Leitfaden}