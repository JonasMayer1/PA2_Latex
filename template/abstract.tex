% !TEX root =  master.tex
%\newgeometry{top=30mm,bottom=40mm}
%\thispagestyle{empty}
\thispagestyle{plain}
\begin{minipage}{\textwidth}
        \vspace{-4.1cm}
        \noindent  \hfill \includegraphics{\imagedir/logo.jpg}
    \end{minipage}
\begin{flushright}
        \vspace{-1.7cm}
        \footnotesize{Wirtschaftsinformatik}\\
        \medskip
        \small{\textbf{\DerTitelDerArbeit}}\\
        \medskip
        \footnotesize{{\DieArtDerArbeit}arbeit}\\
        \footnotesize{\DerAutorDerArbeit}\\
        \footnotesize{\DerNameDerFirma}\\
\end{flushright}
\vspace{-0.5cm}
\section*{Kurzfassung}
%\vspace{-0.3cm}

%\textbf{Kurzfassung}\\\\

Die digitale Transformation, neuartige Technologien und steigende Kundenerwartungen setzen die Kfz-Versicherungsbranche einem starken, grundlegenden Wandel aus. So erwarten Versicherungsnehmer heutzutage einen komfortablen und reibungslosen Online-Service sowie innovative Produkte. Um wettbewerbsfähig zu bleiben, müssen Kfz-Versicherer deshalb mithilfe einer digitalen Plattform ihre Prozesse optimieren, externe Partner in ihre Wertschöpfungskette einbinden und zusätzliche Services für den Endkunden anbieten. Die SAP Business Technology Platform (BTP) ist eine dazu passende Lösung, mit der Unternehmen die digitale Transformation meistern können. Ziel der vorliegenden Abhandlung ist es daher, die Anforderungen der Kfz-Versicherer an eine digitale Plattform zu identifizieren, um anschließend beurteilen zu können, inwiefern diese Anforderungen von der SAP Business Technology Platform erfüllt werden.\\
Zur Beantwortung der Forschungsfrage wird die Task-Technology-Fit-Theorie angewendet. Dabei werden zunächst die Anforderungen der Kfz-Versicherer in einer systematischen Literaturanalyse identifiziert und anschließend durch Experteninterviews ergänzt und evaluiert. Die Priorisierung der Experten zeigt, dass bei Erfüllung der grundlegenden Anforderungen, wie Datensicherheit und Skalierbarkeit, für Kfz-Versicherer in den nächsten Jahren vor allem die Anforderungsbereiche Integration und Anwendungsentwicklung wichtig sind. Nach der Identifikation werden die Anforderungen den Technologie Charakteristika der SAP BTP gegenübergestellt. Hier ergibt die Analyse, dass die SAP BTP 19 der 20 Anforderungen grundlegend erfüllt und damit als digitale Plattform für Kfz-Versicherer gut geeignet ist. Abschließend wird in der Handlungsempfehlung eine schrittweise Vorgehensweise zur Verwendung der SAP Business Technology Plattform bei Kfz-Versicherern aus dem Analyseergebnis und der Priorisierung der Anforderungen durch die Experten abgeleitet. \\
In weiterführenden Fragestellungen kann unter anderem erörtert werden, wie sich der Trend der digitalen Ökosysteme auf die Kfz-Versicherungsbranche auswirkt und welche Rolle dabei insbesondere die digitalen Plattformen spielen.\\
Die vorliegende Arbeit ist für all jene relevant, die sich mit der digitalen Transformation der deutschen Kfz-Versicherungsbranche befassen, da die Nutzung von digitalen Plattformen ein wichtiger Faktor für den zukünftigen Geschäftserfolg ist.



%Die Kfz-Versicherungsbranche befindet sich im Zuge der digitalen Transformation in einem enormen Wandel. So erwarten die Versicherungsnehmer heutzutage einen komfortablen und reibungslosen Online-Service. Um wettbewerbsfähig zu bleiben, müssen Kfz-Versicherer deshalb mithilfe einer digitalen Plattform ihre Prozesse optimieren, „sich mit Partnern integrieren“ und zusätzliche Services für den Endkunden anbieten. Die SAP Business Technology Platform (BTP) ist eine dazu passende Lösung, mit der Unternehmen die digitale Transformation meistern können. Ziel der vorliegenden Abhandlung ist es daher, die Anforderungen der Kfz-Versicherer an eine digitale Plattform zu identifizieren, um anschließend beurteilen zu können, inwiefern diese Anforderungen von der SAP Business Technology Platform erfüllt werden.\\
%Zur Beantwortung der Forschungsfrage wurde die Task-Technology-Fit-Theorie angewendet. Dabei wurden zunächst die Anforderungen der Kfz-Versicherer durch eine systematischen Literaturanalyse identifiziert und anschließend mittels Experteninterviews ergänzt und evaluiert. Die Priorisierung der Experten hat gezeigt, dass bei Erfüllung der grundlegenden Anforderungen, wie Datensicherheit und Skalierbarkeit, für Kfz-Versicherer in den nächsten Jahren vor allem die Anforderungsbereiche Integration und Anwendungsentwicklung wichtig sind. Nach der Identifikation wurden die Anforderungen mit den Technologie-Charakteristika der SAP BTP gegenübergestellt. Hier hat die Analyse gezeigt, dass die SAP BTP 19 der 20 Anforderungen grundlegend erfüllt und damit als digitale Plattform für Kfz-Versicherer gut geeignet ist. Abschließend wurde in der Handlungsempfehlung eine schrittweise Vorgehensweise zur Verwendung der SAP Business Technology Plattform bei Kfz-Versicherern aus dem Analyseergebnis und der Priorisierung der Anforderungen durch die Experten abgeleitet.\\
%In weiterführenden Fragestellungen könnte unter anderem erörtert werden, wie sich der Trend der digitalen Ökosysteme auf die Kfz-Versicherungsbranche auswirkt und welche Rolle dabei insbesondere die in dieser Arbeit untersuchten digitalen Plattformen spielen.\\
%Die vorliegende Arbeit ist für all jene relevant, die sich mit der digitalen Transformation der deutschen Kfz-Versicherungsbranche befassen, da die Verwendung von digitalen Plattformen ein wichtiger Faktor für den zukünftigen Geschäftserfolg ist.




% Inhalte:
% -Problemstellung
% -Forschungsfrage
% -Methodik
% -Wichtige Ergebnisse
% -Interpretation
% -meist ungefähr 150 Wörter



%\restoregeometry