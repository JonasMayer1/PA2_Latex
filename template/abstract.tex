% !TEX root =  master.tex
%\newgeometry{top=30mm,bottom=40mm}
%\thispagestyle{empty}
\thispagestyle{plain}
\begin{minipage}{\textwidth}
        \vspace{-2cm}
        \noindent  \hfill \includegraphics{\imagedir/logo.jpg}
    \end{minipage}
\begin{flushright}
        \footnotesize{Wirtschaftsinformatik}\\
        \medskip
        \small{\textbf{\DerTitelDerArbeit}}\\
        \medskip
        \footnotesize{{\DieArtDerArbeit}arbeit}\\
        \footnotesize{\DerAutorDerArbeit}\\
        \footnotesize{\DerNameDerFirma}\\
\end{flushright}

\section*{Kurzfassung}

\improvement{Empfehlung Luca: Ich muss hier nicht auf das Vergleichen mit der Konkurrenz eingehen}

%\textbf{Kurzfassung}\\\\

Die Kfz-Versicherungsbranche befindet sich im Zuge der digitalen Transformation in einem enormen Wandel. So erwarten die Kunden heutzutage einen komfortablen und reibungslosen Online-Service. Um wettbewerbsfähig zu bleiben, müssen Kfz-Versicherer deshalb mithilfe einer digitalen Plattform ihre internen Prozesse optimieren, „sich mit Partnern integrieren“ und zusätzliche Services für den Endkunden anbieten. 
Ziel der vorliegenden Abhandlung ist es daher, die Anforderungen der Kfz-Versicherer an eine digitale Plattform zu identifizieren, um anschließend beurteilen zu können, inwiefern diese Anforderungen von der SAP Business Technology Platform erfüllt werden.
Zur Beantwortung der Forschungsfrage wurde die Task-Technology-Fit-Theorie angewendet. Dabei wurden zunächst die Anforderungen der Kfz-Versicherer mithilfe einer systematischen Literaturanalyse identifiziert und anschließend mithilfe von Experteninterviews ergänzt und evaluiert. Die Priorisierung der Experten hat gezeigt, dass bei Erfüllung der grundlegenden Anforderungen wie Datensicherheit und Skalierbarkeit, für Kfz-Versicherer in den nächsten 3-5 Jahren insbesondere die Anforderungsbereiche Integration und Anwendungsentwicklung wichtig sind. Nach der Identifikation wurden die Anforderungen mit den Technologie-Charakteristika der SAP BTP gegenübergestellt. Hier hat die Analyse gezeigt, dass die SAP BTP 19 der 20 Anforderungen grundlegend erfüllt und damit als digitale Plattform für Kfz-Versicherer geeignet ist. Abschließend wurde in der Handlungsempfehlung eine schrittweise Vorgehensweise zur Verwendung der SAP Business Technology Plattform bei Kfz-Versicherern aus dem Analyseergebnis und der Priorisierung der Anforderungen durch die Experten abgeleitet. In weiterführenden Fragestellungen könnte unter anderem erörtert werden, wie sich der Trend der digitalen Ökosysteme auf die Kfz-Versicherungsbranche auswirkt und welche Rolle dabei insbesondere die in dieser Arbeit untersuchten digitalen Plattformen spielen.
Die Arbeit ist damit für alle relevant, die sich mit der digitalen Transformation der deutschen Kfz-Versicherungsbranche befassen, da …..



% Inhalte:
% -Problemstellung
% -Forschungsfrage
% -Methodik
% -Wichtige Ergebnisse
% -Interpretation
% -meist ungefähr 150 Wörter



%\restoregeometry