% !TEX root =  master.tex
%\newgeometry{top=30mm,bottom=40mm}
%\thispagestyle{empty}
\thispagestyle{plain}
\begin{minipage}{\textwidth}
        \vspace{-2cm}
        \noindent  \hfill \includegraphics{\imagedir/logo.jpg}
    \end{minipage}
\begin{flushright}
        \footnotesize{Wirtschaftsinformatik}\\
        \medskip
        \small{\textbf{\DerTitelDerArbeit}}\\
        \medskip
        \footnotesize{{\DieArtDerArbeit}arbeit}\\
        \footnotesize{\DerAutorDerArbeit}\\
        \footnotesize{\DerNameDerFirma}\\
\end{flushright}

\section*{Kurzfassung}
%\textbf{Kurzfassung}\\\\

Im Rahmen dieser Ausarbeitung wird die Implementierung eines Kundenwunsches aus dem SAP Customer Connection Programm dargestellt. Dabei soll eine Suchfunktion im SAP \ac{plm} um weitere Suchmöglichkeiten erweitert werden. Bei der Suchoberfläche handelt es sich um die Websuche des \ac{plm}-Objektes Spezifikation. Die neuen Suchmöglichkeiten sollen es dem Benutzer bei der Suche von Spezifikationen ermöglichen, nach der Person, die das Objekt zuletzt bearbeitet hat, und dem dazugehörigen Datum zu selektieren. Die neue Funktionalität wurde von diversen SAP Kunden gefordert und kann folglich den Prozess beim Kunden entscheidend optimieren.

Für die Umsetzung des Kundenwunsches wurde die Referenzmodellierung genutzt, da die neuen Suchobjekte bereits bei den Suchoberflächen, der Materialstückliste, des Rezeptes und der Dokumente vorhanden waren. Aus der Referenzmodellanalyse konnten für die neue Suchfunktionalität für Spezifikationen hilfreiche Erkenntnisse für die Anpassung der beiden relevanten Suchscreens gewonnen werden.

%Die gewünschten Viewanpassungen wurden realisiert, dokumentiert und dem Kunden vorgestellt. Sie befinden sich aktuell bei den relevanten Kunden im Test. 

Im Rahmen des Customer Connection Programmes wurden die gewünschten Viewanpassungen realisiert, dokumentiert und an die Kunden in Form eines Hinweises übergeben. Der Hinweis befindet sich aktuell bei den relevanten Kunden im Test und beinhaltet allen wichtigen Informationen, die zum Implementieren und Nachvollziehen der Änderungen benötigt werden.

%Im Rahmen des Customer Connection Programmes wurde bereits ein Hinweis mit allen erforderlichen Systemanpassungen erstellt, der über die Funktionalität SNOTE in SAP-Systeme von Kunden eingespielt werden kann.

Nach Abnahme der Funktionalität durch den Kunden wird dieser Hinweis veröffentlicht und mit vielen anderen Anpassungen im nächsten Support Package für SAP On Premise Systeme zur Verfügung gestellt.

\acresetall
%\restoregeometry