%%%%%%%%%%%%%%%%%%%%%%%%%%%%%%%%%%%%%%%%%%%%%%%%%%%%%%%%%%
%   Vorlage von:
%
%   Prof. Dr. Bernhard Drabant
%   Prof. Dr. Dennis Pfisterer
%   Prof. Dr. Julian Reichwald
%
%%%%%%%%%%%%%%%%%%%%%%%%%%%%%%%%%%%%%%%%%%%%%%%%%%%%%%%%%%
%
% Änderungen und Anpassungen von Max Darmstadt
%
%%%%%%%%%%%%%%%%%%%%%%%%%%%%%%%%%%%%%%%%%%%%%%%%%%%%%%%%%%
%	ANLEITUNG: 
%
%   1. Ersetzen Sie firmenlogo.jpg im Verzeichnis img
%   2. Passen Sie alle Stellen im Dokument an, die mit 
%      @stud markiert sind
%
%%%%%%%%%%%%%%%%%%%%%%%%%%%%%%%%%%%%%%%%%%%%%%%%%%%%%%%%%%

%%%%%%%%%%%%%%%%%%%%%%%%%%%%%%%%%%%%%%%%%%%%%%%%%%%%%%%%%%
%	ACHTUNG: 
%
%   Für das Erstellen des Literaturverzeichnisses wird das 
%   modernere Paket biblatex in Kombination mit biber 
%   verwendet -- nicht mehr das ältere BibTex!
%   Bitte stellen Sie ggf. Ihre TeX-Umgebung entsprechend 
%   ein (z.B. TeXStudio: Einstellungen --> Erzeugen --> 
%   Standard Bibliographieprogramm: biber)
%
%%%%%%%%%%%%%%%%%%%%%%%%%%%%%%%%%%%%%%%%%%%%%%%%%%%%%%%%%%

% Layout des Dokumentes festlegen: 12pt = Schriftgröße,Abstand für die Bindung der Arbeit, DIV immer gleichgroß wie die Schrift + Satzspiegelgröße, Für mehr Platz auf der Seite, wahrscheinlich ob man bei der A4 seite auch nach unten mehr Platz hat ,Größe der Beschriftungsfläche,ermöglicht Absätze, Literaturverzeichnis zum Inhaltsverzeichnis , ,irgendwas mit Abblidungensnamen in Abbildungsverzeichnis aufnehmen, toc= TableOfContents ,Nummerierung im Inhaltsverzeichnis ohne Punkt, Maße der Fußlinie aus Plain übernommen
\documentclass[12pt,BCOR=5mm,DIV=12,headinclude=on,footinclude=off,parskip=half,bibliography=totoc,listof=entryprefix,
               toc=listof,pointlessnumbers,plainfootsepline]{scrreprt}

%Edit MaxD If you want to change the margins:
%\usepackage[paper=a4paper,left=30mm,right=30mm,top=25mm,bottom=25mm]{geometry}

% Elementare Pakete, Konfigurationen und Definitionen werden geladen
% !TEX root =  master.tex
%      HYPERREF

%%%%%%%%%%%%%%%%%%%%%%%%%%%%%%%%%%%%%%%%%%%%%%%%%%%%%%%%%%
%	ANLEITUNG: 
%
% Passen Sie alle Stellen im Dokument an, die mit 
% @stud markiert sind
%
% Die untersten packages können die oberen überschreiben, deswegen die wichtigen als letztes
%%%%%%%%%%%%%%%%%%%%%%%%%%%%%%%%%%%%%%%%%%%%%%%%%%%%%%%%%%

\usepackage{makeidx}      			  		% allows index generation, mit \index Wörter zum Stichwort Verzeichnis hinzufügen
\usepackage{listings}						%Format Listings properly , um code in Latex als Box einzufügen
\usepackage{lipsum}   						%Blindtext
\usepackage{graphicx}						% use various graphics formats, um Grafiken einzubinden
\usepackage[german]{varioref} 				% nicer references \vref
\usepackage{caption}						%better Captions
\usepackage{booktabs} 						%nicer Tabs
\usepackage{array}
\usepackage{chngcntr} 						%zum Neudefinieren von Zählern
\usepackage{fnpct} 							% Correct superscripts ; tauscht Fußnote mit Satzzeichen, wenn die Fußnote vor dem Satzzeichen steht
\usepackage[T1]{fontenc} 					%ermöglicht das Verwenden von Umlauten
\usepackage[utf8]{inputenc}					%Inputcode festlegen
\usepackage{calc}							% Used for extra space below footsepline
\usepackage[printonlyused]{acronym} 		%delete content of the square brackets to force print all acronyms
\usepackage{algorithm}  					% um code in Latex als Box(ähnlich wie eine figure) einzufügen
\usepackage{algpseudocode} 					%für verschiedene darstellweisen des pseudocodes von algorithm
%\usepackage{url} 							% Um urls einzufügen, veraltet, funktion mitlerweile bei hyperref integriert
\usepackage{comment}						%Für Kommentarabschnitte: \begin{comment} \end{comment}
%Edit MaxD %%%%%%%%%%%%%%%%%%%%%%%%%%%%%%%%%%%%%%%%%%%%%%%%%%%%
\usepackage{geometry}						%Zum Verändern der Gestaltung einer Seite: breite, höhe, oberer, unterer Rand, Ausdruck(einseitig, doppelseitig etc)
\usepackage{tikz}							%Zum Darstellen von Grafiken wie:  geometrische Formen oder Koordinatensysteme


\usepackage{tikz}
\usetikzlibrary{calc}
\usetikzlibrary{arrows.meta}
\usetikzlibrary{positioning}
\usetikzlibrary{shapes}
\usetikzlibrary{fit}
\usetikzlibrary{positioning}
\usetikzlibrary{shadows}


\usepackage{footnote}						%Für Fußnoten: \footnote{<text>}
\usepackage{wrapfig}						%Für Schrift umflossene Bilder
\usepackage{float}							%Ermöglicht bei figures: "das bestmögliche platzieren auf der Website"
\usepackage{array,multirow,graphicx}		%array = ermöglicht es, Spalten mit fester Breite auch zentriert oder rechtsbündig auszurichten
											%multirow = ermöglicht in Tabellen, dass Spalten und Zeilen zusammengefasst werden können
											%graphicx = Zum Einbinden von Bildern
\usepackage{listings}						% Um Programmcode in Latex einzubinden: \begin{lstlisting} ; \end{lstlisting}
\lstset{breaklines,numbers=left,xleftmargin=2em,frame=single,framexleftmargin=2em}    % automatische Zeilenumbrüche, wo die line numbers hinsollen, dicke des linken Randes; Rahmen um den Code;  

\usepackage{xargs}%Commands mit mehreren optionalen Parametern möglich
\usepackage[colorinlistoftodos,prependcaption,textsize=tiny]{todonotes} % Todonotes: mit \todo[]{} \missingfigure{} notes für fehlende Elemente einfügen, mit \listoftodos Liste der todos anzeigen lassen
\newcommandx{\unsure}[2][1=]{\todo[linecolor=red,backgroundcolor=red!25,bordercolor=red,#1]{#2}}
\newcommandx{\change}[2][1=]{\todo[linecolor=blue,backgroundcolor=blue!25,bordercolor=blue,#1]{#2}}
\newcommandx{\info}[2][1=]{\todo[linecolor=green,backgroundcolor=green!25,bordercolor=green,#1]{#2}}
\newcommandx{\improvement}[2][1=]{\todo[linecolor=purple,backgroundcolor=purple!25,bordercolor=purple,#1]{#2}}
\newcommandx{\thiswillnotshow}[2][1=]{\todo[disable,#1]{#2}} % usable todo commands: todo, unsure, change, info, improvement, thiswillnotshow

\usepackage[mediumspace,mediumqspace,Grey,squaren]{SIunits}
\usepackage{pgfplots}						%Zum Darstellen von Grafiken, Koodinatensystemen(auch mehrdimensional)
\usepgfplotslibrary{dateplot}				%Bei Grafiken zum Benutzen von Daten
\usepackage{pgfplotstable}					%Zur Tabellennachbearbeitung
\pgfplotsset{compat=1.16}
%\usepackage{draftwatermark}%Delete after you are done with your thesis. This line creates the "Draft" Watermark
\usepackage{setspace}						%Zeilenabstand 3 Optionen: singlespacing; onehalfspacing; doublespacing

%Edit MaxD Fixes line breaks in columns of tables. Example: C{2cm}
\newcolumntype{L}[1]{>{\raggedright\arraybackslash}p{#1}} % left fixed width
\newcolumntype{C}[1]{>{\centering\arraybackslash}p{#1}} % center fixed width
\newcolumntype{R}[1]{>{\raggedleft\arraybackslash}p{#1}} % flush right fixed width
%%%%%%%%%%%%%%%%%%%%%%%%%%%%%%%%%%%%%%%%%%%%%%%%%%%%%%%%%%%%%%%%%%%%%

%originally in line 21 needed to be moved to the end
\usepackage[hidelinks=true]{hyperref} % keine roten Markierungen bei Links

%
% @stud
%
%	FONT SELECTION: Entweder 1) Latin Modern oder 2) Times / Helvetica ( FONT = Schriftart)
\usepackage{lmodern}             % 1) Latin modern font
%\usepackage{mathptmx}           % 2) Helvetica / Times New Roman fonts (2 lines)
%\usepackage[scaled=.92]{helvet} % 2) Helvetica / Times New Roman fonts (2 lines)

%
% @stud
%
%	LANGUAGE SETTINGS
\usepackage[ngerman]{babel} 	        % german language
\usepackage[german=quotes]{csquotes} 	% correct quoting using \enquote{}
%\usepackage[english]{babel}          % english language
%\usepackage{csquotes} 	              % correct quoting using \enquote{}

%
% @stud
%
% Uncomment the following lines to support hard URL breaks in bibliography 
%\apptocmd{\UrlBreaks}{\do\f\do\m}{}{}
%\setcounter{biburllcpenalty}{9000}% Kleinbuchstaben
%\setcounter{biburlucpenalty}{9000}% Großbuchstaben

%
% @stud
%
%	FOOTNOTES: Count footnotes over chapters
\counterwithout{footnote}{chapter}

%	ACRONYMS
\makeatletter
\@ifpackagelater{acronym}{2015/03/20}
{\renewcommand*{\aclabelfont}[1]{\textbf{{\acsfont{#1}}}}}{}
\makeatother

%	LISTINGS
\renewcommand{\lstlistingname}{Quelltext} 
\renewcommand{\lstlistlistingname}{Quelltextverzeichnis}
\lstset{numbers=left,
	numberstyle=\tiny,
	captionpos=b,
	basicstyle=\ttfamily\small}

%	ALGORITHMS
\renewcommand{\listalgorithmname}{Algorithmenverzeichnis }
\floatname{algorithm}{Algorithmus}

%		PAGE HEADER / FOOTER
%	    Warning: There are some redefinitions throughout the master.tex-file!  DON'T CHANGE THESE REDEFINITIONS!
\RequirePackage[automark,headsepline,footsepline]{scrlayer-scrpage}
\pagestyle{scrheadings}
%\renewcommand*{\pnumfont}{\upshape\sffamily}
%\renewcommand*{\headfont}{\upshape\sffamily}
%\renewcommand*{\footfont}{\upshape\sffamily}
\renewcommand{\chaptermarkformat}{}
\RedeclareSectionCommand[beforeskip=0pt]{chapter}
\clearscrheadfoot

\ifoot[\rule{0pt}{\ht\strutbox+\dp\strutbox}DHBW Mannheim]{\rule{0pt}{\ht\strutbox+\dp\strutbox}DHBW Mannheim}
\ofoot[\rule{0pt}{\ht\strutbox+\dp\strutbox}\pagemark]{\rule{0pt}{\ht\strutbox+\dp\strutbox}\pagemark}
\ohead{\headmark}

\newcommand{\TitelDerArbeit}[1]{\def\DerTitelDerArbeit{#1}\hypersetup{pdftitle={#1}}}
\newcommand{\AutorDerArbeit}[1]{\def\DerAutorDerArbeit{#1}\hypersetup{pdfauthor={#1}}}
\newcommand{\Firma}[1]{\def\DerNameDerFirma{#1}}
\newcommand{\Kurs}[1]{\def\DieKursbezeichnung{#1}}
\newcommand{\Abteilung}[1]{\def\DerNameDerAbteilung{#1}}
\newcommand{\Studiengangsleiter}[1]{\def\DerStudiengangsleiter{#1}}
\newcommand{\WissBetreuer}[1]{\def\DerWissBetreuer{#1}}
\newcommand{\FirmenBetreuer}[1]{\def\DerFirmenBetreuer{#1}}
\newcommand{\Bearbeitungszeitraum}[1]{\def\DerBearbeitungszeitraum{#1}}
\newcommand{\Abgabedatum}[1]{\def\DasAbgabedatum{#1}}
\newcommand{\Matrikelnummer}[1]{\def\DieMatrikelnummer{#1}}
\newcommand{\Studienrichtung}[1]{\def\DieStudienrichtung{#1}}
\newcommand{\ArtDerArbeit}[1]{\def\DieArtDerArbeit{#1}}
\newcommand{\Literaturverzeichnis}{Literaturverzeichnis}

\newcommand{\settingBibFootnoteCite}{
	\setlength{\bibparsep}{\parskip}		  % Add some space between biblatex entries in the bibliography
	\addbibresource{bibliography.bib}	    % Add file bibliography.bib as biblatex resource
	\DefineBibliographyStrings{ngerman}{andothers = {{et\,al\adddot}},}
	%Following two lines commented, because those features are no longer supported
	%\AdaptNoteOpt\footcite\multfootcite   % Will add  separators if footcite is called multiple consecutive times 
	%\AdaptNoteOpt\autocite\multautocite   % Will add  separators if autocite is called multiple consecutive times
}

\newcommand{\setTitlepage}{
	% !TEX root =  master.tex
\begin{titlepage}
\begin{minipage}{\textwidth}
		\vspace{-2cm}
		\noindent \includegraphics[scale=0.25]{\imagedir/firmenlogo.jpg} \hfill \includegraphics{\imagedir/logo.jpg}
\end{minipage}
\vspace{1em}
%\sffamily
\begin{center}
	{\textsf{\large Duale Hochschule Baden-W\"urttemberg Mannheim}}\\[4em]
	{\textsf{\textbf{\large{\DieArtDerArbeit}arbeit}}}\\[6mm]
	{\textsf{\textbf{\Large{}\DerTitelDerArbeit}}} \\[6mm]%[1.5cm]
	%{\textsf{\large{}Eine vergleichende Untersuchung und Konzeption}}\\[1.5cm]
	{\textsf{\textbf{\large{}Studiengang Wirtschaftsinformatik}}\\[6mm]
	\textsf{\textbf{Studienrichtung \DieStudienrichtung}}}\vspace{7em}
	
	\begin{minipage}{\textwidth}
		\begin{tabbing}
		Wissenschaftliche(r) Betreuer(in): \hspace{0.85cm}\=\kill
		Verfasser(in): \> \DerAutorDerArbeit \\[1.5mm]
		Matrikelnummer: \> \DieMatrikelnummer \\[1.5mm]
		Firma: \> \DerNameDerFirma  \\[1.5mm]
		Abteilung: \> \DerNameDerAbteilung \\[1.5mm]
		Kurs: \> \DieKursbezeichnung \\[1.5mm]
		Studiengangsleiter: \> \DerStudiengangsleiter \\[1.5mm]
		Wissenschaftliche(r) Betreuer(in): \> \DerWissBetreuer \\[1.5mm]
		%\> 0621 4105 1218 \\[1.5mm]
		%\> hans-peter.engel@dhbw-mannheim.de \\[1.5mm]
		Firmenbetreuer(in): \> \DerFirmenBetreuer \\[1.5mm]
		%\> +49 6227 7-70239 \\[1.5mm]
		%\> pierre.grosse@sap.com \\[1.5mm]
		Bearbeitungszeitraum: \> \DerBearbeitungszeitraum\\[1.5mm]
%		alternativ:\\[1.5mm]
		Eingereicht: \> \DasAbgabedatum	
		\end{tabbing}
	\end{minipage}
\end{center}
\end{titlepage}
	\pagenumbering{roman} % Römische Seitennummerierung
	\normalfont	
}

%
% @stud
%
\newcommand{\settingLists}{
	%	Inhaltsverzeichnis
	%\begin{spacing}{0.95} %Change line spacing to fit table of contents on one page
		\tableofcontents
	%\end{spacing}

	%	Abbildungsverzeichnis
	\listoffigures
	%	Tabellenverzeichnis
	%\listoftables
	%	Listingsverzeichnis / Quelltextverzeichnis
	%\lstlistoflistings
	% Algorithmenverzeichnis
	%\listofalgorithms
}

\newcommand{\initializeText}{
	\clearpage
	\ihead{\chaptername~\thechapter} % Neue Header-Definition
	\pagenumbering{arabic}           % Arabische Seitenzahlen
	\onehalfspacing
}

\newcommand{\initializeBibliography}{
	\ihead{}
	\printbibliography[title=Quellenverzeichnis]
	\pagebreak 
	\pagenumbering{roman}
	\cleardoublepage
	\singlespacing
	
	
}

\newcommand{\initializeAppendix}{
	\appendix
	\ihead{\appendixname~\thechapter}
	\ihead{}
	\singlespacing
	
}

%EditMaxD
\newcommand{\setTodo}{
	\listoftodos[Todos]
	\newpage
}

%Defining Custom Colors
\definecolor{sapBlue1}{RGB}{0,185,242}
\definecolor{sapBlue2}{RGB}{1,156,224}
\definecolor{sapBlue3}{RGB}{12,126,207}
\definecolor{sapBlue4}{RGB}{22,97,190}

\definecolor{sapGold}{RGB}{240,171,0}


%%%%%%%%%%%%%%%%%%%%%%%%%%%%
%
% @stud
%
%	SCHRIFTART (Schrift mit oder ohne Serifen im gesamten Text) 
%
% mit Serifen
%\addtokomafont{disposition}{\rmfamily}
%\renewcommand*{\familydefault}{\rmdefault}
%
% ohne Serifen (default)
%\addtokomafont{disposition}{\sffamily}
%
%%%%%%%%%%%%%%%%%%%%%%%%%%%%

%%%%%%%%%%%%%%%%%%%%%%%%%%%%
%
% @stud
%
% PERSÖNLICHE ANGABEN (BITTE VOLLSTÄNDIG EINGEBEN zwischen den Klammern: {...})
%
\ArtDerArbeit{Projekt} % "Bachelor" oder "Projekt" wählen
\TitelDerArbeit{Digitale Plattformen als Wettbewerbsfaktor für den deutschen Kfz-Versicherungsmarkt am Beispiel der SAP Business Technology Platform}
\AutorDerArbeit{Jonas Mayer}
\Abteilung{Practice Unit Financial Service Industries}
\Firma{SAP SE}
\Kurs{WWI20SCB}
\Studienrichtung{Sales \& Consulting}
\Matrikelnummer{7415525}
\Studiengangsleiter{Herr Prof. Dr. -Ing. Martin}
\WissBetreuer{Dominik Bredel}
\FirmenBetreuer{Bernd Helb}
\Bearbeitungszeitraum{13.02.2023 -- 08.05.2023}
\Abgabedatum{08.05.2023}
%
%%%%%%%%%%%%%%%%%%%%%%%%%%%%

%%%%%%%%%%%%%%%%%%%%%%%%%%%%
%
% @stud
%
%	BIBLIOGRAPHY (@stud: Bibliographie-Stil wählen - Position und Indizierung)
%
% Auswahl zwischen: IEEE Style, ALPHABETIC Style, HARVARD Style, AUTHOR-YEAR Style 
%
% (oder eigenen zulässigen Stil wählen) 
%

% Position des Zitats
%
%\newcommand{\position}{inline} 
\newcommand{\position}{footnote}

% Indizierung des Zitats
%
% 1) NUMERIC Style - e. g. [12]
%\newcommand{\indextype}{numeric} 
%
% 2) ALPHABETIC Style - e. g. [AB12]
%\newcommand{\indextype}{alphabetic} 
%
% 3) IEEE Style - numeric kind of style 
%\newcommand{\indextype}{ieee} 
%
% 4) HARVARD Style 
\newcommand{\indextype}{apa} 
%
% 5) CHICAGO Style 
%\newcommand{\indextype}{authoryear}
%
%%%%%%%%%%%%%%%%%%%%%%%%%%%%
%Edit MaxD commented the line below, because it made the font look weird.
%\renewcommand*{\familydefault}{\sfdefault}

\usepackage[backend=biber, autocite=\position, style=\indextype]{biblatex} 	%package zur Gestaltung des Literaturverzeichnisses:Legt das Backend zum Sortieren der Bibliographie fest, mit sorting=ynt lässt sich festlegen wie das Literaturverzeichnis sortiert werden soll, lässt sich bibliography style festlegen: style=alphabetic
\apptocmd{\UrlBreaks}{\do\f\do\m}{}{}
\setcounter{biburllcpenalty}{9000}% Kleinbuchstaben
\setcounter{biburlucpenalty}{9000}% Großbuchstaben

\settingBibFootnoteCite

\newcommand{\abs}{\par\vskip 0.2cm\goodbreak\noindent}
\newcommand{\nl}{\par\noindent}
\newcommand{\mcl}[1]{\mathcal{#1}}
\newcommand{\nowrite}[1]{}
\newcommand{\NN}{{\mathbb N}}

\newcommand{\imagedir}{img}

\makeindex %erstellt die Indexdatei, in die dann mit \index{Stichwort} aus dem Main Einträge gemacht werden können
\begin{document}
%\setTodo
\setTitlepage

%%%%%%%%%%%%%%%%%%%%%%%%%%%%%%%%%%%%%%%%%%%%%%%%%%%%%%%%%%%%%%%%%%%%%%%%%%%%%%%%%%%%%%%%%%
% KAPITEL UND ANHÄNGE
%
% @stud:
%   - nicht benötigte: auskommentieren/löschen
%   - neue: bei Bedarf hinzufügen mittels input-Kommando an entsprechender Stelle einfügen
%%%%%%%%%%%%%%%%%%%%%%%%%%%%%%%%%%%%%%%%%%%%%%%%%%%%%%%%%%%%%%%%%%%%%%%%%%%%%%%%%%%%%%%%%%

%%%%%%%%%%%%%%%%%%%%%%%%%%%%%%%%%%%
% SPERRVERMERK
%
% @stud: nondisclosurenotice.tex bearbeiten
%
%\input{nondisclosurenotice} 
%%%%%%%%%%%%%%%%%%%%%%%%%%%%%%%%%%%

%%%%%%%%%%%%%%%%%%%%%%%%%%%%%%%%%%%
%	KURZFASSUNG
%
% @stud: acknowledge.tex bearbeiten
%
%\input{acknowledge} 
%%%%%%%%%%%%%%%%%%%%%%%%%%%%%%%%%%%

%%%%%%%%%%%%%%%%%%%%%%%%%%%%%%%%%%%
%	Disclaimer
%
% @stud: acknowledge.tex bearbeiten
%
% Für Gender und englische Begriffe: 
\chapter*{Disclaimer}

In dieser Arbeit wird aus Gründen der besseren Lesbarkeit das generische Maskulinum verwendet. Weibliche und anderweitige Geschlechteridentitäten werden dabei ausdrücklich mitgemeint, soweit es für die Aussage erforderlich ist. Des Weiteren werden die englischen Begriffe der Informationstechnologie in ihrer Originalsprache verwendet, um die Bedeutung der Ausdrücke nicht zu verfälschen.
\pagebreak 
%%%%%%%%%%%%%%%%%%%%%%%%%%%%%%%%%%%

%%%%%%%%%%%%%%%%%%%%%%%%%%%%%%%%%%%
%	KURZFASSUNG
%
% @stud: abstract.tex bearbeiten
%
% !TEX root =  master.tex
%\newgeometry{top=30mm,bottom=40mm}
%\thispagestyle{empty}
\thispagestyle{plain}
\begin{minipage}{\textwidth}
        \vspace{-2cm}
        \noindent  \hfill \includegraphics{\imagedir/logo.jpg}
    \end{minipage}
\begin{flushright}
        \footnotesize{Wirtschaftsinformatik}\\
        \medskip
        \small{\textbf{\DerTitelDerArbeit}}\\
        \medskip
        \footnotesize{{\DieArtDerArbeit}arbeit}\\
        \footnotesize{\DerAutorDerArbeit}\\
        \footnotesize{\DerNameDerFirma}\\
\end{flushright}

\section*{Kurzfassung}

\improvement{Empfehlung Luca: Ich muss hier nicht auf das Vergleichen mit der Konkurrenz eingehen}

%\textbf{Kurzfassung}\\\\

Die Kfz-Versicherungsbranche befindet sich im Zuge der digitalen Transformation in einem enormen Wandel. So erwarten die Kunden heutzutage einen komfortablen und reibungslosen Online-Service. Um wettbewerbsfähig zu bleiben, müssen Kfz-Versicherer deshalb mithilfe einer digitalen Plattform ihre internen Prozesse optimieren, „sich mit Partnern integrieren“ und zusätzliche Services für den Endkunden anbieten. 
Ziel der vorliegenden Abhandlung ist es daher, die Anforderungen der Kfz-Versicherer an eine digitale Plattform zu identifizieren, um anschließend beurteilen zu können, inwiefern diese Anforderungen von der SAP Business Technology Platform erfüllt werden.
Zur Beantwortung der Forschungsfrage wurde die Task-Technology-Fit-Theorie angewendet. Dabei wurden zunächst die Anforderungen der Kfz-Versicherer mithilfe einer systematischen Literaturanalyse identifiziert und anschließend mithilfe von Experteninterviews ergänzt und evaluiert. Die Priorisierung der Experten hat gezeigt, dass bei Erfüllung der grundlegenden Anforderungen wie Datensicherheit und Skalierbarkeit, für Kfz-Versicherer in den nächsten 3-5 Jahren insbesondere die Anforderungsbereiche Integration und Anwendungsentwicklung wichtig sind. Nach der Identifikation wurden die Anforderungen mit den Technologie-Charakteristika der SAP BTP gegenübergestellt. Hier hat die Analyse gezeigt, dass die SAP BTP 19 der 20 Anforderungen grundlegend erfüllt und damit als digitale Plattform für Kfz-Versicherer geeignet ist. Abschließend wurde in der Handlungsempfehlung eine schrittweise Vorgehensweise zur Verwendung der SAP Business Technology Plattform bei Kfz-Versicherern aus dem Analyseergebnis und der Priorisierung der Anforderungen durch die Experten abgeleitet. In weiterführenden Fragestellungen könnte unter anderem erörtert werden, wie sich der Trend der digitalen Ökosysteme auf die Kfz-Versicherungsbranche auswirkt und welche Rolle dabei insbesondere die in dieser Arbeit untersuchten digitalen Plattformen spielen.
Die Arbeit ist damit für alle relevant, die sich mit der digitalen Transformation der deutschen Kfz-Versicherungsbranche befassen, da …..



% Inhalte:
% -Problemstellung
% -Forschungsfrage
% -Methodik
% -Wichtige Ergebnisse
% -Interpretation
% -meist ungefähr 150 Wörter



%\restoregeometry 
%%%%%%%%%%%%%%%%%%%%%%%%%%%%%%%%%%%

%%%%%%%%%%%%%%%%%%%%%%%%%%%%%%%%%%%
% VERZEICHNISSE
%
% @stud: ggf. nicht benötigte Verzeichnisse auskommentieren/löschen in Def. von \settingLists in config.tex
%
\settingLists
%%%%%%%%%%%%%%%%%%%%%%%%%%%%%%%%%%%

%%%%%%%%%%%%%%%%%%%%%%%%%%%%%%%%%%%
% ABKÜRZUNGSVERZEICHNIS
%
% @stud: acronyms.tex bearbeiten
%
% !TEX root =  master.tex
\clearpage
\chapter*{Abkürzungsverzeichnis}	
\addcontentsline{toc}{chapter}{Abkürzungsverzeichnis}

\begin{acronym}[XXXXXXX]
	\acro{4gl}[4GL]{fourth generation langugage}
	\acro{abap}[ABAP]{Advanced Business Application Programming}
	\acro{adt}[ADT]{ABAP Development Tools}
	\acro{api}[API]{Application Programming Interface}
	\acro{bwa}[BWA]{Business Warehouse Accelerator}
	\acro{erp}[ERP]{Enterprise-Resource-Planning}
	\acro{es}[ES]{Enterprise Search}
	\acro{hana}[HANA]{High Performance Analytic Appliance}
	\acro{ide}[IDE]{Integrated Development Environment}
	\acro{otr}[OTR]{Online Text Repository}
	\acro{pam}[PAM]{Product Availibility Matrix}
	\acro{plm}[PLM]{Product Lifecycle Management}
	\acro{plmwui}[PLMWUI]{Product Lifecycle Management Web User Interface}
	\acro{ui}[UI]{User Interface}
	\acroplural{ui}[UIs]{User Interfaces}
	\acro{s/4hana}[S/4HANA]{SAP-Business-Suite 4 SAP HANA}
	\acro{sql}[SQL]{Structured Query Language}
	\acro{trex}[TREX]{Stand-Alone Engine Search and Classification}
%	\acro{srm}[SRM]{Supplier Relationship Management}
%	\acro{DHBW}{Duale Hochschule Baden-Württemberg}
\end{acronym} 
%%%%%%%%%%%%%%%%%%%%%%%%%%%%%%%%%%%
\initializeText

%%%%%%%%%%%%%%%%%%%%%%%%%%%%%%%%%%%
% KAPITEL
%
% @stud: einzelne Kapitel bearbeiten und eigene Kapitel hier einfügen
%
%Test
%%%%%%%%%%%%%%%%%%%%%%%%%%%%%%%%%%%

%%%%%%%%%%%%%%%%%%%%%%%%%%%%%%%%%%%
% Vorgehensweise
%
% @stud: einzelne Kapitel bearbeiten und eigene Kapitel hier einfügen
%
\chapter{Einleitung}

\section{Motivation und Problemstellung}

Stichpunkte:
-	Wandel von Pipeline zu Plattform-basierten Geschäftsmodellen

-	Ein sehr bekanntes Beispiel für Plattformbasierte Geschäftsmodelle sind D.Ö..

-	Neu entstehende Kundenerwartungen können nur noch erfüllt werden indem verschiedene U. zusammenarbeiten gelang es  ökosystem¬basierten  Unternehmen,  ihren  Marktwert  signifikant  zu  steigern.  Während  vor  zehn  Jahren Mineralölkonzerne und Konsumgüterfirmen das Ran¬king  der  weltweit  größten  Firmen  nach  Marktkapitalisierung  dominierten,  stehen  heute  Technologiegiganten  wie  Apple,  Amazon,  Alphabet,  Microsoft,  Tencent  oder  Alibaba  an  der  Spitze.

-	Bis 2030 werden digitale Ökosysteme weltweit einen Umsatz von rund 60 Billionen Euro erwirtschaften, und die Hoffnungen, die mit ihnen verbunden sind, sind so groß wie ihr Potenzial. (McKinsey)

-	KPMG Studie: Kompositgeschäft in der Versicherungsindustrie: Internetbasierte Ökosysteme werden nur 10 bis 15 Prozent des Marktes ausmachen, aber 30\% des Gewinns nehmen bis 2030

-	6 der 7 größten Unternehmen der Welt setzen auf digitale Ökosysteme

-	Bekanntes sehr erfolgreiches Beispiel aus Versicherungsindustrie ist der chinesische Versicherer Ping An,

-	Aus einem Anbieter von Lebensversicherungen hat sich hier ein Ökosystem entwickelt, das weit über das Kerngeschäft hinaus zu Gesundheit, Banking, Wohnen, Smart Home, Mobilität und Unterhaltung anbietet. In China umfasst eine Online-Gebrauchtwagenplattform beispielsweise integrierte Finanzdienstleistungen von Ping An Insurance. Personen, die ein gebrauchtes Auto online kaufen oder handeln, haben die Möglichkeit, am Kaufort eine Kfz-Versicherung abzuschließen - und die Kunden reagieren. Über 20\% der aktuellen Autoversicherungsverkäufe von Ping An werden derzeit über diese hochvolumige Plattform generiert.

-	Nach einer Accenture Studie ist der vielversprechende Bereich für D.Ö. in der Vers.branche der Mobility Bereich

-	Digitale Plattform bilden die technische Grundlage digitaler Ökosysteme

-	Darüber hinaus ist der Kfz-Bereich, der digitalisierteste Bereich der Branche --> Im Markt der Schaden- und Unfallversicherungen liegt der Anteil der direkt abgeschlossenen Policen hingegen bei 15,0\%. Hauptgrund dafür ist der hohe Anteil der Kfz-Versicherungen, die über Aggregatoren und die Online-Vertriebswege der Versicherer verkauft werden.










\newpage
\section{Zielsetzung und Abgrenzung}

Zielsetzung:

-	Übergeordnete Forschungsfrage: Wie kann die SAP Business Technology Plattform Kfz-Versicherern bei Aufbau oder Partizipation an Digitalen Ökosystemen helfen?

-	Unterfrage 1: Welche Anforderungen muss eine technologische Plattform erfüllen, um für Kfz-Versicherer einen Mehrwert schaffen zu können?

-	Unterfrage 2: Inwiefern werden diese Anforderungen von der SAP Business Technology Plattform erfüllt?


Abgrenzung:

-	Konzentration auf den deutschen Markt

-	Begrenzung auf Erstversicherer bzw. Kfz-Versicherer, da hier das Thema D. Ö. am vielversprechendsten ist

-	Nicht im Fokus der Arbeit ist die Frage, ob Kfz-Versicherer zukünftig bei Mobilitätssystemen Orchestrator oder Teilnehmer sein werden


\newpage
\section{Aufbau der Arbeit}

\begin{figure}[h]
    \centering
    \includegraphics[width=1\textwidth]{img/Aufbau_der_Arbeit.jpg}
    \caption[Aufbau der Arbeit]{Aufbau der Arbeit\autocite{Aufbau}}
    \label{fig:Aufbau}
\end{figure}
\footnotetext{eigene Darstellung}

\newpage
%%%%%%%%%%%%%%%%%%%%%%%%%%%%%%%%%%%

%%%%%%%%%%%%%%%%%%%%%%%%%%%%%%%%%%%
% Vorgehensweise
%
% @stud: einzelne Kapitel bearbeiten und eigene Kapitel hier einfügen
%
\chapter{Grundlagen}

\section{Digitale Plattformen als disruptive Innovation}

\subsection{Definition und begriffliche Abgrenzung}

Aktuell existieren sowohl in der wissenschaftlichen Literatur als auch in der unternehmerischen Praxis eine Vielzahl unterschiedlicher Ansätze zur Definition und Kategorisierung Digitaler Plattformen. Diese haben gemeinsam, dass sie grundsätzlich zwischen einer marktorientierten und einer technologieorientierten Betrachtungsperspektive unterscheiden, welche in einem komplementären Bezug zueinanderstehen und sich somit ergänzen. \autocite[Vgl.][S. 21-23]{ENGELS2017} \autocite[Vgl.][S.99]{MEINHARDT2019}

Bei einem technologieorientierten Ansatz wird eine digitale Plattform als eine Menge von Kernprodukten, -technologien oder -services verstanden, auf deren Basis weitere komplementäre Produkte, Technologien oder Services entwickelt und angebunden werden können. \autocite[Vgl.][S. 21]{ENGELS2017} Folglich werden hier auch komplexe IT-Systeme, Software- und Hardware-Plattformen dazugezählt.\autocite[Vgl.][S.222f]{WEINREICH2016}

Aus einer marktorientierten Perspektive wird eine Plattform als ein Markt betrachtet, in welchem gegenseitige Nutzerinteraktionen zu Netzwerkeffekten führen. \autocite[Vgl.][S. 1273f]{EISENMANN2011} Diese sind aufgrund der digitalen Plattform mit minimalen Investitionen skalierbar, weshalb sie insbesondere in traditionellen Märkten als disruptive Innovation angesehen werden. \autocite[Vgl.][S. 17ff]{MOAZED2016} Hierbei begünstigen die technologischen Möglichkeiten wie Cloud-Computing in Verbindung mit den Netzwerkeffekten die Bildung von Oligopol- bzw. Monopol-Marktstrukturen.\autocite[Vgl.][S. 23]{ENGELS2017} Ein Beispiel hierfür sind die sogenannten GAFA Unternehmen: Google(Alphabet), Amazon, Facebook(Meta) und Apple, welche digitale Plattformen zur Realisierung von digitalen Ökosystemen einsetzen und sich dadurch ihre führende Marktposition sichern konnten. \autocite[Vgl.][S. 92f]{BUNTE2020} 

Nach dem Frauenhofer Institut für Experimentelles Software Engineering (IESE) ist ein Digitales Ökosystem ein sozio-technisches System, bei dem Menschen und Unternehmen über eine digitale Plattform zusammenzuarbeiten, um eine Wertschöpfung zu erzielen, welcher ohne das Ökosystem nicht möglich wäre. \autocite[Vgl.][S. 376]{MULLERSTEWENS2019} Dabei stellt die digitale Plattform die technologische Grundlage für das dazugehörige Ökosystem dar. \autocite[Vgl.][S. 376]{IESE2021} Folglich sind die beiden Begrifflichkeiten sehr eng miteinander verknüpft.

Im Rahmen dieser Arbeit soll die Einsatzmöglichkeiten SAP BTP bei deutschen Kfz-Versicherern evaluiert werden, weshalb insbesondere die technische Betrachtungsweise für die Arbeit relevant ist. Diese wird in ... mit Cloud Computing auf digitalen Plattformen noch weiter erläutert.

%Im Rahmen dieser Arbeit sollen digitale Plattformen ganzheitlich untersucht werden, weshalb beide Betrachtungsweisen für die Arbeit relevant sind. --> zu groß

\subsection{Wertschöpfung auf digitalen Plattformen}

Bei der Betrachtung der Wertschöpfung muss neben der digitalen Plattform als solches auch das dazugehörige Ökosystem betrachtet werden, in welchem die Plattform die Koordinationsstruktur für die unterschiedlichen Teilnehmer darstellt. Die digitale Plattform ermöglicht das gemeinsame Wertschöpfen, auch Value Co-Creation genannt, durch die Bereitstellung von komplementären Services und Produkten. Dabei ist der Wert der Plattform von den in Wechselwirkungen stehenden Faktoren der Größe des Kundenstamms als auch von der Vielfältigkeit der angebotenen Produkte und Services abhängig. Diese bestehende Wechselwirkung wird umgangssprachlich auch als „Henne-oder-Ei Prinzip“ bezeichnet und stellt eine besondere Herausforderung für neu entstehende digitale Plattformen dar, weil keine der beiden Seiten bereit ist, sich anzuschließen, solange die andere Partei nicht ausreichend vertreten ist. \autocite[Vgl.][S. 310]{CAILLAUD2003} Diese Wechselwirkung führt zu den sogenannten Netzwerkeffekten, welche vor allem mit einer steigenden Anzahl an Teilnehmern exponentiell wachsen. Daher haben sowohl Plattformbetreiber als auch Kunden und Partner ein großes Interesse möglichst viele potenzielle Nutzer auf die Plattform zu bringen und mit ihnen zu interagieren. \autocite[Vgl.][S. 596-600]{HAHN2016} 

Neben dem Ermöglichen von Komplementären gehört auch das Schaffen neuer Vertriebskanäle zur Wertschöpfung digitaler Plattformen. So werden traditionelle Vertriebsstrukturen durch elektronische Marktplätze ersetzt und dadurch der Kundenzugang für komplementäre Dienstleister erleichtert. Diese profitieren darüber hinaus auch von vereinfachten und vereinheitlichten Transaktions- und Koordinationsaktivitäten. Dies kann beispielsweise durch die Standardisierung von Prozessen und Informationsflüssen erfolgen. So sind Applikationen für den Endkunden bereits vor integriert, können bei Bedarf schnell erworben und produktiv genutzt werden.\autocite[Vgl.][S. 599f.]{HAHN2016}

Des Weiteren ergeben sich aus Sicht des Plattformbetreibers ebenfalls Mechanismen zur Wertabschöpfung: Bundling und Lock-in-Effekt. Bundling bezeichnet den Verkauf von gebündelten Produkten und Dienstleistungen für einen vorher bestimmten Preis. Abhängig vom Preismodell können diese entweder zu einem festen Preis erworben oder bis zum produktiven Nutzen in einem Freemium Model kostenlos getestet werden.\autocite[Vgl.][S. 178-185]{TEECE2010} Der Lock-in-Effekt beschreibt die Tendenz von Nutzern und Partnern aufgrund hoher Wechselkosten und mangelnder Interoperabilität langfristig an eine bestimme Plattform gebunden zu sein. \autocite[Vgl.][S. 22]{STEUR2022} Er wird in der Praxis vor allem durch proprietäre Technologie oder das strategische Platzieren von Produkten und Dienstleistungen erreicht. \autocite[Vgl.][S. 704]{BALLON2011}

Um dieser Bedrohung zu entgehen, können Kunden und Partner auf mehreren konkurrierenden Plattformen vertreten sein und damit das sogenannte Multihoming betreiben. \autocite[Vgl.][S. 461ff]{CENNAMO2018} Zudem profitieren Kunden von der Modularität digitaler Plattformen, welche entlehnt aus der Systemtheorie, die Zerlegung komplexer Systeme in separate Subsysteme (Module) bezeichnet, die jeweils alle für sich alleine funktionieren. \autocite[Vgl.][S. 2]{LECHNER2019} Folglich können Kunden selbst entscheiden, welche Funktionen und Services sie von der Plattform nutzen möchten.

%Nicht 2 Mal auf digitalen Plattformen schreiben
\subsection{Cloud-Computing auf digitalen Plattformen}

Die Bereitstellung und Nutzung von Software und Hardware als Service hat sich als langfristiger Trend sowohl im Bereich der Verbraucher als auch bei Geschäftskundensegment etabliert und ist mittlerweile unter dem Schlagwort „Cloud Computing“ allgegenwärtig. Eine einheitliche Definition dieses Begriffs ist in der Wirtschaft und Wissenschaft bisher nicht vorhanden. Allerdings haben die meisten Definitionen Überschneidungspunkte, welche in der Beschreibung des National Institute of Standards and Technology (NIST) zusammengefasst werden: 

\begin{center}
    \textit{\enquote{Cloud Computing ist ein Modell für den allgegenwärtigen und bedarfsgerechten Netzzugang zu einem gemeinsam genutzten Pool konfigurierbarer Rechenressourcen (z.B. Netze, Server, Speicher, Anwendungen und Dienste), die mit minimalen Managementaufwand oder geringer Serviceprovider-Interaktion zur Verfügung gestellt werden können.}} \autocite[S. 2]{MELL2011}
\end{center}

Dabei wird Cloud Computing in drei verschiedene Servicemodelle: Software-as-a-Service (SaaS), Platform-as-a-Service (PaaS) und Infrastructure-as-a-Service (IaaS) unterteilt. Während bei IaaS Hardware-Ressourcen wie Server, Netzwerke, Middleware und Speicherplatz bereitgestellt und vom Nutzer selbst verwaltet werden können, sind es bei SaaS die Software-Applikationen, die der Anwender über das Netz nutzt. Die dazugehörigen Verwaltungsaufgaben der genutzten Cloud Infrastruktur werden bei SaaS vom Anbieter durchgeführt. Im Kontext digitaler Plattformen sind insbesondere PaaS-Lösungen relevant.\autocite[Vgl.][S. 2f]{MELL2011} Diese stellen ein Set an Technologien zum Betreiben und Entwickeln von SaaS Applikationen bereit und übernehmen für den Nutzer die Verwaltung der Infrastruktur.\autocite[Vgl.][S. 8]{BRAUNINGER2012}

Die Hauptkomponente von PaaS Lösungen ist die Application Runtime Environment (ARE), welche als Ausführungsumgebung für SaaS-Applikationen dient und die üblichen Anforderungen an Cloud-Software wie zum Beispiel Mandantenfähigkeit, Skalierbarkeit und Verfügbarkeit erfüllen muss. Damit mehrere Nutzer eine Instanz einer SaaS-Applikation nutzen können, ermöglicht die ARE häufig eine Multi-Tenancy-Architektur. Darüber hinaus wird eine integrierte Entwicklungsumgebung (IDE) bereitgestellt, welche mehrere Programmiersprachen unterstützt und Entwicklern verschiedene Bibliotheken und Werkzeuge zum Modellieren, Implementieren und Testen bietet. Abhängig von der Anwendungsdomäne können unterschiedliche Datenbankensysteme unterstützt und zusätzlich auch externe Datenquellen über Application Programming Interfaces (API) integriert werden. \autocite[Vgl.][S. 371]{BEIMBORN2011}

Zunehmend werden zudem auf PaaS-Lösungen auch Services zur Datenverarbeitung und Datenanalyse, Dienste zur Anbindung von Internet-of-Things (IoT) Geräten als auch eine Reihe von Tools zur Entwicklung von IoT-Anwendungen zur Verfügung gestellt. \autocite[Vgl.][]{IBM2023}

Darüber hinaus lassen sich PaaS-Angebote auf das Vorhandensein von sogenannten SaaS-Kern-Applikationen unterscheiden. Die bisherigen Angebote beschreiben reine PaaS-Angebote wie beispielsweise die Google App Engine. Zusätzlich gibt es jedoch auch von großen Softwareherstellern betriebene PaaS-Lösungen auf Applikationsbasis (APaaS), auf denen Entwickler Erweiterungen für eine SaaS-Applikation entwickeln und dabei auf Daten und Funktionen der Kernapplikation zugreifen können. Ein Beispiel hier ist die von PaaS von Salesforce Force.com mit der Kernapplikation Salesforce.com, bei der die Erweiterung ausschließlich als Ergänzung zur Kern-Applikation sinnvoll sind. \autocite[Vgl.][S. 371]{BEIMBORN2011} 

Neben den bisher dargestellten Komponenten stellen Plattformbetreiber häufig weitere Dienste entlang der Wertschöpfungskette bereit, wie beispielsweise Marketing und Vertrieb über elektronische Marktplätze. Weitere Mehrwertdienste umfassen hier ebenfalls die Abwicklung von Transaktionen und Zahlungen einschließlich Aktivitäten wie Vertragsabschluss, Rechnungsstellung und regelmäßiger Zahlungsabwicklung. Des Weiteren können Support-Dienstleistungen wie First-Level-Support, Zertifizierung, Qualitätssicherung sowie Monitoring Funktionalitäten angeboten werden. \autocite[Vgl.][S. 598]{HAHN2016} Eine Übersicht über die obligatorischen und optionalen Komponenten einer PaaS Lösung ist in Abbildung \vref{fig:PaaSK} dargestellt.

%\autocite[Vgl.][S. 599]{HAHN2016}
%\autocite[Vgl.][S. 372]{BEIMBORN2011}

\begin{figure}[h]
    \centering
    \includegraphics[width=1\textwidth]{img/PaaS_Komponenten.jpg}
    \caption[Komponenten einer PaaS-Lösung]{Komponenten einer PaaS-Lösung\autocite{PaaSK}}
    \label{fig:PaaSK}
\end{figure}
\footnotetext{Vgl. eigene Darstellung angelehnt an: Hahn, 2016, S. 599 und Beimborn et al., 2011, S. 372.}

In einem solchen Plattform-Ökosystem gibt es mindestens drei Akteure: Den Plattformbetreiber, die Komplementäre, welche beispielsweise Entwickler oder auch ein Independent Service Provider (ISV)Entwickler sein können, sowie die Endkunden (B2B oder B2C). Der Plattformbetreiber stellt im Falle von APaaS die Plattform mit der dazugehörigen Kernapplikation zur Verfügung und ermöglicht es Komplementären wie ISVs Add-Ons für sein Kernprodukt zu entwickeln, um die Reichweite des Hauptproduktes zu steigern. Die ISV können zur Entwicklung von Erweiterungen und Stand-alone-Applikationen dabei auch bestehende Teile der Plattform nutzen und neu kombinieren.(Mash-Up) Darüber hinaus kann der Plattformanbieter auch selbst die Rolle eines Anwendungsentwicklers einnehmen.\autocite[Vgl.][S. 444f]{FOERDERER2018} Für den Endkunden selbst ist häufig nicht erkennbar, von wem die Applikation erstellt wurde, da er diese im Sinne des SaaS direkt von PaaS-Anbieter bezieht. \autocite[Vgl.][S. 372]{BEIMBORN2011}

\subsection{SAP Business Technology Platform}



\section{Die deutsche Versicherungsbranche}

\subsection{Definition und Grundprinzipien der Versicherung}

Aufgrund der kontinuierlichen Weiterentwicklung des Versicherungsmarktes und der wachsenden Teilnahme unterschiedlicher Wirtschaftsdisziplinen an der Versicherungswissenschaft gibt es heutzutage eine Vielzahl von Definition für den Begriff der Versicherung. Wird sich bei dem Begriff auf die ökonomischen Gesichtspunkte konzentriert, wird in der Literatur insbesondere auf den Wirtschaftswissenschaftler Prof. Dr. Dieter Farny verwiesen, welcher das Versicherungsprodukt, auch Versicherung oder Police genannt, definiert als die: 

\begin{center}
    \textit{\enquote{Deckung eines im Einzelnen ungewissen, insgesamt geschätzten Mittelbedarfs auf der Grundlage des Risikoausgleichs im Kollektiv und in der Zeit.}} \autocite[S. 8f.]{FARNY2011}
\end{center}

Folglich ist die Versicherung ein Produkt zur Deckung von Sicherheitsbedürfnissen, indem es Risiken auf eine Gefahrengemeinschaft, das Kollektiv überträgt. Die Befriedigung dieser Sicherheitsbedürfnisse umfassen die Absicherung von materiellen als auch beruflichen Risiken und sind auf der zweiten Ebene der Maslowschen Bedürfnispyramide zu finden. \autocite[Vgl.][S. 30]{BECKER2019} Das zugrunde liegende Konzept wird als Risikotransfer bezeichnet, bei dem der Versicherungsnehmer gegen Zahlung einer Prämie das Risiko auf den Versicherer überträgt. Beim Eintreten eines entsprechenden Schadens, dem Versicherungsfall, erhält der Versicherungsnehmer von der Versicherung einen Schadensausgleich. Dieser Risikoausgleichseffekt wird von Versicherungsunternehmen genutzt, um die systematische Übernahme von Risiken mit einem im Hinblick auf die Gewinnmöglichkeiten akzeptablen unternehmerischen Risiko durchzuführen. \autocite[Vgl.][S. 9]{FARNY2011}
Grundvoraussetzung ist hierbei, dass der Umfang der Schäden statistisch abschätzbar und dadurch der benötigte Beitrag jedes Mitglieds des Kollektivs mathematisch bestimmbar ist. Aufgrund dieser Kombinationen von Risikotransfer und Prämienzahlung werden Versicherungsunternehmen(VU) auch als Finanz- und Risikointermediäre bezeichnet. \autocite[Vgl.][S. 53]{ZWACK2017}

Grundsätzlich wird zwischen zwei Arten von Versicherungsunternehmen unterschieden: Erstversicherer und Rückversicherer. Erstere schließen ausschließlich Versicherungsgeschäfte mit gewerblichen Unternehmen, privaten und öffentlichen Haushalten ab, währenddessen Rückversicherer das daraus resultierende Risiko der Erstversicherer übernehmen.\autocite[Vgl.][S. 240f.]{FARNY2011} Im Rahmen dieser Arbeit werden insbesondere Kfz-Versicherer betrachtet, welche der Gruppe der Erstversicherer zuzuordnen sind.

Darüber hinaus sind deutsche Erstversicherer durch verschiedene regulatorische Anforderungen wie dem Versicherungsaufsichtsgesetz (VAG) und der Solvency II-Richtlinie der Europäischen Union verpflichtet, um die Stabilität und Integrität des Versicherungsmarktes zu wahren. \autocite[Vgl.][]{BAFIN2016} Eine der fundamentalen Vorschriften ist die Spartentrennung nach § 6 II VAG. Demnach müssen Erstversicherungsunternehmen getrennte Geschäftsbereiche für Lebens-, Kranken- und Kompositversicherungen führen. 

Zu Gruppe der Kompositversicherungen, welche auch als Schaden- und Unfallversicherung bezeichnet wird, zählen seit dem 30.06.1990 alle Versicherungen, die nicht zur Lebens- und Krankenversicherung gehören. \autocite[Vgl.][S. 241-243]{FARNY2011} 


\begin{figure}[h]
    \centering
    \includegraphics[width=1\textwidth]{img/Struktur_VKonzern2.jpg}
    \caption[Struktur eines Versicherungskonzerns]{Struktur eines Versicherungskonzerns\autocite{StVKonzern}}
    \label{fig:StVKonzern}
\end{figure}
\footnotetext{Vgl. eigene Darstellung angelehnt an: Nguyen, 2013, S.172}


In der Praxis führt das Spartentrennungsprinzip häufig zur Bildung von größeren Mutterkonzernen, bei denen eine Holding über mehrere rechtlich selbstständige Versicherungsunternehmen verfügt. (siehe Abbildung \vref{fig:StVKonzern}). Eine beispielhafte Konzernstruktur der AXA SE ist dem Anhang \ref{fig:AXAKstr} zu entnehmen. 

\subsection{Aufbau und Besonderheit der Kfz-Versicherungssparte}

Die Kraftfahrtversicherung ist mit einem Volumen von 29 Milliarden € in 2021 die größte Sparte in der Schadens- und Unfallversicherung.\autocite[Vgl.][]{GDVSUV} Zu ihr gehören die Kfz-Haftpflichtversicherung, die Fahrzeugversicherung, welche aus der Voll- und Teilkaskoversicherung besteht, sowie die Kraftfahrtunfallversicherung.\autocite[Vgl.][S. 8]{MURINGER2000}

Wie in Abbildung \vref{fig:KfzVVBestand} zu erkennen, macht die Kfz-Haftpflicht- versicherung den größten Teil der Kfz-Versicherung aus. Sie gehört zu den Pflichtversicherungen und deckt den Schaden, der durch die Fahrzeugverwendung gegenüber einem Dritten entsteht. \autocite[Vgl.][S. 81]{STADLER2008}

\begin{figure}[h]
    \centering
    \includegraphics[width=0.9\textwidth]{img/KfzV_Bestände_an_Verträgen_2021.jpg}
    \caption[Bestände an Verträgen in der Kfz-Versicherung in Deutschland 2021]{Bestände an Verträgen in der Kfz-Versicherung in Deutschland 2021\autocite{KfzVVBestand}}
    \label{fig:KfzVVBestand}
\end{figure}
\footnotetext{Vgl. eigene Darstellung angelehnt an: GDV - Gesamtverband der Versicherer, 2022.}

%Grafik für den Wettbewerbsfaktor: Rechte Seite: Marktanteile der Versicherungsunternehmen wie Allianz und Co.

Die Fahrzeugversicherung, welche umgangssprachlich auch als Kaskoversicherung bezeichnet wird, ist in die Fahrzeugteil- und die Fahrzeugvollversicherung untergliedert. Der wesentliche Unterschied zwischen den beiden Versicherungsarten besteht in dem Leistungsumfang. So sind bei der Fahrzeugteilversicherung ausschließlich, die durch Brand, Diebstahl, Elementarereignisse und Wildschäden verursacht wurden, gedeckt. Bei der Vollversicherung werden darüber hinaus auch Schäden abgedeckt, die durch den Versicherungsnehmer selbst verursacht werden.\autocite[Vgl.][S. 48]{FELTEN2012}

Die Kraftfahrtunfallversicherung oder auch Insassenunfallversicherung genannt, deckt im Falle eines Unfalls Personenschäden von Insassen eines Fahrzeugs ab. Versichert sind alle Unfälle, welche ausschließlich in unmittelbaren Zusammenhang mit dem Gebrauch eines Fahrzeugs entstehen, wie zum Beispiel das Fahren, Ein- und Aussteigen oder das Be- und Entladen.\autocite[Vgl.][S. 6f]{STADLER1998} Dennoch werden aufgrund der großen Deckung der Leistung durch andere Versicherungen, wie in Abbildung \vref{fig:KfzVVBestand} zu erkennen, verhältnismäßig wenige Insassenunfallversicherungen abgeschlossen. So kommt bei einem fremdverschuldeten Unfall die Kfz-Haftpflichtversicherung des Unfallverursachers für alle Unfallopfer auf. Bei selbst verschuldeten Unfällen sind die Mitfahrer durch die Kfz-Haftpflichtversicherung des Fahrers bzw. des Halters abgesichert. Somit ist das Abschließen einer Insassenunfallversicherung vor allem zum Schutz des Fahrers selbst sinnvoll.\autocite[Vgl.][S. 173f]{LAMMERS2006} In der Praxis wird hierfür allerdings häufig auf andere Alternativen wie die private Unfallversicherung zurückgegriffen.\autocite[Vgl.][]{GRATZLA2018}  

Darüber hinaus gibt es in der Kfz-Versicherungssparte einen enormen Konkurrenzdruck. So versuchen die einzelnen Anbieter insbesondere im Herbst neue Kunden zu gewinnen, da die Verträge der Versicherungsnehmer in der Regel zum 31.12 eines jeden Jahres enden und folglich bis zum 30.11 eines jeden Jahres gekündigt werden können.\autocite[Vgl.][]{WARENTEST2022} Des Weiteren veröffentlich der Gesamtverband der deutschen Versicherer jedes Jahr im Herbst ausgehend von der Unfallhäufigkeit der verschiedenen Fahrzeugtypen, die Preiskategorien für die einzelnen Typklassen. Daraufhin passen die meisten Versicherer ihre Beiträge an, was wiederum ihren Kunden ermöglicht, von einem Sonderkündigungsrecht Gebrauch zu machen.\autocite[Vgl.][]{NUS2022} Diese beiden Faktoren führen dazu, dass die Versicherer insbesondere im Herbst versuchen, neue Kunden mit Sonderangeboten und Rabatten für sich zu gewinnen. 

Dabei sind die Kunden oftmals bereits neben der Kfz-Versicherung weitere Policen beim gleichen Anbieter abzuschließen. Um von diesem Interesse und der Wechselbereitschaft der Kunden zu profitieren, sind die Versicherungsunternehmen bereit, kleinere Profite bis hin zu Verlusten bei der Kfz-Versicherung in Kauf zu nehmen.\autocite[Vgl.][]{HARTUNG2019} Dies lässt sich ebenfalls an der Schadensquote der Kfz-Versicherungen erkennen, welche zwischen 2015 und 2021 bei durchschnittlich 96,4 \% lag. Sie stellt das Verhältnis zwischen den Versicherungsleistungen und den Beitragseinnahmen dar.\autocite[Vgl.][]{GDVKFZ}  



\newpage

%%%%%%%%%%%%%%%%%%%%%%%%%%%%%%%%%%%

%%%%%%%%%%%%%%%%%%%%%%%%%%%%%%%%%%%
% Vorgehensweise
%
% @stud: einzelne Kapitel bearbeiten und eigene Kapitel hier einfügen
%
\clearpage
\chapter{Vorgehensweise}
\section{Task-Technology-Fit Theorie}



%Das Ziel dieser Projektarbeit ist es, die Eignung der SAP BTP als digitale Plattform für Kfz-Versicherer zu untersuchen. Hierfür sollen zunächst die Anforderungen der Kfz-Versicherer an technische Plattformen identifiziert und anschließend mit den Funktionen und Services der SAP Business Technology Platform verglichen werden.


%Zur Unterstützung dieser Analyse wird das Task-Technology-Fit (TTF) Modell (TTF) von Goodhue und Thompson angewendet, welches in Abbildung \ref{fig:TTF} dargestellt ist. Es vergleicht die Charakteristika einer bestimmten Aufgabe mit den Charakteristika einer bestimmten Technologie, die zur Erfüllung dieser Aufgabe verwendet werden soll. Gemäß Goodhue et. al (1995) kann eine Technologie nur dann eine positive Auswirkung auf die Leistung von Einzelpersonen oder Organisationen haben, wenn eine Übereinstimmung zwischen den Funktionalitäten der Technologie und den Anforderungen der Nutzer besteht. Die Leistungen werden umso positiver beeinflusst, je besser die Technologie mit der zu unterstützenden Aufgabe übereinstimmt.\autocite[Vgl.][S. 214-216]{GOODHUE1995}

Zur Untersuchung der Eignung der SAP \c{btp} als digitale Plattform für \ac{kfz}-Versicherer wird in dieser Arbeit das Task-Technology-Fit-Modell (\acs{ttf}-Modell) von Goodhue und Thompson angewendet, welches in Abbildung \ref{fig:TTF} dargestellt ist. Es vergleicht die Charakteristika einer bestimmten Aufgabe mit den Charakteristika einer bestimmten Technologie, die zur Erfüllung dieser Aufgabe verwendet werden soll. Gemäß Goodhue et. al (1995) kann eine Technologie nur dann eine positive Auswirkung auf die Leistung von Einzelpersonen oder Organisationen haben, wenn eine Übereinstimmung zwischen den Funktionalitäten der Technologie und den Anforderungen der Nutzer besteht. Die Leistungen werden umso positiver beeinflusst, je besser die Technologie mit der zu unterstützenden Aufgabe übereinstimmt.\autocite[Vgl.][S. 214-216]{GOODHUE1995}


%\autocite[Vgl.][S. 215]{GOODHUE1995}
%\autocite[Vgl.][S. 399]{SPIES2020}

\begin{figure}[h]
    \centering
    \includegraphics[width=0.8\textwidth]{img/TTF_einfach.jpg}
    \caption[Modell der Task-Technology-Fit Theorie]{Modell der Task-Technology-Fit Theorie\autocite{TTF}}
    \label{fig:TTF}
\end{figure}
\footnotetext{Vgl. eigene Darstellung angeleht an: Spies et al., 2020, S. 399.}
%\footnotetext{Vgl. eigene Darstellung angelehnt an: Goodhue und Thompson, 1995, S. 215.}


Das \acs{ttf}-Modell kann auf jeder Abstraktionsebene angewendet werden, da die aus der Anforderungsübereinstimmung resultierenden Produktivitätsverbesserungen nicht auf einzelne Personen beschränkt sind und auch für ganze Teams oder komplette Organisationen auftreten können.\autocite[Vgl.][S. 1827f]{GOODHUE1995b} 


%Hierbei kann die TTF-Analyse auf verschiedenen Abstraktionsebenen durchgeführt werden, da die Vorteile der Verbesserung der Produktivität nicht nur auf Einzelpersonen, sondern auch auf ganze Organisationen übertragen werden können.\autocite[Vgl.][S. 1827f]{GOODHUE1995b} Eine verbesserte Leistung kann gemäß TTF auf die reibungslose Ausführung der Aufgabe, die Verringerung der Kosten für die Ausführung der Aufgabe oder die Erleichterung der Aufgabe zurückzuführen. \autocite[Vgl.][S. 96]{LEE2007}

Die Task Charakteristika beziehen sich dabei auf die Gesamtheit der physischen und kognitiven Handlungen und Prozesse, welche von einer Organisation oder einer Einzelperson in einer bestimmten Umgebung ausgeführt werden. Sie werden speziell in Bezug zur Technologie, welche sie bei der Ausführung unterstützen soll, betrachtet und je nach Komplexität auf unterschiedliche Detailebenen heruntergebrochen. \autocite[Vgl.][S. 398]{SPIES2020} Nach Goodhue (1998) können die zur Evaluation der Technologie notwendigen Anforderungen mithilfe eines Task-Modells, einer Literaturrecherche oder mittels Interviews erhoben werden.\autocite[Vgl.][S. 126]{GOODHUE1998} Eine Anforderung wird in diesem Kontext definiert als eine Aussage, \enquote{die einen Bedarf und die damit verbundenen Einschränkungen und Bedingungen darstellt und erläutert}.\autocite[Vgl.][]{ISO2017}

Innerhalb des \acs{ttf}-Modells beziehen sich die Technologie Charakteristika auf die Werkzeuge, welche von Einzelpersonen zur Ausführung ihrer Aufgaben verwendet werden oder diese bei der Ausführung ihrer Aufgaben unterstützen.\autocite[Vgl.][S. 399]{SPIES2020} Dabei kann sowohl der Einfluss eines einzelnen Systems, als auch die Wirkung einer Gesamtheit von bereitgestellten Systemen und Diensten betrachtet werden. \autocite[Vgl.][S. 216]{GOODHUE1995}

%Als Technologie Charakteristiken werden im Rahmen des TTF Modells die Werkzeuge bezeichnet, welche von Einzelpersonen zur Ausführung ihrer Aufgaben oder zur Unterstützung bei der Ausführung ihrer Aufgaben verwendet werden sollen.\autocite[Vgl.][S. 216]{GOODHUE1995} Dabei ist das Modell so allgemein gehalten, dass es sich entweder auf die Einfluss eines bestimmten Systems oder auf die umfassendere Wirkung der Gesamtheit der bereitgestellten Systeme, Strategien und Dienste ausgerichtet sein kann. \autocite[Vgl.][S. 399]{SPIES2020}




\section{Systematische Literaturanalyse}


Im Rahmen des \acs{ttf}-Models wurde eine systematische Literaturanalyse zur Identifikation der Anforderungen der Kfz-Versicherer an digitale Plattformen durchgeführt. Hierfür wurden zunächst die für die Task Charakteristika relevanten Suchbegriffe auf Deutsch und Englisch festgelegt, siehe Tabelle \ref{tab:suchbegriffe} im Anhang. Daraufhin wurde mithilfe dieser Suchbegriffe die Datenbanken Google Scholar, \ac{ebsco} Discovery Service (aufgerufen über die Metasuche der DHBW Mannheim), \ac{jstor} sowie Google nach Fachliteratur, internationale und nationale Zeitschriften, Studien, Magazinen und Internetartikeln  durchsucht. Die dabei angewendeten Suchkriterien sind in Tabelle \ref{tab:kriterien} im Anhang aufgezeigt. Grundlage für die Betrachtung der Recherche-Ergebnisse waren der Titel, die Kurzfassung, die Gliederung, die Einleitung sowie die Zusammenfassung der jeweiligen Quelle. Dabei wurden die ausgewählte Literatur zur Identifikation der Task Charakteristika als Ganzes oder in Ausschnitten gelesen und analysiert. \autocite[Vgl.][]{SOLIS2021}

\improvement{Optional: Quelle austauschen}




\section{Semistrukturiertes Leitfadeninterview und qualitative Inhaltsanalyse}

Im Rahmen des \ac{ttf}-Models werden in dieser Abhandlung zur Validierung, Erweiterung und Priorisierung der in Literatur gefundenen Task-Charakteristika Experteninterviews als qualitative Forschungsmethode eingesetzt. Die Methode ermöglicht das Erschließen bisher unbekannter Aspekte sowie das induktive Erarbeiten von Schlussfolgerungen. Experteninterviews eignen sich insbesondere für explorative Fragestellungen, da sie helfen, das Forschungsfeld zu strukturieren und zu präzisieren. Als Experten können diejenigen Personen betrachtet werden, die durch ihre Funktion oder Tätigkeit spezielles Sonderwissen erwerben konnten. \autocite[Vgl.][S. 119-127]{MISOCH2019} Die Wahl der Experten ist dem Anhang \ref{sec:Expertenwahl} zu entnehmen.

Um ein Experteninterview systematisch zu strukturieren, wird im Voraus vom Interviewer ein Leitfaden, bestehend aus geschlossenen und offenen Fragen sowie optional auch Stichpunkten ausgearbeitet und den Experten zur Verfügung gestellt \autocite[Vgl.][S. 670]{HELFFERICH2019}. Die Kombination von offenen und geschlossenen Fragen ermöglicht es, zum einen ein umfassendes Bild über die Ansichten des Experten zu gewinnen und zum anderen quantitative Aussagen zu erheben, die mit anderen Expertenmeinungen verglichen werden können. Der in der Arbeit verwendete Leitfaden ist thematisch strukturiert, halbstandardisiert, nach den Grundprinzipien von Misoch in vier Phasen untergliedert und im Anhang unter \ref{sec:Fragenkatalog} zu finden \autocite[Vgl.][S. 68f]{MISOCH2019}.  

Für die Auswertung der erhobenen Daten wird die qualitative Inhaltsanalyse nach Mayring verwendet, bei der ein Kategoriensystem deduktiv aus der Literatur und induktiv aus den Antworten der Experten definiert wird \autocite[Vgl.][S. 633-634]{MAYRING2019}. Anhand der Kategorien können Aussagen aus den Interviews zusammengeführt werden, um eine Gesamtanalyse aller Interviews durchzuführen \autocite[Vgl.][S. 74]{BOGNER2014}. 




%%%%%%%%%%%%%%%%%%%%%%%%%%%%%%%%%%%

%%%%%%%%%%%%%%%%%%%%%%%%%%%%%%%%%%%
% Vorgehensweise
%
% @stud: einzelne Kapitel bearbeiten und eigene Kapitel hier einfügen
%
\chapter{Evaluation der SAP Business Technology Platform für die Anforderungen auf dem deutschen Kfz-Versicherungsmarkt}

\section{Task Charakteristiken - Identifikation der Anforderungen der Kfz-Versicherer an digitale Plattformen}

\subsection{Literaturbetrachtung aktueller Anforderungen an digitale Plattformen}

Nach jeder Anforderung (eine Zahl als eine Art Aufzählung einführen)



\subsection{Anforderungen aus Sicht der Experten}

\section{Technologie Charakteristiken - Komponenten und Services der SAP Business Technology Platform}

\section{Synthese der Task und Technology Charakteristiken}



%%%%%%%%%%%%%%%%%%%%%%%%%%%%%%%%%%%

%%%%%%%%%%%%%%%%%%%%%%%%%%%%%%%%%%%
% Vorgehensweise
%
% @stud: einzelne Kapitel bearbeiten und eigene Kapitel hier einfügen
%
\chapter{Handlungsempfehlung}

\improvement{Idee Bernd: BTP ist vielleicht nicht die beste Lösung am Markt, aber bietet alle Bausteine, da kann man Erfahrung sammeln und später auf ein noch besseres Produkt wechseln}

Die Task-Technology-Fit Analyse hat gezeigt, dass die SAP Business Technology Platform zu 95\% die Anforderungen der Kfz-Versicherer erfüllt und somit eine vielversprechende Plattform für die Versicherungsbranche und den Bereich der Kfz-Versicherung darstellt.

Um durch den Einsatz der SAP BTP Wettbewerbsvorteile erzielen zu können, sollten Kfz-Versicherer die einzelnen Services der SAP BTP schrittweise implementieren. Dabei hat sich anhand der Priorisierung der Anforderungen durch die Experten gezeigt, dass neben den grundlegenden Anforderungen wie Datensicherheit und Skalierbarkeit vor allen Dingen die Anforderungen im Bereich der Integration besonders wichtig sind.

(1)Hier sollten Kfz-Versicherer im Rahmen einer SOA ihre bestehenden Backendsysteme kapseln und diese an die BTP anbinden, um die dort vorgehaltenen Daten und Prozessabläufe für andere Anwendungen zugänglich zu machen. Sofern die Legacy-Systeme der Kfz-Versicherer hinreichend eingebunden sind, sollten die Schnittstellen der BTP zur Anbindung an Mobilitätsökosysteme, sowie andere Vertriebskanäle genutzt werden. Dadurch könnten sie ihren Kundenstamm erweitern und folglich ihre Marktposition weiter ausbauen. 

(2) Darauf aufbauend sollten Kfz-Versicherer über die BTP eine App bereitstellen, mit der Versicherungsnehmer alle ihre Kfz-Versicherungsaktivitäten, wie das Vereinbaren eines Werkstatttermins, zentral steuern können, da eine solche Anwendung insbesondere von der jungen Generation heutzutage erwartet wird. (vgl. Eduard 0:9:03)  Dabei können Kfz-Versicherer die dafür notwendige App auch auf der BTP selbst mithilfe des Mobile Services entwickeln.

(3) Zur Erweiterung des eigenen Angebotsportfolios sollten Kfz-Versicherer darüber hinaus die IoT Services der SAP BTP nutzen. Um Sensordaten zum Fahrverhalten von Kunden bei der Prämienkalkulation zu berücksichtigen und damit weitere Finanzdienstleistungen wie z.B. passgenaue Telematik Tarife anbieten zu können.

(4) Um die Entwicklung und Erweiterung der oben aufgeführten Applikationen möglichst schnell realisieren zu können, sollten Kfz-Versicherer neben den klassischen Entwicklungstools auch die Low-Code No-Code Werkzeuge von SAP Build Apps nutzen. Mit diesen könnten Fachexperten Applikationen anpassen, ohne tiefe Programmierkenntnisse mitzubringen. Damit könnten Kfz-Versicherer die Zusammenarbeit zwischen IT und Geschäftsabteilung in funktionsübergreifenden Teams fördern und gleichzeitig den eigenen Bedarf an Entwicklern reduzieren bzw. die IT-Mitarbeiter entlasten.

(5) Nach dem die Anbindung der verschiedenen Altsysteme an die BTP abgeschlossen ist, sollten Kfz-Versicherer mit der SAC Analytics Cloud die historischen Daten der Kunden mit der SAP Analytics Cloud analysieren, um die Prämien- und Risikokalkulation zu optimieren. Dabei sollten für die dafür benötigten Daten zunächst mit dem Data Intelligence Service aus den verschiedenen Quellen aggregiert und mit dem Datasphere Service aufbereitet werden.

(6) Weiterhin sollten Kfz-Versicherer nach der Durchführung der Schritte 1 und 2 die repetitiven manuellen Prozesse wie das Erstellen von Berichten oder das Überprüfen von Kfz-Schäden mit dem BTP Service SAP Build Process Automation automatisieren, um die Bearbeitungszeiten zu reduzieren und mehr Ressourcen für die Kundenbetreuung zu haben.








%%%%%%%%%%%%%%%%%%%%%%%%%%%%%%%%%%%

%%%%%%%%%%%%%%%%%%%%%%%%%%%%%%%%%%%
% Vorgehensweise
%
% @stud: einzelne Kapitel bearbeiten und eigene Kapitel hier einfügen
%
\chapter{Fazit und Ausblick}

\section{Zusammenfassung und Fazit}

Das Ziel dieser Projektarbeit bestand darin, die Anforderungen der Kfz-Versicherer an eine digitale Plattform zu identifizieren und zu untersuchen, inwieweit die SAP Business Technology Plattform diese Anforderungen erfüllt. Die technologische- wirtschaftswissenschaftliche Auseinandersetzung im Rahmen des Task- Technology-Fit-Modells erlaubt nun die zureichende Beantwortung dieser Fragestellung.

Als Task Charakteristika konnten mithilfe der systematischen Literaturanalyse zunächst 16 Anforderung der Kfz-Versicherer an digitale Plattformen identifiziert und diese in die Bereiche: Integration, Entwicklung, Datenverarbeitung, Prozessautomatisierung sowie grundlegende Anforderungen unterteilt werden. Um die Richtigkeit und Wichtigkeit der einzelnen Anforderungen sicherzustellen, wurden stets mehrere Quellen miteinander verglichen und zur Validierung der Anforderungen drei Experteninterviews geführt. Dabei wurden alle 16 Anforderungen von den Experten bestätigt und zudem vier weitere benannt. Die anschließende Priorisierung der Anforderungen durch die Experten hat gezeigt, dass bei Erfüllung von grundlegenden Anforderungen wie Datensicherheit und Skalierbarkeit, digitale Plattformen bei Kfz-Versicherern in den nächsten Jahren insbesondere für Integrations- und Entwicklungsaufgaben eingesetzt werden. Nach der Darstellung der wesentlichen Funktionen und Services der SAP BTP aus den Bereichen Datenmanagement, Datenanalyse, Integration, Entwicklung sowie intelligente Technologien konnte die Synthese der Task und Technologie Charakteristika durchgeführt werden. Diese hat gezeigt, dass die SAP BTP 19 der 20 Anforderungen erfüllt und damit eine geeignete digitale Plattform für Kfz-Versicherer darstellt. 

Dieses Ergebnis ermöglichte es unter Berücksichtigung der Priorisierung der Anforderungen, Kfz-Versicherern eine gezielte Vorgehensweise zur Realisierung der technischen Möglichkeiten der BTP zu empfehlen und zu zeigen, wie die BTP den Kfz-Versicherern hilft, Innovationen zu beschleunigen und Unternehmenspotentiale zu realisieren.

Susanne: \improvement{Susanne fragen, was sie machen würde}

Aufgrund dieses Ergebnisses konnte unter Berücksichtigung der priorisierten Anforderungen eine zielgerichtete Vorgehensweise zur Umsetzung der technischen Möglichkeiten der BTP für Kfz-Versicherer empfohlen werden. Es wurde dargestellt, wie die BTP dazu beiträgt, Innovationen zu beschleunigen, Unternehmenspotenziale zu erschließen und dadurch Wettbewerbsvorteile für Kfz-Versicherer zu erzielen.

%\newpage
\section{Kritische Reflexion und Ausblick}

Einschränkend sind nachfolgende Aspekte kritisch zu reflektieren: Die ausgewählten Experten verfügten alle über langjährige Erfahrung in der Kfz-Versicherungs- branche sowie über ein breites Fachwissen zu digitalen Plattformen. Dabei konnten im Rahmen dieser Untersuchung jedoch ausschließlich Experten von SAP und SAP Fioneer befragt werden. Um ein ganzheitliches Bild der Anforderungen der Kfz-Versicherer zu bekommen, könnte in Folgeuntersuchungen ebenfalls Technologieexperten auf Seiten der Kfz-Versicherer befragt werden. 

Des Weiteren wurden die von den Experten ergänzten Anforderungen bisher noch nicht von allen Experten priorisiert. Dies sollte in einer weiteren Expertenbefragung noch nachgeholt werden, um eine Gewichtung über alle Experten zu bekommen. 

\improvement{Zu negativ: Komination aus Literaturrecherche, Experteninterviews und Gartner hat sich bewehrt und gibt ein gutes Ergebnis, deswegen sind die Ergebnisse sehr repräsentativ}

Darüber hinaus wurden aufgrund des begrenzten Umfangs einer Projektarbeit die Funktionalitäten, der SAP BTP abstrahiert und Anforderungen der Kfz-Versicherer teilweise zu größeren umfangreicheren Anforderungen subsumiert. In der Handlungsempfehlung wird allgemein dargestellt, wie die Funktionalitäten der SAP BTP zielgerichtet für Versicherer genutzt und implementiert werden können. Später muss aber für den jeweiligen Einzelfall entschieden werden, welche Implementierungsreihenfolge und Servicenutzung für einen Kunden unter Betrachtung der bestehenden IT-Landschaft sowie der individuellen Kundenbedürfnisse im Detail am sinnvollsten ist.

In weiteren Forschungsarbeiten könnte darüber hinaus untersucht werden, wie sich der Trend der digitalen Ökosysteme auf die Kfz-Versicherungsbranche auswirkt und welche Rolle dabei insbesondere die in dieser Arbeit untersuchten digitalen Plattformen spielen.



%%%%%%%%%%%%%%%%%%%%%%%%%%%%%%%%%%%

%List of Todos von Alexander Rong übernommen
\listoftodos[Notes]

%\initializeAppendix %original place EditMaxD

%%%%%%%%%%%%%%%%%%%%%%%%%%%%%%%%%%%
% LITERATURVERZEICHNIS
% 
% @stud: Literaturverzeichnis in Datei bibliography.bib anpassen 
%
%\raggedright
\initializeBibliography
%%%%%%%%%%%%%%%%%%%%%%%%%%%%%%%%%%%

%%%%%%%%%%%%%%%%%%%%%%%%%%%%%%%%%%%
% ANHÄNGE
\initializeAppendix
%
% @stud: einzelne Anhänge bearbeiten und eigene Anhänge hier einfügen 
% 
\addtocontents{toc}{\protect\enlargethispage{2\normalbaselineskip}}
% !TEX root =  master.tex
\chapter{Anhang}\label{anhang}

\section{Konzernstruktur großer Versicherer}
\label{sec:KonzernStrukturen}

\begin{figure}[h]
  \centering
  \includegraphics[width=1\textwidth]{img/beteiligungsstruktur-axa-konzern.pdf}
  \caption[]{AXA SE - Konzernstruktur (AXA Deutschland)}
  \label{fig:AXAKstr}
\end{figure}
% \caption[]{}{AXA SE - Konzernstruktur (AXA Deutschland)}
%wie bekomme ich es hin dass die nicht im Abbildungsverzeichnis auftaucht?
%%%%%%%%%%%%%%%%%%%%%%%%%%%%%%%%%%%
%\input{img/abap_performance_program.tex}
%%%%%%%%%%%%%%%%%%%%%%%%%%%%%%%%%%%
% EHRENWÖRTLICHE ERKLÄRUNG
%
% @stud: ewerkl.tex bearbeiten
%
\input{template/ewerkl} 
%%%%%%%%%%%%%%%%%%%%%%%%%%%%%%%%%%%

%\addcontentsline{toc}{chapter}{Index}
%\printindex                                        % fügt die Indexdatei mit den gemachten Einträgen am Ende des Dokumentes hinzu

\end{document}
