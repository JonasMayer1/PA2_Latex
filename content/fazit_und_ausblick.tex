\chapter{Fazit und Ausblick}

Im Rahmen dieser Projektarbeit konnte die Suchoberfläche der Spezifikation im \ac{plm}-Bereich erfolgreich um einen Kundenwunsch aus dem SAP Customer Connection Programm erweitert werden. 

Dabei wurde mit der Referenzmodellierung eine geeignete wissenschaftliche Methodik gefunden, um den Vorschlag umzusetzen. Die Erkenntnisse der Referenzmodellanalyse konnten bei der Verwirklichung des Kundenwunsches als solide Grundlage wiederverwendet werden und ermöglichten es, den Entwicklungsprozess zu beschleunigen.


%Nichts desto trotz, sind bei der Umsetzung des Kundenwunsches kleine Herausforderungen aufgetreten. Bei dem Testen der Änderungen, hat ein Eingabefeld aufgrund von unpassenden Struktur-Komponenten nicht funktioniert. Mit Hilfe des ABAP Debuggers konnte die Ursache der Probleme gefunden werden und anschließend die dazugehörigen Lösungen erfolgreich umgesetzt werden. 

Abschließend bleibt festzustellen, dass durch die Umsetzung des beschriebenen Kundenwunsches positiv Einfluss auf die Zufriedenheit der Kunden genommen wurde. Die in der Arbeit vorgestellten Änderungen erleichtern den Kunden ihren Arbeitsablauf im \acl{plm}. Durch die Vielzahl an Suchanfragen, die regelmäßig im \ac{plmwui} gestartet werden, hatte die Umsetzung des Kundenwunsches außerdem einen klaren betriebswirtschaftlichen Mehrwert.

Da sich der Softwaremarkt rasant weiterentwickelt und die Zukunft von Unternehmen in der Softwarebranche von Innovationen abhängig ist, sollte auch in Zukunft kontinuierlich mit dem Kunden über das Customer Connection Programm zusammengearbeitet werden. Das Programm bietet eine hervorragende Möglichkeit, Optimierungspotentiale wie z. B. noch fehlende Funktionalität zu identifizieren und SAP Applikationen gemäß den Marktanforderungen anzupassen. 

Um die \ac{plmwui}-Suche kundenorientiert weiterzuentwickeln, könnte zu dem geprüft werden, ob die in der Projektarbeit thematisierten Änderungen auch bei anderen \ac{plm}-Objekten, wie zum Beispiel der Zugriffskontrolle umgesetzt werden sollten.


\begin{comment}
Die Zukunft von Unternehmen in der Softwarebranche ist von Innovationen, wie sie durch das Customer Connection Programm eingereicht werden können, abhängig, da sich der Softwaremarkt rasant weiterentwickelt. Folglich müssen die SAP eigenen Produkte kontinuierlich kundenorientiert weiterentwickeln werden, da Stillstand im unternehmerischen Kontext immer Rückschritt bedeutet.

Um das \ac{plmwui} auch in Zukunft weiterzuentwickeln, können die in der Projektarbeit thematisierten Änderungen auch bei anderen \ac{plm}-Objekt, wie zum Beispiel der Zugriffskontrolle umgesetzt werden. Des Weiteren sollte also auch in Zukunft eng mit dem Kunden zusammengearbeitet werden, damit die Wettbewerbsfähigkeit des \ac{plmwui}s gesichert ist.



Damit die Wettbewerbsfähigkeit am Markt gesichert ist und sich das \ac{plmwui} kontinuierlich weiterentwickeln kann, sollte also auch in Zukunft eng mit dem Kunden zusammengearbeitet werden. Dafür könnten die in der Projektarbeit thematisierten Änderungen bei anderen PLM-Objekten, wie der Zugriffskontrolle umgesetzt werden.

%Was wurde nicht erreicht ? --> Nichts ---> Auf die volle Umsetzung des Kundenwunsches aufmerksam machen, Bezug zu den Punkten im Kundenwunsch herstellen
% Danach den ABlauf eines CCR und die Kriterien nennen
% Dann Asublick in die Zukunft
% Interpretation


Einleitung und Fazit sind in der Arbeit am Wichtigsten
Reflexion was ich gemacht haben
wissenschaftlich neutral, kein ich oder man
Keine neuen Inhalte, beispiele Zitate
Keine reißerischen Formulierungen wie logischerweise oder selbstverständlich
im Präsens


alles Aufgreifen: 
Bezug zur Einleitung und der Fragestellung herstellen
Erklären wie kam ich zu dem Ergebnis
Zusammenfassen der Ergebnisse; Was habe ich erreicht, was habe ich nicht erreicht; selbstkritsisch
Interpretation der Ergebnisse
Einordnung in den Forschungskontext
Fragestellungen nennen an denen man weiter forschen könnte --> Ausblick, was könnte noch gemacht werden
Gelunger Schlusssatz als Abrundung des Themas sehr wichtig
\end{comment}

