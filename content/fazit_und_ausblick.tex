\chapter{Fazit und Ausblick}

\section{Zusammenfassung und Fazit}

Das Ziel dieser Projektarbeit bestand darin, die Anforderungen der Kfz-Versicherer an eine digitale Plattform zu identifizieren und zu untersuchen, inwieweit die SAP Business Technology Plattform diese Anforderungen erfüllt. Die wirtschafts- wissenschaftliche-technologische Auseinandersetzung im Rahmen des Task- Technology-Fit-Modells erlaubt nun die zureichende Beantwortung dieser Fragestellung.

Als Task Charakteristika konnten mithilfe der systematischen Literaturanalyse zunächst 16 Anforderung der Kfz-Versicherer an digitale Plattformen identifiziert und diese in die Bereiche: Integration, Entwicklung, Datenverarbeitung, Prozessautomatisierung sowie grundlegende Anforderungen unterteilt werden. Um die Richtigkeit und Wichtigkeit der einzelnen Anforderungen sicherzustellen, wurden stets mehrere Quellen miteinander verglichen und zur Validierung der Anforderungen drei Experteninterviews geführt. Dabei wurden alle 16 Anforderungen von den Experten bestätigt und zudem vier weitere benannt. Die anschließende Priorisierung der Anforderungen durch die Experten hat gezeigt, dass bei Erfüllung von grundlegenden Anforderungen wie Datensicherheit, digitale Plattformen bei Kfz-Versicherern in den nächsten Jahren insbesondere für Integrations- und Entwicklungsaufgaben eingesetzt werden. Folgend wurden die wesentlichen Funktionen und Services der SAP BTP aus den Bereichen Datenmanagement, Datenanalyse, Integration, Entwicklung sowie intelligente Technologien vorgestellt. Die darauf aufbauende Synthese der Task und Technologie Charakteristika hat gezeigt, dass die SAP BTP 19 der 20 Anforderungen erfüllt und damit eine geeignete digitale Plattform für Kfz-Versicherer darstellt. Dieses Ergebnis ermöglichte es unter Berücksichtigung der Priorisierung der Anforderungen, Kfz-Versicherern eine gezielte Vorgehensweise zur Realisierung der technischen Möglichkeiten der BTP zu empfehlen und zu zeigen, wie die BTP den Kfz-Versicherern hilft, Innovationen zu beschleunigen und Unternehmenspotentiale zu realisieren.

\newpage
\section{Kritische Reflexion und Ausblick}

\improvement{darauf achten, dass nichts über den Rand geht}

Einschränkend sind nachfolgende Aspekte kritisch zu reflektieren: Die ausgewählten Experten verfügten alle über langjährige Erfahrung in der Kfz-Versicherungs- branche sowie über ein breites Fachwissen zu digitalen Plattformen. Dabei konnten im Rahmen dieser Untersuchung jedoch ausschließlich Experten von SAP und SAP Fioneer befragt werden. Um ein ganzheitliches Bild der Anforderungen der Kfz-Versicherer zu bekommen, könnte in Folgeuntersuchungen ebenfalls Technologieexperten auf Seiten der Kfz-Versicherer befragt werden. 

Des Weiteren wurden die von den Experten ergänzten Anforderungen bisher noch nicht von allen Experten priorisiert. Dies sollte in einer weiteren Expertenbefragung noch nachgeholt werden, um eine Gewichtung über alle Experten zu bekommen. 

Darüber hinaus wurden aufgrund des begrenzten Umfangs einer Projektarbeit die Funktionalitäten, der SAP BTP abstrahiert und Anforderungen der Kfz-Versicherer teilweise zu größeren umfangreicheren Anforderungen subsumiert. In der Handlungsempfehlung wird allgemein dargestellt, wie die Funktionalitäten der SAP BTP zielgerichtet für Versicherer genutzt und implementiert werden können. Später muss aber für den jeweiligen Einzelfall entschieden werden, welche Implementierungsreihenfolge und Servicenutzung für einen Kunden unter Betrachtung der bestehenden IT-Landschaft sowie der individuellen Kundenbedürfnisse im Detail am sinnvollsten ist.

In weiteren Forschungsarbeiten könnte darüber hinaus untersucht werden, wie sich der Trend der digitalen Ökosysteme auf die Kfz-Versicherungsbranche auswirkt und welche Rolle dabei insbesondere die in dieser Arbeit untersuchten digitalen Plattformen spielen.


