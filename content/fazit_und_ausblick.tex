\chapter{Fazit und Ausblick}

Das Ziel dieser Projektarbeit war es, die Anforderungen der Kfz-Versicherer an eine digitale Plattform zu identifizieren und anschließend zu untersuchen, inwiefern diese Anforderungen von der SAP Business Technology Plattform erfüllt werden. Die wirtschaftswissenschaftliche-technologische Auseinandersetzung im Rahmen des Task-Technology-Fit-Models erlaubt nun die zureichende Beantwortung dieser Fragestellung. 

Als Task Charakteristika konnten mithilfe der systematischen Literaturanalyse zunächst 16 Anforderung der Kfz-Versicherer an digitale Plattformen identifiziert und diese in die Bereiche: Integration, Entwicklung, Datenverarbeitung, Prozessautomatisierung sowie grundlegende Anforderungen unterteilt werden. Um die Richtigkeit und Wichtigkeit der einzelnen Anforderungen sicherzustellen, wurden stets mehrere Quellen miteinander verglichen und zur Validierung der Anforderungen 3 Experteninterviews geführt. Dabei wurden alle 16 Anforderungen von den Experten bestätigt und zudem 4 weitere benannt. Die anschließende Priorisierung der Anforderungen durch die Experten hat gezeigt, dass bei Erfüllung von grundlegenden Anforderungen wie Datensicherheit, digitale Plattformen bei Kfz-Versicherern in den nächsten 3-5 Jahren insbesondere für Integrationsaufgaben eingesetzt werden. Folgend wurden die wesentlichen Funktionen und Services der SAP BTP aus den Bereichen Datenmanagement, Datenanalyse, Integration, Entwicklung sowie Intelligente Technologien vorgestellt. Die darauf aufbauende Synthese der Task und Technology Charakteristika hat gezeigt, dass die SAP BTP 19 der 20 Anforderungen erfüllt und damit eine geeignete digitale Plattform für Kfz-Versicherer darstellt. Dieses Ergebnis ermöglichte es unter Berücksichtigung der Priorisierung der Anforderungen, Kfz-Versicherern eine gezielte Vorgehensweise zur Realisierung der technischen Möglichkeiten der BTP auszusprechen.

Einschränkend sind nachfolgende Aspekte kritisch zu reflektieren: Die ausgewählten Experten verfügten alle über langjährige Erfahrung in der Kfz-Versicherungsbranche sowie über ein breites Fachwissen zu digitalen Plattformen. Dabei konnten im Rahmen dieser Untersuchung jedoch ausschließlich Experten von SAP und SAP Fioneer befragt werden. Um ein ganzheitliches Bild der Anforderungen der Kfz-Versicherer zu bekommen, könnte in Folgeuntersuchungen ebenfalls Technologieexperten auf Seiten der Kfz-Versicherer befragt werden. Dabei sollten die bereits identifizierten Anforderungen noch einmal evaluiert werden, um auch bei den von den Experten ergänzten Anforderungen mehrere Priorisierungen zu erhalten. Darüber hinaus wurden aufgrund des beschränkten Umfangs einer Projektarbeit 2 die Funktionalitäten der SAP BTP abstrahiert und Anforderungen der Kfz-Versicherer teilweise zu größeren umfangreicheren Anforderungen subsumiert. Mögliche Folgearbeiten könnten darüber hinaus auch den Einfluss digitaler Plattformen auf die fünf Wettbewerbskräfte nach Porter betrachten oder die BTP mit anderen digitalen Plattformen vergleichen . Letzteres wurde in dieser Arbeit bewusst nicht untersucht, da aus Sicht der SAP vor allem die Einsatzmöglichkeiten des eigenen Produktes von Bedeutung sind.

%***Punkte bei der kritischen Reflexion restrukturieren in der Reihenfolge: 1. Handlungsempfehlung 2. Umfang 3.Experten 
