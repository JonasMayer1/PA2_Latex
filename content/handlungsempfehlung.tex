\chapter{Handlungsempfehlung}

\improvement{Idee Bernd: BTP ist vielleicht nicht die beste Lösung am Markt, aber bietet alle Bausteine, da kann man Erfahrung sammeln und später auf ein noch besseres Produkt wechseln}

Die Task-Technology-Fit Analyse hat gezeigt, dass die SAP Business Technology Platform zu 95\% die Anforderungen der Kfz-Versicherer erfüllt und somit eine vielversprechende Plattform für die Versicherungsbranche und den Bereich der Kfz-Versicherung darstellt.

Um durch den Einsatz der SAP BTP Wettbewerbsvorteile erzielen zu können, sollten Kfz-Versicherer die einzelnen Services der SAP BTP schrittweise implementieren. Dabei hat sich anhand der Priorisierung der Anforderungen durch die Experten gezeigt, dass neben den grundlegenden Anforderungen wie Datensicherheit und Skalierbarkeit vor allen Dingen die Anforderungen im Bereich der Integration besonders wichtig sind.

(1)Hier sollten Kfz-Versicherer im Rahmen einer SOA ihre bestehenden Backendsysteme kapseln und diese an die BTP anbinden, um die dort vorgehaltenen Daten und Prozessabläufe für andere Anwendungen zugänglich zu machen. Sofern die Legacy-Systeme der Kfz-Versicherer hinreichend eingebunden sind, sollten die Schnittstellen der BTP zur Anbindung an Mobilitätsökosysteme, sowie andere Vertriebskanäle genutzt werden. Dadurch könnten sie ihren Kundenstamm erweitern und folglich ihre Marktposition weiter ausbauen. 

(2) Darauf aufbauend sollten Kfz-Versicherer über die BTP eine App bereitstellen, mit der Versicherungsnehmer alle ihre Kfz-Versicherungsaktivitäten, wie das Vereinbaren eines Werkstatttermins, zentral steuern können, da eine solche Anwendung insbesondere von der jungen Generation heutzutage erwartet wird. (vgl. Eduard 0:9:03)  Dabei können Kfz-Versicherer die dafür notwendige App auch auf der BTP selbst mithilfe des Mobile Services entwickeln.

(3) Zur Erweiterung des eigenen Angebotsportfolios sollten Kfz-Versicherer darüber hinaus die IoT Services der SAP BTP nutzen. Um Sensordaten zum Fahrverhalten von Kunden bei der Prämienkalkulation zu berücksichtigen und damit weitere Finanzdienstleistungen wie z.B. passgenaue Telematik Tarife anbieten zu können.

(4) Um die Entwicklung und Erweiterung der oben aufgeführten Applikationen möglichst schnell realisieren zu können, sollten Kfz-Versicherer neben den klassischen Entwicklungstools auch die Low-Code No-Code Werkzeuge von SAP Build Apps nutzen. Mit diesen könnten Fachexperten Applikationen anpassen, ohne tiefe Programmierkenntnisse mitzubringen. Damit könnten Kfz-Versicherer die Zusammenarbeit zwischen IT und Geschäftsabteilung in funktionsübergreifenden Teams fördern und gleichzeitig den eigenen Bedarf an Entwicklern reduzieren bzw. die IT-Mitarbeiter entlasten.

(5) Nach dem die Anbindung der verschiedenen Altsysteme an die BTP abgeschlossen ist, sollten Kfz-Versicherer mit der SAC Analytics Cloud die historischen Daten der Kunden mit der SAP Analytics Cloud analysieren, um die Prämien- und Risikokalkulation zu optimieren. Dabei sollten für die dafür benötigten Daten zunächst mit dem Data Intelligence Service aus den verschiedenen Quellen aggregiert und mit dem Datasphere Service aufbereitet werden.

(6) Weiterhin sollten Kfz-Versicherer nach der Durchführung der Schritte 1 und 2 die repetitiven manuellen Prozesse wie das Erstellen von Berichten oder das Überprüfen von Kfz-Schäden mit dem BTP Service SAP Build Process Automation automatisieren, um die Bearbeitungszeiten zu reduzieren und mehr Ressourcen für die Kundenbetreuung zu haben.







