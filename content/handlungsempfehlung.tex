\chapter{Handlungsempfehlung}

Nach dem Abgleich der Task und Technologie Charakteristiken konnte festgestellt werden, dass alle Anforderungen der Kfz-Versicherer an eine technische Plattform von der SAP BTP erfüllt wurden. Dabei hat sich anhand der Priorisierung der Anforderungen durch die Experten gezeigt, dass es neben den grundlegenden Anforderungen Datenschutz, Skalierbarkeit, einfache Wartung und Content, bestimmte Kernaktivitäten im Bereich der Integration durchgeführt werden müssen, bevor die Anforderungen in den anderen Bereichen erfüllt werden können. So können Kfz-Versicherer im Rahmen einer SOA ihre bestehenden Backendsysteme kapseln und diese anschließend über die Cloud-Konnektoren an die BTP anbinden, um die dort vorgehaltenen Daten und Prozessabläufe für andere Anwendungen zugänglich zu machen. Dabei kann der API Management Service der BTP komplementär verwendet werden, um die notwendigen Schnittstellenaufrufe zu überwachen/zu verwalten. 

Sind die bestehenden Altsysteme der Versicherer entsprechend integriert, können Kfz-Versicherer die offenen Schnittstellenstandards der BTP dazu verwenden, sich in Mobilitätsökosysteme sowie weitere Vertriebskanäle zu integrieren.

Zudem können die Kfz-Versicherer mithilfe die SAP BTP Zugang zu neuen Technologien in den Bereichen Datenanalyse, Entwicklung und Prozessautomatisierung erhalten, die sie in ihren Altsystemen gar nicht abbilden können. So können sie im Bereich der Datenverarbeitung die unstrukturierten Daten unstrukturierten Daten zunächst aufbereiten und anschließend mit ML-Modellen auswerten, um eigene interne Verbesserungspotentiale zu identifizieren.

Mithilfe der SAP BTP können Kfz-Versicherer zudem ihre digitalen Kundenservice verbessern, indem sie beispielsweise eine mobile App für ihre Versicherungsnehmer bereitstellen, die entsprechend mit den Kooperationspartnern des Versicherers wie Werkstätten integriert ist. So könnten die Versicherungsnehmen im Schadensfall über die App des Versicherers einen Werkstatttermin vereinbaren. Zu den weiteren Diensten die Kfz-Versicherer durch die BTP anbieten können, gehören Telematik-Tarife, da die dafür notwendigen Sensordaten mit der BTP verarbeitet und in Geschäftsprozesse integriert werden können. Darüber hinaus können mit dem Conversational AI Service der SAP BTP Chatbots in die bestehenden Applikationen der Kfz-Versicherer und damit in die .

Kundeninteraktion miteingebunden werden. Hier profitieren die Kunden der Kfz-Versicherer bei Fragen zu Schadensfällen oder Anträgen von kürzeren Wartezeiten und der Versicherer selbst von reduzierten Kosten bei der Bearbeitung von Kundenanfragen. Weiterhin können Kfz-Versicherer mit dem Intelligent RPA Service der SAP BTP die Kosten in der Sachbearbeitung reduzieren, indem repetitive Aufgaben wie das Erstellen von Berichten oder das Überprüfen von Kfz-Schäden automatisiert wird. Eine weitere Technologie mit der Kfz-Versicherer sich über die BTP vom Wettbewerb differenzieren können, sind die Low-Code, No-Code Tools des SAP Build Apps Services. Mit diesen können die Facharbeiter zur Entwicklung von Prozesserweiterungen befähigt werden und damit trotz des IT-Fachkräftemangels in der Kfz-Versicherungsbranche über mehr Entwicklungspower verfügen.
Abschließend lässt sich festhalten, dass die Anforderungen der Kfz-Versicherer an eine digitale Plattform von der BTP erfüllt werden und diese damit für die Kfz-Versicherer einen Möglichkeit darstellt, sich gegenüber der Konkurrenz einen Wettbewerbsvorteil zu verschaffen.








