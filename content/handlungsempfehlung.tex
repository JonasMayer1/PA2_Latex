\chapter{Handlungsempfehlung}

Um im intensiven Wettbewerb bestehen zu können, müssen Versicherer die Digitalisierung  vorantreiben, Innovationen beschleunigen und sich dynamisch den Marktbedingungen und -trends anpassen.

Deutschen Kfz-Versicherern kann dafür der Einsatz der SAP BTP empfohlen werden, da sie alle Anforderungen, abgesehen von der Bereitstellung von Chatbot-Funktionalitäten, an eine digitale Plattform erfüllt. Insbesondere in den für Kfz-Versicherern wichtigen Anforderungsbereichen Integration und Entwicklung wird die BTP besser bewertet als die Produkte der Wettbewerber. Um nun durch den Einsatz der SAP BTP Wettbewerbsvorteile zu erzielen, könnten Kfz-Versicherer die SAP BTP, wie nachfolgend beschrieben schrittweise nutzen:

(1) Im Rahmen einer SOA sollten Kfz-Versicherer ihre bestehenden Backend-Systeme über Cloud-Konnektoren an die BTP anbinden, um die dort vorgehaltenen Daten und Prozessabläufe für andere Anwendungen zugänglich zu machen. Nach erfolgreicher Anbindung der Legacy-Systeme, sollten die REST-, SOAP- und OData-APIs der BTP für die Integration in Mobilitätsökosysteme sowie zur Anbindung weiterer Vertriebskanäle genutzt werden. Dadurch könnten Versicherungen ihren Kundenstamm erweitern und folglich ihre Marktposition ausbauen. Durch Verknüpfung eigener Versicherungsangebote mit den Services anderer Unternehmen ist außerdem die Erstellung neuer Dienstleistungen und Produkte möglich. Komplementär sollte dabei zur Steuerung der Schnittstellenaufrufe der API Management Service der BTP verwendet werden.

(2) Darauf aufbauend sollten Kfz-Versicherer die Mobile Services der SAP BTP nutzen, um für Versicherungsnehmer eine App bereitzustellen, mit der mobile Interaktionen für verschiedene Versicherungsaktivitäten wie das Melden eines Schadensfalls oder die Anlage und Veränderung von Versicherungsverträgen durchgeführt werden können. So präferierten gemäß einer Analyse von Deloitte Insights im Jahr 2020 bereits 15 \% der deutschen Kfz-Kunden einen mobilen Vertragsabschluss.\autocite[Vgl.][S. 15]{BAUMANN2020} Zur Entwicklung der mobilen Applikation sollte das SAP Business Application Studio verwendet und auf die notwendigen Daten über die BTP zugegriffen werden. 

(3) Zusätzlich zu der Anbindung der verschiedenen Altsysteme sollten Kfz- Versicherer den Datasphere Service der BTP verwenden, um die historischen Daten aus den Altsystemen zu extrahieren und in ein einheitliches, strukturiertes Format zu transformieren. Darauffolgend sollten die aufbereiteten Daten in die SAP Analytics Cloud geladen und dort mithilfe von \ac{ml}-Modellen ausgewertet werden, um die Prämien- und Risikokalkulation zu optimieren. Dadurch könnten die Versicherer ihre Tarife genauer auf die individuellen Bedürfnisse und Risiken der Kunden zuschneiden und sich möglicherweise durch günstigere Tarife von ihren Konkurrenten differenzieren.

(4) Wenn aufgrund von sich ändernden Marktbedingung geschäftskritische Prozesse schnell verknüpft oder angereicht werden müssen, sollten Kfz-Versicherer auf die verschiedenen Laufzeitumgebungen der BTP zurückgreifen, um Microservices - oder Container basierte Erweiterung zu entwickeln. Des Weiteren sollten Kfz-Versicherer, sofern möglich die Low-Code No-Code Werkzeuge des SAP Build Apps Services nutzen, um schnell einfachere Applikationen zu erstellen oder zu erweitern. 

(5) Aber auch Services der BTP, bei denen es am Markt favorisiertere Lösungen gibt, könnten genutzt werden. So könnten beispielsweise die IoT Services der SAP BTP verwendet werden, um Sensordaten zum Fahrverhalten der Kunden zu integrieren, und diese später bei der Kundentarifierung zu berücksichtigen. Dadurch könnten Kfz-Versicherer langfristig nutzungsbasierte Telematik-Tarife anbieten.

(6) Darüber hinaus könnten Kfz-Versicherer repetitive Prozesse wie z. B. das Erstellen von Berichten automatisieren, indem sie mit dem SAP Build Process Automation Service beaufsichtigte oder unbeaufsichtigten Bots entwickeln und einsetzen oder falls möglich einen von SAP vordefinierten Bot verwenden. Dadurch könnte die Bearbeitungszeit der Prozesse reduziert und mehr Humankapital für die Kundenbetreuung zur Verfügung stehen.


