\chapter{Handlungsempfehlung}

\improvement{Idee Bernd: BTP ist vielleicht nicht die beste Lösung am Markt, aber bietet alle Bausteine, da kann man Erfahrung sammeln und später auf ein noch besseres Produkt wechseln}

Deutschen Kfz-Versicherern kann der Einsatz der SAP BTP empfohlen werden, da sie nahezu alle Anforderungen, ausgenommen das Bereitstellen von Chatbot Funktionalitäten, an eine digitale Plattform erfüllt. Insbesondere in den für Kfz-Versicherer wichtigen Bereichen Integration und Entwicklung wird die BTP von Benutzern besser bewertet als die Produkte der Wettbewerber. Um nun durch den Einsatz der SAP BTP Wettbewerbsvorteile zu erzielen, könnten Kfz-Versicherer die einzelnen Services der SAP BTP wie nachfolgend beschrieben schrittweise realisieren:

(1) Im Rahmen einer SOA sollten Kfz-Versicherer ihre bestehenden Backendsysteme über Cloud-Konnektoren an die BTP anbinden, um die dort vorgehaltenen Daten und Prozessabläufe für andere Anwendungen zugänglich zu machen. Nach erfolgreicher Anbindung der Legacy-Systeme, sollten die REST-, SOAP- und ODATA-APIs der BTP für die Integration in Mobilitätsökosysteme, sowie zur Anbindung weiterer Vertriebskanäle genutzt werden. Durch Verknüpfung eigener Versicherungsangebote mit den Services anderer Dienstleister ist die Entwicklung neuer Services und Produkte sowie die Erweiterung  des Kundenstamms möglich. So können Versicherer ihre Marktposition stärken. Komplementär sollte dabei zur Steuerung der Schnittstellenaufrufe der API Management Service der BTP verwendet werden.

(2) Darauf aufbauend sollten Kfz-Versicherer die Mobile Services der SAP BTP nutzen, um für Versicherungsnehmer eine App bereitzustellen, mit der mobile Interaktionen für verschiedene Versicherungsaktivitäten wie z.B. der Abschluss und die Verwaltung von Versicherungsverträgen oder das Melden eines Schadendalls durchgeführt werden können. So präferierten gemäß einer Analyse von Deloitte Insights im Jahr 2020 bereits 9% der deutschen Kfz-Kunden einen mobilen Vertragsabschluss – in anderen Ländern sogar noch deutlich mehr.(Vgl. BAUMANN2020 S. 15)  Dabei sollte zur Entwicklung der mobilen Applikation das SAP Business Application Studio verwendet und auf die notwendigen Daten über die BTP zugegriffen werden. 

(3) Zusätzlich sollten Kfz-Versicherer nachdem die Anbindung der verschiedenen Altsysteme an die BTP abgeschlossen ist, den Datasphere Service der BTP verwenden, um die historischen Daten aus den Altsystemen zu extrahieren und diese in ein konsistentes und strukturiertes Format zu transformieren. Darauffolgend sollten die aufbereiteten Daten in die SAP Analytics Cloud geladen und dort mithilfe von ML-Modellen ausgewertet werden, um die Prämien- und Risikokalkulation zu optimieren und anschließend Kunden auf dieser Basis maßgeschneiderte Tarife anbieten zu können.

(4) Wenn individuelle Entwicklungen benötigt werden, sollten die leistungsfähigen Anwendungsentwicklungsfunktionalitäten der SAP BTP genutzt werden wie (SAP Cloud Foundry Environment, SAP BTP ABAP Environment , Business Application Studio, CAP) . Des Weiteren sollten Kfz-Versicherer wenn möglich die Low-Code No-Code Werkzeuge des SAP Build Apps Services nutzen, um einfache Applikationen zu erstellen oder zu erweitern. Damit könnte die Zusammenarbeit zwischen IT- und Geschäftsabteilung in funktionsübergreifenden Teams gestärkt und gleichzeitig das Problem der fehlenden IT-Fachkräfte gemildert werden.

(5) Aber auch Services der BTP, bei denen es am Markt bevorzugtere Anwendungen gibt , könnten genutzt werden. So könnten beispielsweise die IoT Services der SAP BTP verwendet werden, um Sensordaten zum Fahrverhalten der Kunden zu integrieren, und dann später bei der Kundentarifierung berücksichtigen zu können. Dadurch könnten Kfz-Versicherer langfristig  nutzungsbasierte Telematik-Tarife anbieten.

(6) Darüber hinaus könnten Kfz-Versicherer repetitive Prozesse wie z.B. das Erstellen von Berichten automatisieren, indem sie mit dem SAP Build Process Automation Service beaufsichtigte oder  unbeaufsichtigten Bots entwickeln und einsetzen oder falls möglich einen von SAP vordefinierten Bots verwenden. Dadurch könnte die Bearbeitungszeit der Prozesse reduziert und mehr Humankapital für die Kundenbetreuung zur Verfügung stehen.



