\chapter{Einleitung}

\section{Motivation}

Heutzutage nutzen wir in den verschiedensten Bereichen jeden Tag Suchfunktionen, ohne länger darüber nachzudenken. Eine hilfreiche Suche zeichnet sich dadurch aus, dass sie dem Benutzer genau das Ergebnis liefert, nachdem er gesucht hat. Dies wird in Zeiten, in denen immer mehr Daten wegen geringer Datenspeicherungskosten abgespeichert werden, immer wichtiger, aber auch schwieriger.

Damit man das zu findende Ergebnis möglichst gut beschreiben kann, gibt es insbesondere im Bereich der Unternehmenssoftware Suchkriterien, um die Liste an möglichen Resultaten zu reduzieren. Je mehr geeignete Suchkriterien zur Eingrenzung der potenziellen Ergebnisse angegeben werden, desto wahrscheinlicher ist es, eine passende Ergebnisliste beziehungsweise das passende Zielobjekt angezeigt zu bekommen.

Wenn eine Suche so modifiziert werden kann, dass die ausgegebene Ergebnisliste die Bedürfnisse des Nutzers präziser trifft, steigert dies die Effizienz. Zeit und Kosten können gespart und teilweise zusätzliche oder unnötige Arbeitsschritte vermieden werden. Betrachtet man die Vielzahl der täglich durchgeführten Suchanfragen, fällt auf, dass auch kleine Zeiteinsparungen auf längere Zeit einen großen betriebswirtschaftlichen Vorteil bieten.

Die Abteilung Application Innovation Services der SAP SE setzt genau hier an. Sie erarbeitet mit dem Kunden Möglichkeiten, SAP-Software-Anwendungen so zu verbessern, dass sie die Bedürfnisse der Kunden noch genauer treffen und deren Geschäftsprozesse noch besser unterstützen können, um dadurch Qualitäts- und Kostenvorteile zu erzielen. Im Rahmen des Customer Connection Programms wurde hierbei für das Produkt Lifecycle Management der Verbesserungswunsch identifiziert, den Suchprozess für Spezifikationen zu optimieren.\autocite[Vgl.][]{ADSESPEC}

\begin{comment}
Damit die eigenen Produkte 
Um die Suche zu verbessern bietet es sich an, eng mit dem Kunden zusammenzuarbeiten, da er selbst am besten einschätzen kann, wie Unternehmenssoftware sein Suchprozess erleichtern können.


Damit Unternehmen ihre Wettbewerbsfähigkeit am Markt sichern können, müssen sie ständig ihre Kosten senken, Prozesse verbessern und ihre Kunden von der Vorteilhaftigkeit ihrer Produkte überzeugen. Ein Weg die oben genannten Punkte zu erreichen, ist die eigenen Produkte ständig weiterzuentwickeln und an die Bedürfnisse der Kunden anzupassen. Da Innovationen langfristig für das erfolgreiche Bestehen jedes Unternehmens verantwortlich sind, ist es wichtig, dass Unternehmen sich aktiv mit diesem Thema auseinandersetzen. Eine Idee oder ein Kundenwunsch kann als Innovation betrachtet werden, sobald die neu entwickelten Produkte bzw. Produktionsmethoden am Markt eingeführt werden. Ein sich nicht veränderndes Unternehmen wird durch seine Konkurrenten und die Veränderung der Umwelt innert kürzester Zeit überholt und außer Gefecht gesetzt / 
Es ist wichtig zu verstehen, dass Stillstand im unternehmerischen Kontext immer Rückschritt bedeutet.

Der Bereich AIS PLM beschäftigt sich mit genau diesen beiden Punkten für die Suchen im Logistik Bereich von SAP-Anwendung SAP ERP oder SAP \acs{s/4hana} onPremise. Die Abteilung setzt die qualifizierten Kundenwünsche bezüglich der Suche um und ist so ein wichtiges Puzzleteil für die zukunftsgerichtete und kundenorientierte SAP Software.

%Es muss sich überlegt werden, wie Innovationsprozesse möglichst effizient gestaltet und in die bestehenden Geschäftstätigkeiten integriert werden können, so dass erfolgsversprechende Ideen ausgewählt, gefördert und umgesetzt werden können und dadurch die anspruchsvollen Bedürfnisse der Kunden  langfristig befriedigt werden
Wichtigkeit der Suche 
Workflow verbessern, Zeitaufwand und Arbeitsschritte reduizieren
Orientierung mit Autocomplete und Vorschlägen
was macht Suchmaschinen wie Google so erfolgreich
Sollte direkt auf der Website gefunden werden könnenEingrenzung der Suchergebnisse
Verstehen wonach der Benutzer sucht
(Fehlertoleranz)

Beides im PLM Bereich, Abteilung vorstellen: 
Kundensupport und der Wartung, Entwicklung und dem Testen von Software in allen Bereichen der SAP
beschäftigt sich mit
dem Bereich der Logistik der SAP-Anwendung SAP ERP oder SAP S/4HANA onPremise
Eigenentwickeln und testen von neuen Komponenten
Kundenmeldungen und und CCR
\end{comment}

\section{Ziel des Projektes}

%Das Ziel dieser Projektarbeit ist das Evaluieren und Implementieren einer geänderten Suchoberfläche für das SAP Product Lifecycle Management im S/4HANA Umfeld. Dabei soll die bestehende Suchoberfläche für Spezifikationen um weitere Suchmöglichkeiten, die sich von SAP - Kunden gewünscht wurden, erweitert werden. 

Im Rahmen des Customer Connection Programmes bietet SAP allen Kunden die Möglichkeit an, Verbesserungsvorschläge für alle SAP-Applikationen unabhängig vom jeweiligen Softwarestand des Kunden einzureichen.\autocite[Vgl.][]{CCP}

Innerhalb dieses Programmes wurde von verschiedenen Kunden für die Applikation SAP \ac{plm} die Optimierung der Suchoberfläche für Spezifikationen durch Ergänzung zusätzlicher Selektionskriterien gewünscht. 
Die Anpassung der Suchoberfläche wurde von der SAP aufgrund der hohen Relevanz für verschiedene Kunden zur Umsetzung freigegeben.\autocite[Vgl.][]{ADSESPEC}

Ziel des Projektes ist somit die Realisierung und Implementierung dieses Kundenwunsches. Dabei soll dem Anwender in zwei bestehenden Suchmasken für Produktspezifikationen zukünftig die Möglichkeit gegeben werden, Datensätze nicht nur mit den bereits vorhandenen Suchkriterien, sondern zusätzlich auch nach dem letzten Änderer der Spezifikation und dem letzten Änderungsdatum auszuwählen.\autocite[Vgl.][]{ADSESPEC}

Die gewünschte Funktionalität steht dem Anwender bereits in anderen Suchbildschirmen für die \acs{plm}-Objekte Rezepte, Dokumente und der Materialstückliste zur Verfügung und soll nun in ähnlicher Form auch bei der Spezifikation umgesetzt werden.\autocite[Vgl.][]{ADSESPEC}

Die Anpassungen müssen sowohl für die SAP-\ac{plm}-Applikation im SAP \ac{erp} Umfeld als auch für \ac{s/4hana} implementiert werden.

Nicht Bestandteil dieser Arbeit sind die Suchverfahren und Suchstrategien der SAP-\ac{hana} Suche, da diese für die Implementierung des neuen Suchscreens nicht relevant sind und den Rahmen der Projektarbeit übersteigen würden.

\section{Gang der Untersuchung}

Zu Beginn der Projektarbeit werden zunächst das SAP Customer Connection Programm, aus dem der Kundenwunsch resultiert, und die SAP Applikation \ac{plm} vorgestellt, in der der Kundenwunsch realisiert werden soll. Danach werden wichtige technische Grundlagen für die Suchanwendung erläutert: Die \ac{es} als zugrunde liegende Suchtechnologie, \acs{abap} als verwendete Programmiersprache und Web-Dynpro als Basistechnologie für die Web-Oberfläche. Anschließend wird die Wahl der wissenschaftlichen Methodik begründet, die Hauptmerkmale der Referenzmodellierung und der Ablauf eines genehmigten Customer Connection Requests in der Entwicklung präsentiert. 

Darauffolgend wird bei der Ist-Analyse der Suchoberfläche die Ausgangssituation vor der Implementierung näher betrachtet und die Anforderungen aus dem Customer Connection Request dargelegt. 

Nach der Ist-Analyse werden die bereits vorhandenen erweiterten Suchen analysiert und die Umsetzung unter Nutzung der gewonnenen Erkenntnisse dargestellt. Abschließend wird das Ergebnis der Projektarbeit kritisch reflektiert und ein Ausblick auf mögliche anstehende Aufgaben gemacht. 
