\chapter{Analyse}
%Evaluation der SAP Business Technology Platform für die Anforderungen auf dem deutschen Kfz-Versicherungsmarkt
\section{Task Charakteristiken - Identifikation der Anforderungen der Kfz-Versicherer an digitale Plattformen}

\subsection{Literaturbetrachtung aktueller Anforderungen an digitale Plattformen}

Eine der größten Herausforderungen, denen sich Versicherungsunternehmen stellen müssen, sind die historisch bedingten IT-Landschaften, die auch als IT-Legacy oder Legacy-Systeme bezeichnet werden. Diese sind seit Beginn der 1970er Jahre entstanden und wurden dabei größtenteils von den Versicherern selbst entwickelt. Mittlerweile führt allerdings genau diese IT-Legacy aufgrund der starken Abhängigkeiten zwischen den Systemen und der monolithischen Programmstruktur häufig zu langen Entwicklungszyklen (vgl. S. 10 – 12 Gunter2020) (BAIN) Um dennoch adäquat auf die sich ändernden Kundenbedürfnisse und Marktbedingungen reagieren zu können, müssen Kfz-Versicherer eine neue technologische Plattform zur Neugestaltung ihrer IT-Landschaft einführen. Dabei sollte die Plattform eine Service-orientierte Architektur (SOA) unterstützen, damit bestehende, bereits abgeschlossene Software-Bausteine, wie zum Beispiel die Rechenkerne der verschiedenen Versicherungssparten, weiterhin genutzt und in die neue IT-Infrastruktur integriert werden können. (vgl. S. 10-11 Urlaß) Darüber hinaus erleichtert der modulare Aufbau einer SOA auch das Hinzufügen und Entfernen neuer Anwendungen an der Kundenschnittstelle.(BAIN)

Bei der Einführung einer neuen technologischen Plattform ist eine grundlegende Anforderung der Versicherungsunternehmen, die Anbindung der bestehenden Legacy-Systeme, da diese eng mit den bestehenden Geschäftsprozessen verknüpft sind und Daten enthalten, welche aus regulatorischen Gründen noch aufbewahrt werden müssen. Hierfür können im Rahmen einer SOA die einzelnen, logischen Elemente der Systemlandschaft isoliert und anschließend mithilfe einer (SOAP-)Schnittstelle als Service den anderen Systemen zur Verfügung gestellt werden. Daher muss eine technologische Plattform für Kfz-Versicherer das Simple Object Access Protocol unterstützen. (vgl. GUNTER2020 S. 10-12)

Darüber hinaus sollte es die Plattform ermöglichen, bestehende Anwendungen mithilfe von kleinen Programmcodes eigenständig erweitern zu können, um die Anwendungen nach den Bedürfnissen der Kfz-Versicherer anpassen zu können. Hierbei gilt es zu berücksichtigen, dass insbesondere bei kleinen und mittelgroßen Versicherern die Entwicklungsressourcen sehr knapp sind. Daher sollte die Plattform ebenfalls sogenannte Low- oder besser No-Code-Werkzeuge bereitstellen, damit auch Nutzer aus den Fachabteilungen Erweiterungen erstellen können.(3) (vgl. WEINGARTNER2023)

Eine Trend der insbesondere in den nächsten Jahren die Versicherungsbranche bestimmen wird, sind die sogenannten Digitalen Ökosysteme und die damit verbundenen Partnerschaften. So erachten gemäß einer Studie der Swiss RE aus dem Jahr 2019 mehr als 75\% der Führungskräfte von Versicherungsunternehmen weltweit digitale Ökosysteme und andere Partnerschaften als wesentlich für die Schaffung von Wettbewerbsvorteilen. (vgl. PAYNE2022) (vgl. AVRAMAKIS2023) Für die Partizipation an Digitalen Ökosystemen ist dabei vor allem das Vorhandensein offener Standardschnittstellen und Austauschformate maßgebend, sodass ähnlich wie im Banking mit der PDS2 Richtlinie auch Versicherungsunternehmen ihre Daten leicht mit Partnern und Drittanbietern  teilen können. Zu diesem Zweck hat sich 2018 (vgl. 2021z) die Free Insurance Data Initiative (FRIDA)  gebildet, welche für die einzelnen Versicherungssparten offene Schnittstellen Standards schaffen möchte. In der in der Kfz-Versicherungssparte ist es die sogenannte Car-ClaimsAPI, welche als REST-Schnittstelle den Austausch von Kfz-Police Daten vereinfachen soll. Folglich sollte eine Plattform für Kfz-Versicherer auch Rest-API-Calls unterstützen, um sich mit Partnern und InsurTechs in einem Digitalen Ökosystem verbinden zu können und somit wettbewerbsfähig zu bleiben. (vgl. KRETZ2023) 

Zudem wird es immer wichtiger nicht die Kontrolle und Übersicht über die einzelnen Schnittstellen zu verlieren, insbesondere dann, wenn alter Code der nie für eine Internet-Anbindung ausgelegt war, mit einem Mal öffentlich zugänglich wird. (vgl. B. S.137) (S.393 HANSCHKE2021) Daher sollten technologische Plattformen für Kfz-Versicherer über eine API-Management-Tool verfügen, um die API Aufrufe in und außerhalb der Unternehmens analysieren, kontrollieren und verwalten zu können. (vgl S.83fHANSCHKE2021)

Darüber hinaus zeigt sich die Herausforderung der Legacy-Systeme auch im Datenmanagement. So existieren in den fragmentierten IT-Landschaften viele firmeninterne Datensilos, die keine integrierte Sicht auf die Daten erlaubt. Um diese Herausforderung zu überwinden sollte die technische Plattform eine spartenübergreifende Sicht auf alle Daten ermöglichen. (vgl. GUNTER2020 S. 11)

Daneben ist es im Rahmen des Datenmanagements wichtig sicherzustellen, dass die Daten in einem konsistenten und strukturierten Format vorliegen, sodass auf dieser Grundlage eine effektive Analyse und Verwaltung der Daten möglich ist. Folglich muss die Plattform eine Extract-Transform-Load (ETL)-Lösung bereitstellen, um Daten aus verschiedenen Quellen zu extrahieren, sie zu transformieren und schließlich in ein Zielsystem zur Analyse laden zu können. (vgl. WEINGARTNER2023), (vgl. ASCHENBRENNER2010 S. 342-345)

Durch die Technologisierung der Automobilbranche und den umfangreichen Daten, die in Fahrzeugen erhoben werden, hat sich die Kfz-Versicherung als Vorreiter der Assekuranz im Bereich Data Analytics hervorgetan.(vgl. GATZERT2023 S. 230) Die Auswertung von großen vernetzten Datensätzen, auch als Big Data bezeichnet, ermöglicht eine genauere Abschätzung bestehender Risiken und führt somit zu einer exakteren Kalkulation von Prämien und neuen Produkten. Dies ist auf die Anwendung der Wahrscheinlichkeitstheorie zur Prämienberechnung zurückzuführen, bei der ein höheres Ausmaß an untersuchten Fällen zu einem präziseren Ergebnis führt. Daher sollte eine technologische Plattform für Kfz-Versicherungen auch Lösungen zur Analyse großer Datenmengen bieten, um den Kfz-Versicherern eine optimale Prämien- und Risikokalkulation zu ermöglichen (vgl. MANGEI 2019 S. 146).

Des Weiteren wird das von einem Menschen gesteuerte KFZ immer selbstständiger und vernetzter, bis hin zu autonom verkehrenden Fahrzeugen. Dies hat zu einer Zunahme von Telematik-Tarifen in der Branche geführt, wobei eine Studie von Roland Berger aus dem Jahr 2015 prognostiziert, dass bis 2030 etwa 25\% des Kfz-Versicherungsmarktes von Telematik-Produkten dominiert werden (vgl. ROLANDBERGER 2015 S.4) Telematik bezeichnet allgemein die Verbindung von Informatik und Telekommunikation, (vgl. ABTS2017 S. 10-11) und ermöglicht in der Kfz-Versicherungsbranche die Berücksichtigung des individuellen Fahrverhaltens des Versicherungsnehmers sowie anderer externer Rahmenbedingungen bei der Prämienermittlung.(vgl. MERZINGER2017 S.84) So kann der Versicherer seine Produkte besser an das Risikoprofil seiner Kunden anpassen und die Kunden bei einer zurückhaltenden Fahrweise von günstigeren Preisen profitieren. Hierfür werden mithilfe von Sensoren im Auto verschiedene Daten zum Fahrstil des Nutzers erfasst und zur Prämienermittlung an der Versicherer übermittelt.(vgl. ELING2020 S. 3-4) Für den Einsatz von technologischen Plattformen bei Kfz-Versicherern ist es folglich wichtig, dass die Plattform Sensordaten von physischen Objekten in Echtzeit empfangen und verarbeiten kann.(vgl. Link zu: Das Kfz, die Telematik und der Datenschutz S. 10-15) Zudem sollte die Plattform auch Möglichkeiten zur Entwicklung von IoT-Anwendungen bereitstellen, um Sensordaten in Geschäftsprozesse integrieren zu können.

Neben den bestehenden Legacy-Systemen und der Wertschöpfung aus Daten, haben zahlreiche Kfz-Versicherer mit dem Problem zu kämpfen, dass viele ihrer internen repetitiven Prozesse noch manuell von Mitarbeitern ausgeführt werden müssen. Dies führt zu langen Bearbeitungszeiten, Anfälligkeit für menschliche Fehler und einer Bindung von wertvollem Humankapital, das nicht für die Kundenbetreuung eingesetzt werden kann. Eine Lösung zur Bewältigung dieser Herausforderung, die auch von der technologischen Plattform unterstützt werden sollte, ist die Robotic Process Automation (RPA). So können durch RPA repetitive Aufgaben wie beispielsweise die Übertragung von Daten in Systeme, das Erstellen von Berichten oder das Überprüfen von Schäden automatisiert werden. Die Allianz vermutet, dass durch den Einsatz von RPA, die Arbeitszeit um 40\% reduziert und gleichzeitig die Qualität der Leistung verbessert werden. Dies betont noch einmal die Bedeutung von RPA für die Wettbewerbsfähigkeit in der Branche. (vgl. REICH2019 S. 296-298)

Gemäß Reich(2019) besteht eine zusätzliche Option für die Automatisierung von Kundeninteraktion in Fällen wie Anträgen, Auskünften oder Schadensmeldungen durch den Einsatz von Chatbots. So können Fragen von Kunden, insbesondere solche, die sich auf Versicherungspolicen, Abrechnungen oder Schadensmeldungen beziehen, jederzeit beantwortet werden, ohne dass der Kunde auf einen Mitarbeiter warten muss. Dadurch kann der Bedarf nach Mitarbeitern in der Sachbearbeitung reduziert und die Kundenzufriedenheit erhöht werden, weshalb die Plattform für Versicherer auch Chatbot-Funktionalitäten bereitstellen sollte. (vgl. REICH2019 S. 300-302)

Des Weiteren ist das Smartphone und die damit verbundenen Anwendungen ein ständiger Begleiter der Versicherungsnehmer. Dabei können die Anwendungen im Kfz-Versicherungskontext beispielsweise bei einem Autounfall genutzt werden, um den eigenen Standort sowie Fotos vom Schaden zu teilen. Folglich ist die Unterstützung mobiler Anwendung eine Anforderung für die Verwendung einer technologischen Plattform in der Kfz-Versicherungsbranche. (vgl. BAIN2013 S.16)

Aufgrund der strengen regulatorischen Anforderungen in der Versicherungsbranche ist die Datensicherheit eine Grundvoraussetzung für jegliche eingesetzte Softwarelösung und somit auch für technische Plattformen. Dabei ist die Bewertung des Sicherheitsniveaus einer Software mühsam und zeitaufwendig, weshalb Versicherungsunternehmen in der Praxis auf Zertifizierungen und Testate zurückgreifen. (vgl. ZDANOWIECKI2016 S. 777) Bei cloud-basierten Lösungen gilt dabei die ISO/IEC 27001 Norm als  Standard, da sie die Anforderungen an ein Informationssicherheitsmanagementsystem (ISMS) definiert. Darüber hinaus müssen Kfz-Versicherer sicherstellen, dass die Verarbeitung von Daten den Bestimmungen der Datenschutzgrundverordnung (DSGVO) entspricht, um den Schutz personenbezogener Daten zu gewährleisten. Zu diesem Zweck wurde im Jahr 2014 der ISO/EIC 27018-Standard speziell entwickelt. Daher sollten technische Plattformen für Kfz-Versicherer sowohl nach ISO/EIC 27001 als auch nach ISO/EIC 27018 zertifiziert sein.

Eine weitere Anforderung, die ebenfalls von Beginn an erfüllt sein sollte, ist die Fähigkeit der Plattform zur horizontalen und vertikalen Skalierbarkeit. Dies erfordert, dass die Plattform zusätzliche Server automatisch starten und Lasten auf mehrere Server verteilen kann, um plötzliche und hohe Nutzerzahlen zu bewältigen, wie sie zum Beispiel im Herbst bei Kfz-Versicherungen vorkommen können. (horizontale Skalierbarkeit). Des Weiteren muss die Plattform in der Lage sein, je nach Bedarf die Rechenleistung und Speicherkapazität des Servers zu erhöhen, um einem höheren Ressourcenbedarf gerecht zu werden (vertikale Skalierbarkeit). (vgl. JAHNERT2020 S. 23)

Eine tabellarische Auflistung der Anforderungen ist dem Anhang ... zu entnehmen. (kommt noch)











\subsection{Evaluation und Ergänzung der Anforderungen aus Sicht der Experten}

\section{Technologie Charakteristiken - Komponenten und Services der SAP Business Technology Platform}\label{sec:TechCharak}

\section{Synthese der Task und Technology Charakteristiken}


