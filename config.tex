% !TEX root =  master.tex
%      HYPERREF

%%%%%%%%%%%%%%%%%%%%%%%%%%%%%%%%%%%%%%%%%%%%%%%%%%%%%%%%%%
%	ANLEITUNG: 
%
% Passen Sie alle Stellen im Dokument an, die mit 
% @stud markiert sind
%
% Die untersten packages können die oberen überschreiben, deswegen die wichtigen als letztes
%%%%%%%%%%%%%%%%%%%%%%%%%%%%%%%%%%%%%%%%%%%%%%%%%%%%%%%%%%

\usepackage{makeidx}      			  		% allows index generation, mit \index Wörter zum Stichwort Verzeichnis hinzufügen
\usepackage{listings}						%Format Listings properly , um code in Latex als Box einzufügen
\usepackage{lipsum}   						%Blindtext
\usepackage{graphicx}						% use various graphics formats, um Grafiken einzubinden
\usepackage[german]{varioref} 				% nicer references \vref
\usepackage{caption}						%better Captions
\usepackage{booktabs} 						%nicer Tabs
\usepackage{array}
\usepackage{chngcntr} 						%zum Neudefinieren von Zählern
\usepackage{fnpct} 							% Correct superscripts ; tauscht Fußnote mit Satzzeichen, wenn die Fußnote vor dem Satzzeichen steht
\usepackage[T1]{fontenc} 					%ermöglicht das Verwenden von Umlauten
\usepackage[utf8]{inputenc}					%Inputcode festlegen
\usepackage{calc}							% Used for extra space below footsepline
\usepackage[printonlyused]{acronym} 		%delete content of the square brackets to force print all acronyms
\usepackage{algorithm}  					% um code in Latex als Box(ähnlich wie eine figure) einzufügen
\usepackage{algpseudocode} 					%für verschiedene darstellweisen des pseudocodes von algorithm
%\usepackage{url} 							% Um urls einzufügen, veraltet, funktion mitlerweile bei hyperref integriert
\usepackage{comment}						%Für Kommentarabschnitte: \begin{comment} \end{comment}
%Edit MaxD %%%%%%%%%%%%%%%%%%%%%%%%%%%%%%%%%%%%%%%%%%%%%%%%%%%%
\usepackage{geometry}						%Zum Verändern der Gestaltung einer Seite: breite, höhe, oberer, unterer Rand, Ausdruck(einseitig, doppelseitig etc)
\usepackage{tikz}							%Zum Darstellen von Grafiken wie:  geometrische Formen oder Koordinatensysteme


\usepackage{tikz}
\usetikzlibrary{calc}
\usetikzlibrary{arrows.meta}
\usetikzlibrary{positioning}
\usetikzlibrary{shapes}
\usetikzlibrary{fit}
\usetikzlibrary{positioning}
\usetikzlibrary{shadows}


\usepackage{footnote}						%Für Fußnoten: \footnote{<text>}
\usepackage{wrapfig}						%Für Schrift umflossene Bilder
\usepackage{float}							%Ermöglicht bei figures: "das bestmögliche platzieren auf der Website"
\usepackage{array,multirow,graphicx}		%array = ermöglicht es, Spalten mit fester Breite auch zentriert oder rechtsbündig auszurichten
											%multirow = ermöglicht in Tabellen, dass Spalten und Zeilen zusammengefasst werden können
											%graphicx = Zum Einbinden von Bildern
\usepackage{listings}						% Um Programmcode in Latex einzubinden: \begin{lstlisting} ; \end{lstlisting}
\lstset{breaklines,numbers=left,xleftmargin=2em,frame=single,framexleftmargin=2em}    % automatische Zeilenumbrüche, wo die line numbers hinsollen, dicke des linken Randes; Rahmen um den Code;  

\usepackage{xargs}%Commands mit mehreren optionalen Parametern möglich
\usepackage[colorinlistoftodos,prependcaption,textsize=tiny]{todonotes} % Todonotes: mit \todo[]{} \missingfigure{} notes für fehlende Elemente einfügen, mit \listoftodos Liste der todos anzeigen lassen
\newcommandx{\unsure}[2][1=]{\todo[linecolor=red,backgroundcolor=red!25,bordercolor=red,#1]{#2}}
\newcommandx{\change}[2][1=]{\todo[linecolor=blue,backgroundcolor=blue!25,bordercolor=blue,#1]{#2}}
\newcommandx{\info}[2][1=]{\todo[linecolor=green,backgroundcolor=green!25,bordercolor=green,#1]{#2}}
\newcommandx{\improvement}[2][1=]{\todo[linecolor=purple,backgroundcolor=purple!25,bordercolor=purple,#1]{#2}}
\newcommandx{\thiswillnotshow}[2][1=]{\todo[disable,#1]{#2}} % usable todo commands: todo, unsure, change, info, improvement, thiswillnotshow

\usepackage[mediumspace,mediumqspace,Grey,squaren]{SIunits}
\usepackage{pgfplots}						%Zum Darstellen von Grafiken, Koodinatensystemen(auch mehrdimensional)
\usepgfplotslibrary{dateplot}				%Bei Grafiken zum Benutzen von Daten
\usepackage{pgfplotstable}					%Zur Tabellennachbearbeitung
\pgfplotsset{compat=1.16}
%\usepackage{draftwatermark}%Delete after you are done with your thesis. This line creates the "Draft" Watermark
\usepackage{setspace}						%Zeilenabstand 3 Optionen: singlespacing; onehalfspacing; doublespacing

%Edit MaxD Fixes line breaks in columns of tables. Example: C{2cm}
\newcolumntype{L}[1]{>{\raggedright\arraybackslash}p{#1}} % left fixed width
\newcolumntype{C}[1]{>{\centering\arraybackslash}p{#1}} % center fixed width
\newcolumntype{R}[1]{>{\raggedleft\arraybackslash}p{#1}} % flush right fixed width
%%%%%%%%%%%%%%%%%%%%%%%%%%%%%%%%%%%%%%%%%%%%%%%%%%%%%%%%%%%%%%%%%%%%%

%originally in line 21 needed to be moved to the end
\usepackage[hidelinks=true]{hyperref} % keine roten Markierungen bei Links

%
% @stud
%
%	FONT SELECTION: Entweder 1) Latin Modern oder 2) Times / Helvetica ( FONT = Schriftart)
\usepackage{lmodern}             % 1) Latin modern font
%\usepackage{mathptmx}           % 2) Helvetica / Times New Roman fonts (2 lines)
%\usepackage[scaled=.92]{helvet} % 2) Helvetica / Times New Roman fonts (2 lines)

%
% @stud
%
%	LANGUAGE SETTINGS
\usepackage[ngerman]{babel} 	        % german language
\usepackage[german=quotes]{csquotes} 	% correct quoting using \enquote{}
%\usepackage[english]{babel}          % english language
%\usepackage{csquotes} 	              % correct quoting using \enquote{}

%
% @stud
%
% Uncomment the following lines to support hard URL breaks in bibliography 
%\apptocmd{\UrlBreaks}{\do\f\do\m}{}{}
%\setcounter{biburllcpenalty}{9000}% Kleinbuchstaben
%\setcounter{biburlucpenalty}{9000}% Großbuchstaben

%
% @stud
%
%	FOOTNOTES: Count footnotes over chapters
\counterwithout{footnote}{chapter}

%	ACRONYMS
\makeatletter
\@ifpackagelater{acronym}{2015/03/20}
{\renewcommand*{\aclabelfont}[1]{\textbf{{\acsfont{#1}}}}}{}
\makeatother

%	LISTINGS
\renewcommand{\lstlistingname}{Quelltext} 
\renewcommand{\lstlistlistingname}{Quelltextverzeichnis}
\lstset{numbers=left,
	numberstyle=\tiny,
	captionpos=b,
	basicstyle=\ttfamily\small}

%	ALGORITHMS
\renewcommand{\listalgorithmname}{Algorithmenverzeichnis }
\floatname{algorithm}{Algorithmus}

%		PAGE HEADER / FOOTER
%	    Warning: There are some redefinitions throughout the master.tex-file!  DON'T CHANGE THESE REDEFINITIONS!
\RequirePackage[automark,headsepline,footsepline]{scrlayer-scrpage}
\pagestyle{scrheadings}
%\renewcommand*{\pnumfont}{\upshape\sffamily}
%\renewcommand*{\headfont}{\upshape\sffamily}
%\renewcommand*{\footfont}{\upshape\sffamily}
\renewcommand{\chaptermarkformat}{}
\RedeclareSectionCommand[beforeskip=0pt]{chapter}
\clearscrheadfoot

\ifoot[\rule{0pt}{\ht\strutbox+\dp\strutbox}DHBW Mannheim]{\rule{0pt}{\ht\strutbox+\dp\strutbox}DHBW Mannheim}
\ofoot[\rule{0pt}{\ht\strutbox+\dp\strutbox}\pagemark]{\rule{0pt}{\ht\strutbox+\dp\strutbox}\pagemark}
\ohead{\headmark}

\newcommand{\TitelDerArbeit}[1]{\def\DerTitelDerArbeit{#1}\hypersetup{pdftitle={#1}}}
\newcommand{\AutorDerArbeit}[1]{\def\DerAutorDerArbeit{#1}\hypersetup{pdfauthor={#1}}}
\newcommand{\Firma}[1]{\def\DerNameDerFirma{#1}}
\newcommand{\Kurs}[1]{\def\DieKursbezeichnung{#1}}
\newcommand{\Abteilung}[1]{\def\DerNameDerAbteilung{#1}}
\newcommand{\Studiengangsleiter}[1]{\def\DerStudiengangsleiter{#1}}
\newcommand{\WissBetreuer}[1]{\def\DerWissBetreuer{#1}}
\newcommand{\FirmenBetreuer}[1]{\def\DerFirmenBetreuer{#1}}
\newcommand{\Bearbeitungszeitraum}[1]{\def\DerBearbeitungszeitraum{#1}}
\newcommand{\Abgabedatum}[1]{\def\DasAbgabedatum{#1}}
\newcommand{\Matrikelnummer}[1]{\def\DieMatrikelnummer{#1}}
\newcommand{\Studienrichtung}[1]{\def\DieStudienrichtung{#1}}
\newcommand{\ArtDerArbeit}[1]{\def\DieArtDerArbeit{#1}}
\newcommand{\Literaturverzeichnis}{Literaturverzeichnis}

\newcommand{\settingBibFootnoteCite}{
	\setlength{\bibparsep}{\parskip}		  % Add some space between biblatex entries in the bibliography
	\addbibresource{bibliography.bib}	    % Add file bibliography.bib as biblatex resource
	\DefineBibliographyStrings{ngerman}{andothers = {{et\,al\adddot}},}
	%Following two lines commented, because those features are no longer supported
	%\AdaptNoteOpt\footcite\multfootcite   % Will add  separators if footcite is called multiple consecutive times 
	%\AdaptNoteOpt\autocite\multautocite   % Will add  separators if autocite is called multiple consecutive times
}

\newcommand{\setTitlepage}{
	% !TEX root =  master.tex
\begin{titlepage}
\begin{minipage}{\textwidth}
		\vspace{-2cm}
		\noindent \includegraphics[scale=0.25]{\imagedir/firmenlogo.jpg} \hfill \includegraphics{\imagedir/logo.jpg}
\end{minipage}
\vspace{1em}
%\sffamily
\begin{center}
	{\textsf{\large Duale Hochschule Baden-W\"urttemberg Mannheim}}\\[4em]
	{\textsf{\textbf{\large{\DieArtDerArbeit}arbeit}}}\\[6mm]
	{\textsf{\textbf{\Large{}\DerTitelDerArbeit}}} \\[6mm]%[1.5cm]
	%{\textsf{\large{}Eine vergleichende Untersuchung und Konzeption}}\\[1.5cm]
	{\textsf{\textbf{\large{}Studiengang Wirtschaftsinformatik}}\\[6mm]
	\textsf{\textbf{Studienrichtung \DieStudienrichtung}}}\vspace{7em}
	
	\begin{minipage}{\textwidth}
		\begin{tabbing}
		Wissenschaftliche(r) Betreuer(in): \hspace{0.85cm}\=\kill
		Verfasser(in): \> \DerAutorDerArbeit \\[1.5mm]
		Matrikelnummer: \> \DieMatrikelnummer \\[1.5mm]
		Firma: \> \DerNameDerFirma  \\[1.5mm]
		Abteilung: \> \DerNameDerAbteilung \\[1.5mm]
		Kurs: \> \DieKursbezeichnung \\[1.5mm]
		Studiengangsleiter: \> \DerStudiengangsleiter \\[1.5mm]
		Wissenschaftliche(r) Betreuer(in): \> \DerWissBetreuer \\[1.5mm]
		%\> 0621 4105 1218 \\[1.5mm]
		%\> hans-peter.engel@dhbw-mannheim.de \\[1.5mm]
		Firmenbetreuer(in): \> \DerFirmenBetreuer \\[1.5mm]
		%\> +49 6227 7-70239 \\[1.5mm]
		%\> pierre.grosse@sap.com \\[1.5mm]
		Bearbeitungszeitraum: \> \DerBearbeitungszeitraum\\[1.5mm]
%		alternativ:\\[1.5mm]
		Eingereicht: \> \DasAbgabedatum	
		\end{tabbing}
	\end{minipage}
\end{center}
\end{titlepage}
	\pagenumbering{roman} % Römische Seitennummerierung
	\normalfont	
}

%
% @stud
%
\newcommand{\settingLists}{
	%	Inhaltsverzeichnis
	%\begin{spacing}{0.95} %Change line spacing to fit table of contents on one page
		\tableofcontents
	%\end{spacing}

	%	Abbildungsverzeichnis
	\listoffigures
	%	Tabellenverzeichnis
	%\listoftables
	%	Listingsverzeichnis / Quelltextverzeichnis
	%\lstlistoflistings
	% Algorithmenverzeichnis
	%\listofalgorithms
}

\newcommand{\initializeText}{
	\clearpage
	\ihead{\chaptername~\thechapter} % Neue Header-Definition
	\pagenumbering{arabic}           % Arabische Seitenzahlen
	\onehalfspacing
}

\newcommand{\initializeBibliography}{
	\ihead{}
	\printbibliography[title=Quellenverzeichnis]
	\pagebreak 
	\pagenumbering{roman}
	\cleardoublepage
	\singlespacing
	
	
}

\newcommand{\initializeAppendix}{
	\appendix
	\ihead{\appendixname~\thechapter}
	\ihead{}
	\singlespacing
	
}

%EditMaxD
\newcommand{\setTodo}{
	\listoftodos[Todos]
	\newpage
}

%Defining Custom Colors
\definecolor{sapBlue1}{RGB}{0,185,242}
\definecolor{sapBlue2}{RGB}{1,156,224}
\definecolor{sapBlue3}{RGB}{12,126,207}
\definecolor{sapBlue4}{RGB}{22,97,190}

\definecolor{sapGold}{RGB}{240,171,0}